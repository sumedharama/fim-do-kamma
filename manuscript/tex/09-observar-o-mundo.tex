\chapter{Observar o Mundo}

\begin{quote}

``\ldots{}largar o mundo é paz. Não precisamos de nos agarrar seja ao que for e não precisamos de afastar de nós seja o que for.''

\href{https://suttacentral.net/snp5.12/en/sujato}{Sutta Nipāta 5.12}

\end{quote}

Quando visitei recentemente um mosteiro na China, conheci um monge idoso que me
deu como presente um valioso exemplar da sua caligrafia, que resumia o Dhamma,
tal como este monge o compreendia. Constava de dois ideogramas num rolo. A sua
tradução queria dizer: `Observa o mundo'. `Observa o mundo': existe aqui atenção
mas não envolvimento -- desapego. Contudo continuamos a observar o mundo, não a
ignorá"-lo -- isso implica compaixão.

\section{Interdependência}

O que é, na realidade, este mundo? Em termos sociais, psicológicos e ambientais,
trata"-se de uma rede na qual diferentes forças, energias e seres sustentam e
condicionam as respectivas existências. Isto nem sempre é tão bonito como
parece. Na combinação certa, a chuva e o sol sustentam a vida, mas existem
igualmente secas e inundações. Os leões e os antílopes existem em
interdependência -- matar e comer antílopes significa a sobrevivência dos leões,
mas significa também que o número de antílopes se mantém sob controlo, de forma
que estes não comem toda a erva, levando à extinção da sua própria espécie bem
como de outras. Mas se eu fosse um antílope, acho que não ia achar justo que um
leão me comesse -- `não fiz nada de mal, porque é que devo experienciar o medo e
uma morte violenta?' E, se eu fosse um leão: `porque é que não posso comer erva?
Porque é que tenho de correr atrás dos antílopes para comer? Porque é que eles
têm de correr tão rápido? Não é justo.' Contudo, o mundo não funciona de acordo
com a nossa opinião, mas sim de acordo com o facto `disto' ser dependente
`daquilo'. O `mundo' implica interdependência. E esta interdependência pode
significar um equilíbrio ecológico, mas também significa que o mundo do
nascimento e da morte, do medo e da fome, gira de forma incessante.

O Buddha referiu"-se a este mundo de existência dependente como algo que
experienciamos através dos nossos sentidos.\pagenote{``O mundo, perante a
  Disciplina d'Aquele que é Nobre, é o mundo no qual somos alguém que
  percepciona e concebe o mundo. E, amigos, o que é que é esse mundo no qual
  somos alguém que percepciona o mundo, que concebe o mundo? Os olhos são aquilo
  pelo qual somos alguém que percepciona o mundo, que concebe o mundo. Os
  ouvidos\ldots{} O nariz\ldots{} A língua\ldots{} O corpo\ldots{} A mente é
  aquilo pelo qual somos alguém que percepciona o mundo, que concebe o mundo. O
  mundo no qual somos alguém que percepciona o mundo, que concebe o mundo -- a
  isto chama"-se o mundo na Disciplina daquele que é Nobre.''
  \href{https://suttacentral.net/sn35.116/en/bodhi}{SN 35.116}}
De forma resumida, isso chama"-se `forma' (\emph{rupa}). Mas esta `forma' de
informação pura dos sentidos não pode ser experienciada separada da forma como a
nossa consciência a organiza numa realidade coerente. Assim, este mundo inclui
também o nosso mundo interno de sentimentos e interpretações, bem como a atenção
e a intenção através das quais a mente nomeia e organiza. Toda esta nomeação e
organização é resumida em `nome' (\emph{nama}).\pagenote{``\,`Nome' ou
  `Mentalidade' é composto por sensação (\emph{vedanā}), percepção
  (\emph{sañña}), intenção/volição (\emph{cetanā}), impressão (\emph{phassa}),
  atenção (\emph{manasikārā}).'' \href{https://suttacentral.net/mn9/en/bodhi}{MN
    9.54}; \href{https://suttacentral.net/sn12.2/en/bodhi}{SN 12.2}}
Para cada um de nós, o mundo de `nome e forma' surge dependente da consciência
que recebe a informação dos sentidos e do processo concomitante de significação
que organiza aquilo que é visto, tocado ou sentido. Observar o mundo desta
maneira deixa aberta a possibilidade de cada um de nós, através da purificação
dos nossos processos de `nomeação', podermos afectar a forma como vemos e
reagimos ao mundo.\pagenote{A origem e a cessação do mundo: ``E, bhikkhus, qual
  é a origem do mundo? Dependente dos olhos e das formas, surge a consciência
  dos olhos. O encontro dos três perfaz o contacto. Com o contacto enquanto
  condição, surge a sensação; com a sensação enquanto condição, surge o anseio;
  com anseio enquanto condição, o apego; com apego enquanto condição, a
  existência (devir); com a existência enquanto condição, o nascimento; com o
  nascimento enquanto condição, surgem o envelhecimento, a morte, mágoa, a
  lamentação, a dor, o desconforto e o desespero. Esta, bhikkhus, é a origem do
  mundo.''

  \emph{Acontecendo o mesmo para os outros sentidos, incluindo a mente. A
    cessação do mundo segue a mesma sequência até ao anseio, e seguidamente:}

  ``Com o restante desaparecimento e cessar desse anseio, surge a cessação do
  apego\ldots{} [até] com a cessação do nascimento, cessam o envelhecimento, a
  morte, a mágoa, a lamentação, a dor, o desconforto e o desespero.''
  \href{https://suttacentral.net/sn12.44/en/bodhi}{SN 12.44}

  O facto de poder cessar nesta vida sugere que este `nascimento' é mais do que
  um acontecimento histórico no passado -- trata"-se de um impulso aqui e agora,
  que procura estados de ser.}

Para ilustrar: ao `vermos', primeiro detectamos um objecto visual, depois
detemo"-nos nele enquanto a mente o reconhece, e define-o (através da percepção,
dum significado sentido) como `pessoa' ou `carro'. Dependendo disso e da
presente disposição ou intenção, surge uma resposta. Depois podemos demorar"-nos
mais sobre este objecto e desenvolver possibilidades e planos. Tendo tudo isto
em consideração, podemos sentir"-nos entusiasmados, assoberbados ou aborrecidos
com as energias e as emoções que surgem. Do mesmo modo, `o mundo' pode parecer
emocionante, pavoroso ou monótono.

Se tivermos uma atenção compreensiva para com o que vemos, colocamos de lado
outras interpretações de forma a antes de mais verificarmos se as possibilidades
e os planos que as nossas mentes criam sustentam a clareza, a bondade e a
ausência de tensão, ou não. Então podemos filtrar e separar o adequado do não
adequado, e olhar para os alicerces do nosso comportamento mental. Até que ponto
os nossos pensamentos e pressupostos relativos ao mundo e a nós próprios se
baseiam apenas no conjunto particular de atitudes e estados de espírito deste
momento? De que forma as minhas atitudes e estados de espírito dão uma
tonalidade diferente àquilo que surge? Como é que eu me encontro, continuamente,
a criar o meu mundo e a mim próprio como alguém inserido nele? Purificar a
atenção é tomar consciência e limpar estes enviesamentos, para nos libertarmos.

\section{O fim do mundo}

Este mundo é o reino interdependente da causa e do efeito. Num plano mais
íntimo, o `meu mundo' está enlaçado com designações e as consequentes respostas
-- trata"-se do processo do kamma. E isto continua \emph{ad aeternum}. O
lembrete do Buddha é que não alcançamos o fim das nossas convulsões enquanto não
atingimos o fim do nosso mundo.\pagenote{\ldots não atinge o final das suas
  provações enquanto não chegar ao fim do mundo\ldots{}

  ``\ldots digo que sem ter atingido o final do mundo não existe forma de colocar um fim ao sofrimento. Amigo, é precisamente neste mesmo corpo, dotado de percepção e de mente que eu torno conhecidos o mundo, a origem do mundo, a cessação do mundo e o meio que leva à cessação do mundo.'' \href{https://suttacentral.net/sn2.26/en/sujato}{SN 2.26}; \href{https://suttacentral.net/an4.45/en/thanissaro}{AN 4.45}}
Ademais, os ensinamentos afirmam que isto não acontece por andarmos no mundo,
fugirmos dele ou criarmos outro mundo, mas sim através da contemplação dos nomes
e das formas e penetrando a base a partir da qual estes surgem. Podemos ter
algum voto na matéria no que diz respeito ao modo como experienciamos o contacto
dos sentidos, em termos da maneira como nomeamos e como respondemos. Não temos
de procurar nem de nos absorvermos em tudo o que vimos ou ouvimos e, se a mente
se encontra centrada, não temos de reagir. Tudo depende do modo como cultivamos
as nossas mentes. Este cultivo constitui o kamma que acaba com o kamma, a
dobradiça que vira em direcção ao Despertar.

Tomemos um exemplo simples: podemos cultivar a reflexão sobre o modo como as
coisas acontecem e os seus resultados e ao fazermos isto, criamos propensão a
agir de maneira ética no mundo. Relativamente a isto, a capacidade dos outros
animais é inferior: os humanos podem desenvolver"-se radicalmente em termos de
generosidade, bondade e clareza. Apesar de podermos chacinar outras criaturas e
até seres humanos, de forma bastante impiedosa, criamos igualmente organizações
para cuidar dos doentes e dos necessitados -- incluindo outras criaturas. Uma
mudança, deste modo, envolve vivermos de acordo com valores de empatia: isto dá
às nossas vidas um significado mais valioso. A cooperação e a amizade dependem
destes valores. Certamente nenhum de nós teria sobrevivido ao nascimento sem uma
considerável ajuda por parte de outras pessoas. E agora, que estamos a partilhar
este planeta com um número progressivamente maior de outros seres humanos,
torna"-se cada vez mais importante afastar o conflito, ao desenvolvermos formas
de cooperação. Continuar a pensar simplesmente em termos de quanto é que eu e os
meus podemos ganhar, enquanto ignoramos ou prejudicamos os outros seres,
constitui uma ameaça à existência no planeta.

O caminho para o fim do mundo envolve lidarmos com as energias cármicas das
nossas mentes com a intenção correcta, o discurso correcto, a acção e o modo de
vida correctos, assim como na meditação -- em todos os aspectos do Caminho
Óctuplo. Grande parte dos ensinamentos do Buddha baseiam"-se em gerarmos bom
kamma na vida quotidiana: estabelecer, com determinação, a pureza da intenção
face ao mundo, constitui uma base para a libertação. Isto porque compreendermos
empaticamente as pessoas, os deveres e tudo o mais, proporciona bons resultados.
Sentimos que se produzirmos pureza e não pedirmos nada em troca, a nossa tensão
relativamente ao mundo termina. Dessa forma, podemos viver com respeito por nós
próprios e em equanimidade, pois já não nos encontramos presos nas armadilhas de
sucesso/fracasso e de elogio/culpabilização do mundo. Conseguimos realmente ver
para além destas armadilhas, o que nem sempre acontece quando só trabalhamos no
nível interno.

Na meditação, a exigência em melhorar, o tentar conseguir fazer bem, não é
necessariamente encarado como uma fonte de tensão. Mas se a avaliação
sucesso/fracasso não é identificada, ela é interiorizada como um crítico
incómodo, como um Tirano Interior que nos censura continuamente por não sermos
suficientemente bons; ou que exige uma recompensa espiritual -- `quanto tempo é
que me leva a conseguir \emph{isto}?' Aí mesmo, temos o mundo a surgir. E este
não acaba se o seguirmos. Em qualquer caso, interno ou externo, o fim da tensão
ocorre na medida em que conseguimos abandonar as nossas preocupações relativas à
imagem que temos de nós próprios. É isto que significa purificar a intenção.
Significa largar aquela busca autocentrada que nos leva a cair.

\section{O caminho da transcendência: pārami}

Porquê agir e porquê meditar, se não ganhamos nada com isso? Bem, agimos e
meditamos de forma a corrigirmos a nossa intenção, não para ganhar prémios. As
recompensas chegam por si próprias. Relativamente a isto, o cultivo persistente
do kamma em termos das `perfeições' (\emph{pāramī} ou \emph{pāramitā}) tem"-se
tornado um tema fulcral nas culturas budistas.\pagenote{Nos textos iniciais do
  Cânone Pali, o Budha não fala acerca destas \emph{pāramī}. As referências
  ocorrem nos livros posteriores do Cânone -- tal como as Jataka -- e no
  comentário clássico Visuddhimagga. Contudo, existem muito exemplos do Buddha e
  dos seus discípulos a praticarem a moralidade, a renúncia, a persistência e as
  restantes \emph{pāramī}. Os textos e a tradição Mahayana referem seis
  \emph{pāramitā}: generosidade, moralidade, paciência, energia, meditação e
  sabedoria, considerando"-as muito importantes e essenciais às práticas de
  Bodhisattva. O facto de surgirem tanto na tradição Theravāda como Mahayana
  parece sugerir que foram formuladas quando o Budismo inicial, do qual ambas
  evoluíram, se desenvolveu na Índia.}
As \emph{pāramī} são formas de gerar bom kamma de modo muito minucioso de forma
que, mesmo ao praticar no mundo, a mente é treinada a largar as atitudes e os
estados de espírito inadequados e a aprofundar"-se nos transcendentes. As
\emph{pāramī} ajudam"-nos a alargar a mente para uma consciência que é
desimpedida e luminosa. Por esta razão, `\emph{pāramī}' tem igualmente o
significado de `qualidades que atravessam'.

Na tradição Theravāda, as \emph{pāramī} estão listadas da seguinte forma:
generosidade, moralidade, renúncia, sabedoria/discernimento, persistência,
paciência, sinceridade, bondade, determinação e equanimidade. Todos elas fazem
com que o nosso coração produza respostas adequadas face ao que experienciamos,
tratando"-se de uma reacção que tem como objectivo a libertação. É bom recordar
que a libertação não é um estado `radical' -- apenas significa o caminho e o
fruto de largar a ganância, a aversão e a ilusão, bem como largar as bases a
partir das quais estas surgem.

A generosidade consiste na partilha e não se trata apenas de bens materiais.
Constitui toda uma atitude perante a vida. Partilhamos recursos de dinheiro,
tempo, competências e energia, devido à noção de inter"-relacionamento que
constitui o alicerce da vida. Encontramo"-nos juntos nisto, tudo tem um efeito
sobre tudo o resto: considere o bem"-estar dos outros como gostaria que estes
considerassem o seu -- isto constitui a transcendência da opinião própria a um
nível mundano. E quando algum bom kamma surge a partir deste sentido de
interdependência, ao invés de resultar de alguma base moralista de `deves ser
boa pessoa', o resultado é que nos sentimos bem. De modo que todos ficam a
ganhar. Aquele que dá sente alegria e o que recebe sente os efeitos da bondade.
Através da prática das \emph{pāramī} surge um sentimento de valor inato
(\emph{puñña}) -- em nós, nos outros, mas principalmente na bênção da acção
adequada. Isto vai contra a noção ocidental `anti"-ego', de acordo com a qual
não nos devemos sentir bem relativamente às nossas acções. Mas não nos sentirmos
bem devido às nossas acções apenas vai gerar um ego cínico e triste. A partir de
uma perspetiva budista, a libertação em relação à `nossa pessoa' constitui um
passo posterior, que se baseia na confiança relativamente às nossas acções e
intenção. Sem este sentimento de confiança, não temos uma fundação
verdadeiramente sólida.

Todas as \emph{pāramī} seguem esse princípio. Falam por elas próprias. A
moralidade leva ao respeito por si próprio e à confiança nas pessoas que nos
rodeiam. A renúncia tira"-nos das garras das energias materialistas que
controlam a maior parte da sociedade. A sabedoria corta através do nevoeiro das
sensações, para nos dizer aquilo que é ou não adequado. A paciência de não nos
apressarmos e de permitir que as coisas se movimentem a um ritmo harmonioso, é
excelente para estados de espírito obstinados: `tenho de ter isto feito'. Há
toda uma vida de cultivo apenas nesta \emph{pārami}.

Nem sempre estas \emph{pāramī} se encontram visíveis no mundo, nem o seu cultivo
significa que nos tornemos um sucesso em termos mundanos. Não é particularmente
provável que o/a leitor/a se torne líder de um partido político ou de uma
companhia com implantação a nível global através deste kamma. Mas quem sabe? Um
amigo disse"-me, há uns anos atrás, ter jurado apenas negociar honestamente com
os clientes: nada de falsas promessas, nada de favores, nada de fuga às
obrigações legais. Ao princípio o seu negócio teve um ligeiro declínio, mas
passado algum tempo, à medida que as pessoas compreendiam que podiam confiar na
sua palavra, começaram a preferir aquela forma sincera de negociar e o seu
negócio aumentou. O negócio com ética pode fazer sentido. De qualquer forma,
podemos sempre ganhar em termos de respeito por nós próprios, de uma consciência
tranquila e de amigos nos quais podemos confiar. Quando a economia tem um
colapso ou a nossa saúde falha, quando sofremos uma perda ou somos acusados de
algo, saber como viver de forma simples e ser uma testemunha equânime das
experiências da vida são uma verdadeira salvação.

\section{Tendências latentes}

No seu todo, a prática das \emph{pāramī} estabelece valores poderosos que
dirigem, de uma forma adequada, a intenção pessoal. E existem benefícios. As
atitudes e energias que vão ao encontro da falsidade, da malícia e da cobiça são
menos nutridas. E à medida que a nossa atenção se corrige, a forma como vemos o
mundo é alterada, pois quer a razão quer a intenção subjacente ao tocarmos ou
olharmos para algo, tem um papel preponderante na forma como se designa a
experiência. À medida que isto se vai purificando, vemos as coisas de um modo
que desemaranha o nosso mundo. Por exemplo: em vez de olharmos para a vida em
termos daquilo que podemos lucrar com ela, podemos olhar sob o ponto de vista
daquilo que podemos dar; em vez de nos interrogarmos quanto tempo vai demorar,
podemos valorizar a paciência e a determinação. Então, em vez de especularmos se
somos admirados ou ignorados, estabelecemo"-nos, de forma serena, na consciência
da nossa integridade. De modo que a nossa `nomeação' do mundo sofre uma
alteração pois passamos a (re)designar este como um veículo para a valorização e
libertação, em vez de uma viagem numa bola saltitona de `eu/eles',
`ganhos/perdas'.

Quanto mais nos apoiamos nos valores de uma vida adequada, mais os processos
purificadores irão revelar disposições e tendências latentes por resolver. Estas
tendências latentes (\emph{anusaya}) incluem inclinações básicas tais como
sensualidade, irritação, opinião pessoal e presunção, as quais, na vida do
quotidiano, podem não ser reveladas como tal pois a forma como funcionamos evita
uma investigação aprofundada das nossas tendências. Por isso pomos em prática a
intrepidez. Assumimos compromissos relativos a actos de valor e de integridade,
particularmente quando as coisas não se encaminham de acordo com os nossos
desejos.

Relativamente a este último aspecto, a prática budista não se trata de ter
momentos altos. Trata"-se de um treino. Trata"-se de fortalecer e de alargar o
compromisso, de forma a manter os padrões e as virtudes, mesmo quando os
momentos altos não acontecem e aquelas tendências que não reconhecemos estão a
crescer. Um ambiente de treino frequentemente utilizado é o de convívio com os
outros. Em muitos mosteiros budistas, bem como no mundo em geral, passamos muito
tempo na companhia e a trabalhar com outras pessoas. Com frequência,
construímos, gerimos, lavamos a loiça, tal como conversamos e meditamos, em
situações de grupo. A mente é deste modo mantida num mundo partilhado, algo que
não se pauta apenas pelas disposições e as energias individuais. Através disto
vemos que os nossos processos de `nomeação' diferem: as nossas interpretações
daquilo que é normal ou amigável, as nossas atitudes relativas à liderança e à
independência, a nossa sensibilidade face às outras pessoas, etc. Ver e
responder a isto significa que temos de gerar uma grande dose de paciência, de
bondade e de compromisso, de modo a limpar as nossas ideias pré"-concebidas. O
objectivo nem é termos uma comunidade maravilhosamente harmoniosa, mas sim a
afrouxar o apego à nossa própria `nomeação'. É este afrouxamento que proporciona
à mente o espaço e o encorajamento para ir além dos pontos de vista habituais.

Gosto desta abordagem integrada, principalmente porque não comecei neste tipo de
perspectiva. No mosteiro tailandês onde iniciei o meu treino como monge budista,
existia uma secção à parte dedicada à prática intensiva de meditação. Neste
mosteiro, os monges iam para esta secção para rever e aprofundar a sua
compreensão do Dhamma. Em geral, passavam aí umas semanas e depois regressavam a
onde estavam integrados. Eu era um dos poucos ocidentais: éramos três ou quatro
e todos novatos (no que dizia respeito à Tailândia, à meditação e à vida
monástica) e não tínhamos fosse o que fosse para fazer, nem convívios nem
qualquer sítio onde ir para além do mosteiro. Não eram permitidas conversas.
Como o/a leitor/a calcula, estar todo o dia numa cabana pequena a tentar meditar
e a ver a mente a saltar os muros do mosteiro durante horas a fio pode ser fonte
de grande tensão. A única coisa que fazíamos em conjunto era ir à mendicância,
em silêncio, todas as manhãs. Era a única ocasião do dia em que estávamos
juntos. Deveria ser fácil, simplesmente andar e receber as oferendas, mas,
contudo, surgiam todo o tipo de coisas, particularmente coisas que não faziam
parte do guião da Iluminação.

A primeira pessoa na minha vida que disse que gostaria de me matar, se possível
com um machado, foi um companheiro monge. Bem, eu andava a um ritmo que ele
achava demasiado lento, enquanto ele tinha de andar atrás de mim\ldots{} No que
me diz respeito, não me lembro de ter muitos impulsos violentos até me tornar
num sério monge meditador\ldots{} Mas podia ter sentimentos violentos
relativamente ao monge que seguia atrás de mim naquela fila. Afinal de contas, o
Buddha afirmou que devíamos andar calmamente, silenciosamente, de forma a
podermos manter"-nos calmos e concentrados para sermos iluminados -- mas todos
os dias aquele monge atrás de mim ia continuamente a pigarrear à medida que
caminhávamos\ldots{} Isto é justificação para um assassinato, não é?

Naturalmente não agíamos de acordo com estes impulsos -- deixávamos que
passassem. O que constituiu um pouco de Despertar. Havia bom kamma suficiente
para existir um sentido forte de moralidade e consciência. Mas a ideia de que
não temos má vontade apenas porque na solidão não está seja quem for a
provocar"-nos, caiu por terra. Assim, no contexto da prática, o impulso violento
foi útil: tive de largar a ideia de mim próprio como um indivíduo razoável, de
fácil convivência, e concentrar"-me na tendência para a má vontade. E, para além
disso, quando reconheci que a prática solitária não tinha tornado mais fácil a
partilha deste planeta, por um par de horas, com outro ser humano inofensivo que
partilhava o meu interesse no Despertar\ldots{} percebi que o paradigma do
cultivo da mente tinha de sofrer uma alteração. Comecei a compreender que não
saímos do kamma por o evitarmos.

Numa sucessão de situações e depois de três anos de prática em solidão, voltei,
em visita, a Inglaterra. Parei em Londres, onde o Ajahn Sumedho dirigia um
pequeno grupo de monges. Nessa comunidade havia maior enfâse na acção e
interacção, bem como uma formação sobre como costurar e cuidar dos nossos mantos
monásticos, como acompanhar e assistir um professor e muito outros pequenos
assuntos protocolares. Pareceu"-me também uma forma de vida mais afável --
apesar de as pessoas por vezes ficarem agitadas. Mas a ideia global era sermos
conscientes em relação a tudo o que a mente trazia à superfície e investigar
onde se situava o sofrimento. Parecia uma forma de alargar e integrar a minha
prática, e por isso fiquei.

\section{Quatro bases do apego}

O Buddha referiu"-se ao apego como tendo quatro níveis sucessivamente mais
profundos: apego aos objetos dos sentidos, às regras e costumes, às opiniões e
às impressões relativas àquilo que somos.\pagenote{Estas quatro bases do apego
  constituem o tema de \href{https://suttacentral.net/mn11/en/bodhi}{MN 11}.}
O primeiro é relativamente óbvio: agarrarmo"-nos ao que possuímos,
`alimentarmo"-nos' do que vemos, ouvimos e tudo o mais. No mosteiro, com a
limitação de estímulos sensoriais e com uma carga razoável de trabalho físico, a
maior parte desta intensidade centra"-se na refeição diária ou na bebida quente
ou eventuais doces à hora do chá. A própria energia do apego ao significado
sentido de sermos alimentados, por vezes, gerava tanta energia no nosso sistema
que termos uma refeição serena depois de esperarmos pacientemente que todos se
reunissem, seguirmos pacientemente na fila para receber a comida, esperarmos
pacientemente que todos regressassem\ldots{} depois os cânticos\ldots{} depois
de esperarmos que a pessoa mais antiga começasse a comer\ldots{} -- constituía
um feito considerável! Na realidade a comida não era nada de especial -- por
vezes mal reparava o que era. Para além disso, o grau de satisfação resultante
de comer não era fantástico e era mais ainda descompensado pela sensação de
entorpecimento resultante. A paixão situava"-se em torno da ideia, do
significado sentido de comer. Tanto o estado como a natureza apelativa da comida
podiam alterar"-se em minutos. Não só o apego era doloroso como, ao contemplar
tudo isto, tornava"-se óbvio que a intensidade se localizava somente em torno do
conjunto de sentimentos e pulsões que este `agarrar' tornava sólidos e reais,
durante algum tempo. O `agarrar' era apenas o apego às suas próprias
designações. Foi uma descoberta interessante, também para continuar a trabalhar:
de repente o celibato começou a fazer muito mais sentido.

Eu conseguia experienciar o mesmo apego em termos do segundo nível de apego: no
que diz respeito às regras e aos costumes do mosteiro. Todos usamos regras e
costumes para regular as nossas vidas ou profissões: formas de etiqueta, que
comida ingerir (e a que horas e dia da semana), o modo como gostamos que o nosso
escritório esteja arrumado, deveres religiosos e tabus sociais. Mas existe uma
tendência para automatizarmos ou para nos tornarmos dogmáticos relativamente ao
nosso sistema. Um sentimento que eu tinha em relação a abraçar a prática budista
tinha na verdade a ver com escapar a isto e ser mais espontâneo, viver no aqui e
agora. Mas depois de três anos sem nada que fazer, nada a que pertencer e,
consequentemente, sem nada acerca do qual ser espontâneo, passei a apreciar
verdadeiramente coisas tais como: os cânticos matinais diários, os deveres
relativos à forma como usávamos e lavávamos a nossa tigela de mendicante, toda a
formação relativa às convenções (que eu anteriormente tinha perdido), etc. Tudo
isto ajudou a manter"-me focado na vida do quotidiano e no desenvolvimento das
\emph{pāramī}. Assim que se tornaram familiares, as regras e os costumes
proporcionaram um sentimento de familiaridade, de segurança.

O mesmo acontecia com o sistema de meditação que estava a usar -- mesmo se eu
nem sempre fosse bom, a ganhar ou a perder, a meditação definia onde me situava.
Sentia"-me sólido, seguro. Mas depois havia uma tendência muito forte, à volta
disto tudo, para me tornar ainda mais sólido, para fazer parte de um sistema
altamente disciplinado e para ser alguém que conseguia sentar"-se sólido que nem
uma rocha, com uma inabalável consciência da respiração. E, juntamente a isto,
insinuava"-se uma condescendência subtil relativamente às pessoas que não eram
tão sólidas, ou que não conseguiam acompanhar o ritmo, bem como uma rejeição
absoluta daqueles do tipo `no aqui e agora' que claramente não possuíam qualquer
sentido de determinação.

Contudo, Ajahn Sumedho, o líder daquilo que era suposto ser uma equipa especial,
ocasionalmente cancelava actividades, se achava que as pessoas estavam a ter
dificuldades ou para simplesmente fazer uma pausa. Por vezes era apenas para
vermos o que as nossas mentes faziam com isso. (O que era interessante\ldots{})
Depois, Ajahn Sumedho diminuiu a intensidade de alguns dos deveres -- de
madrugada, permitiu uma taça de papas de aveia, porque algumas pessoas não se
sentiam muito bem\ldots{} E, apesar de Ajahn Sumedho passar bastante tempo em
meditação, usava uma técnica que para começar não ia para além de uma
concentração básica na respiração; geralmente o tema principal consistia em
largar. Foi uma reviravolta completa no meu sistema, de modo que foi tudo muito
confuso para mim durante mais três anos (e vinte e cinco anos depois continuo a
trabalhar nisto), mas ao mesmo tempo percebia que esta abordagem era muito
directa e que lidava com o que era importante. Largar o apego. Sim, dá para
reconhecermos que agarrarmos um sistema, enraizarmo"-nos nele e tornarmo"-nos
fanáticos relativamente ao mesmo, traz o mesmo tipo de sensação e de paixão que
podemos criar relativamente a uma barra de chocolate. É apego\ldots{} e quer
dizer que estamos prestes a sofrer. E provavelmente a infligir algum sofrimento
em mais alguém.

Algo muito semelhante ocorre na camada seguinte de apego: o apego a opiniões
classificadas enquanto opiniões de `tornar"-se e não se tornar' (estas são
projecções que extraímos da experiência directa do presente). Existe a tendência
para ou evocarmos um futuro ou recusarmos a tê"-lo em consideração; definirmos
um objectivo e uma trajectória na vida ou negarmos que haja qualquer propósito
nesta; envolvermo"-nos na construção e no desenvolvimento das coisas ou
afirmarmos que não há nada a fazer, que tudo é impermanente e que por isso não
vale a pena. Este é o enviesamento subjacente de `tornar"-se/não se tornar' que
cria mundos que tomam uma forma sólida e começam a girar, podendo fazê"-lo com
grande convicção e paixão. Esta é a paixão e o apego que inicia o kamma,
primeiro na mente, depois no discurso, e por aí adiante. Mas não conseguimos
sair do processo de causa e efeito ao tentarmos que tudo fique concluído, sólido
e real pois assim que o kamma é seguido, continua a estabelecer novos
objectivos. Mas afirmar que não existe qualquer objectivo, que é tudo vazio e
que não nos importamos com o futuro, também tem os seus efeitos. O fracasso em
considerar a causa e efeito afecta a nossa acção no mundo de forma definida.

Mesmo as nossas tentativas no sentido do Despertar podem seguir estes
enviesamentos subjacentes. Trata"-se de termos a Experiência Derradeira de
Imortalidade, ou pelo menos algumas experiências gratificantes; ou trata"-se da
Cessação Final, do \emph{Nibbāna}? Seja como for, o apego a estas ideias surge
de perspetivas fundamentais que imaginam alguma `Base Atemporal de Ser' ou um
`Esquecimento Beatífico'. E estes dependem do facto da nossa opinião sobre nós
próprios ter tendência para a infinidade ou para o desaparecimento.
Provavelmente oscilamos entre os dois, dependendo de nos sentirmos entusiasmados
ou saturados, ou apenas de acordo com a flutuação das nossas energias. Mas em
qualquer dos casos, baseamos as nossas opiniões num sentido de nós próprios, que
se situa no âmago de tudo isto. Claro que não faz sentido, porque o enviesamento
subjacente varia: num determinado momento queremos estabelecer contacto e
experienciar e, no seguinte, queremos afastar"-nos. Quem é que está a fazer
isto?

No fundo tudo se resume a isso: o último nível, o de apego ao sentido de nós
próprios. Agarrarmo"-nos à tendência de nos tornarmos algo gera um sentido de
nós próprios. Mas esta sensação, em si própria, é uma designação que ocorre na
mente. E está em mudança constante: de confiante e descontraída, a ansiosa e
tensa. Tenha atenção a isto: à medida que o apego afecta a mente e intensifica a
sua paixão em relação às formas, pensamentos ou universos, os sentimentos e
impressões daí resultantes consubstancializam"-se num sentido de si próprio, que
é o agente ou a vítima do mundo. E esse mundo, quer seja uma sublime e imaterial
Derradeira Realidade, ou a tradição budista pura e autêntica, ou o mundo
ignorante e injusto da geopolítica, é então encarado como sendo muito sólido. E
daqui resulta uma paixão e uma tendência relativamente ao mundo. Deste modo
estabelecem"-se os padrões que são sentidos como `eu'. Então o bem fica manchado
pelo orgulho e pela presunção, enquanto a negatividade gera desânimo ou
irritação. Existe muita oportunidade para o sofrimento e um sem fim de eventos
que decorrem à volta disto. Mas, em termos essenciais, o `sentido de si próprio'
e o `mundo' surgem de forma independente, como dois extremos do mesmo processo
de designação. O `eu' não consegue libertar"-se. Contudo, através da libertação
relativamente ao apego, o sentido de si próprio e o mundo não surgem.

Assim, as quatro bases dão"-nos janelas através das quais podemos contemplar o
apego. Porque, em si próprias, a comida e tudo o resto que é material, é útil;
as regras e os costumes são guias úteis e as opiniões proporcionam um foco para
o trabalho com significado no momento presente. De igual forma, algum sentido de
si próprio, alguma referência relativa às nossas próprias energias, tendências e
competências, é essencial para fazermos bem as coisas. Mas existe igualmente uma
necessidade de observar e de conter a paixão e o apego a tudo isto. É este o
objectivo do cultivo das \emph{pāramī}. Mas não resolvemos e limpamos a mente
apenas com as \emph{pāramī}. A resolução e limpeza das tendências da ignorância
e do `tornar"-se' apela aos factores do Despertar (\emph{bojjhaṅgā}).

\section{Factores do Despertar: o trabalho de libertação}

Aquilo que, como maior frequência, é mais perturbador nestas tendências, em
particular quando ocorrem na meditação, é que surgem como estados que se situam
para além do nosso controlo, sob os quais nos tornamos numa pessoa diferente
daquilo que julgávamos ser. Podemos ser inundados por uma raiva infantil ou ser
vítimas de medos irracionais. Isto resulta do facto de as tendências se
encontrarem enraizadas no nível dos reflexos, um `local' psicológico que
antecede a nossa personalidade. Até um bebé tem isto.\pagenote{``\ldots um
  delicado bebé deitado de bruços nem sequer tem a noção de \textquotesingle
  personalidade\textquotesingle, por isso como é que a noção de personalidade
  poderia surgir nele? Contudo, a tendência subjacente para a noção de
  personalidade (\emph{sakkāyadiṭṭhi"-anusaya}) encontra"-se nele.''
  \href{https://suttacentral.net/mn64/en/bodhi}{MN 64.3}}
Assim, cultivar as \emph{pāramī} desenvolve a personalidade até atingirmos um
ponto no qual podemos ter escolha relativamente a actuarmos em função destas
tendências ou não; contudo estas, em si próprias, permanecem na mente como um
potencial pré"-pessoal.

Pode igualmente acontecer que nem sempre somos claros sobre quais as tendências
que permanecem latentes e por resolver. De forma mais óbvia, a ignorância, a
perda de uma consciência estável, constitui uma tendência fundamental acerca da
qual, por definição, não somos claros. De modo que nos podemos sentir bastante
equilibrados e descontraídos\ldots{}, mas depois, ao interpretarmos uma ameaça
ao nosso território (ou sentirmos uma perda de estatuto) salta a má vontade, a
presunção, as opiniões rígidas\ldots{} e por aí adiante.

Aquilo que é necessário é uma referência, uma base para a acção e para a
introspecção, algo para além do cenário `eu sou a minha mente'. As \emph{pāramī}
dão"-nos uma forma de lidar com isto pois fazem com que testemunhemos, com que
nos desloquemos no sentido oposto e vivamos para além das resistências e das
paixões da mente. Então, se ficarmos concentrados no local onde a mente se
liberta, ocorre uma sensação de descontração e de largueza de espaço. Temos
então um vislumbre do não apego, sentimo"-lo como real e agradável. Então
interrogamo"-nos se isto é que `é', ou porque é que pura e simplesmente não
permanecemos aí\ldots{} mas assim que este largar é sentido como um estado,
então dá"-se uma percepção de ser ou de ter um `eu' que não se apega\ldots{} até
que a próxima vaga de sofrimento nos desperta novamente. Assim, a não ser que
desistamos da `nomeação', a partir da qual surgem todos os estados, irão
permanecer por resolver todas as tendências latentes para a dúvida, para as
opiniões, para o `devir' e para a `identificação'.

Então o que é necessário é discernirmos o que está na base do `nome', através da
meditação ou, mais especificamente, através dos factores do Despertar:
consciência, investigação, persistência, êxtase, tranquilidade, concentração e
equanimidade. Estes fazem surgir a pureza da intenção, uma estabilidade forte e
rica, para fazer face ao impulso do apego que se forma a partir da ignorância e
da perda de uma presença consciente e clara.\pagenote{Estes factores do
  Despertar são, eles próprios, uma forma de kamma:

  ``E o que é o kamma que nem é escuro nem luminoso, sem resultados escuros ou
  luminosos, que conduz ao final do kamma? A consciência, enquanto factor para o
  Despertar, a investigação de qualidades\ldots{} a persistência\ldots{} a
  êxtase\ldots{} a tranquilidade\ldots{} a concentração\ldots{} a equanimidade
  como factor para o Despertar.''
  \href{https://suttacentral.net/an4.238/en/sujato}{AN 4.238}}

Na prática isto resume"-se a manter a consciência e a investigação relativamente
ao aparecimento e ao desaparecimento da sensação, da interpretação ou da
intenção. Quando vemos a forma como surgem e como se condicionam mutuamente,
podemos compreender: tudo isto é causa e efeito. Tudo isto é kamma, não é meu
nem seu. E, à medida que através da persistência a nossa prática de meditação se
estabiliza, a actividade do apego começa a salientar"-se. Isto acontece porque
já temos uma referência relativamente à tranquilidade, ao espaço e ao silêncio
interno, que nos permite conhecer o apego através do modo como o sentimos (um
certo aperto no corpo/na mente) bem como reconhecer as vozes da justificação
moral, da autopiedade e da manha, através das quais ele se pronuncia. À medida
que o contemplamos, ficamos com a ideia básica que o apego não tem dono: `eu'
surge como o resultado do momento de apego, não antes deste. Não decidimos:
`hoje vou apegar"-me, vamos a ver quanto sofrimento consigo criar a mim
próprio'. De forma que não se trata de `eu ter uma quantidade de laços de apego
e me apegar muito' -- acontece apenas que a origem do apego não foi ainda vista.
O apego constitui uma acção, não uma pessoa. Essa compreensão encoraja"-nos a
encontrar uma forma para pararmos esse reflexo de `agarrar'.

\enlargethispage{\baselineskip}

O êxtase e a tranquilidade são importantes para descontrair esse agarrar. Estas
qualidades ajudam a aliviar uma das grandes questões do sentido de nós próprios:
será que consigo sentir"-me bem? Mas vão para além de uma pequena tranquilidade:
a forma como ocorrem na meditação descontrai igualmente o sentido de mim próprio
enquanto fazedor. Não fazemos o êxtase e a tranquilidade -- estes surgem"-nos
quando a mente se aquietou no seu tema de meditação. A experiência é semelhante
a um barco que está encalhado na areia: à medida que a maré sobe, primeiro há um
toque suave de alguma qualidade que nos eleva, depois gradualmente as coisas
começam a oscilar devagar até que o barco se encontra a flutuar. Mas continua
intacto. De modo que podemos largar o `sentido de mim próprio como fazedor', sem
termos de ter o `sentido de mim próprio como rígido' ou o `sentido de mim
próprio como colapso'. E depois: como é que sinto isto? Como é que esta
qualidade de abertura ou de não"-envolvimento pode ser fomentada? A
tranquilidade dá"-nos o conforto e a sensibilidade para olharmos mesmo para
isto. Existe a necessidade de desenvolvermos isto de forma a que as psicologias
baseadas no `eu a tentar' desistam. Há confiança no processo da meditação.
Assim, à medida que as tensões a nível mental e somático diminuem, surge
\emph{samādhi} para unificar as energias físicas e mentais. Com esta
sensibilidade estável, o kamma velho da defesa e da tensão pode ser libertado.

\enlargethispage{\baselineskip}

Esta libertação ocorre tanto a nível energético, como psicológico ou emocional.
Dito de outra forma, não se encontra apenas dependente da atitude ou da
compreensão -- depende igualmente de não ficarmos presos no ímpeto dos padrões
habituais. As psicologias do `eu' (a ansiedade relativa ao que sentimos estar à
nossa volta e para além de nós, bem como a necessidade de termos as coisas sob
controlo) formam um padrão de energia contraída, tanto no corpo como na mente,
sendo que essa energia constitui um bloqueio ou exerce um impulso. A nossa
reactividade é incorporada. Baseia"-se em algo anterior ou subjacente às nossas
atitudes emocionais, a um nível involuntário no qual nós `não somos nós
próprios'. De modo que envolve factores que não dizem respeito a que eu tente ou
a que eu faça algo para penetrar algo -- é necessário chegar a este potencial
pré"-pessoal nas terminações nervosas e recuar face ao impulso do kamma. É por
isso que precisamos de todos os factores do Despertar, não apenas de alguma
consciência ou compreensão.

É o poder que o \emph{samādhi} inculca na consciência que pode manter a calma e
a tranquilidade a este nível reflexivo; depois a equanimidade, o último factor
do Despertar, mantem um espaço desapaixonado. Trata"-se de mais do que um
controlo das reacções emocionais -- a equanimidade, enquanto factor do
Despertar, implica sentirmos a quietude e permanecermos nela sem a possuirmos. A
intenção é liberta de fazer seja o que for, de reclamar seja o que for ou de
formar uma imagem conceptual da quietude. É como se a mente fosse a própria
quietude. Este é o ponto de entrega, onde nada precisa de ser dito ou feito. Mas
se a quietude é sentida como um estado que `eu tenho', em vez deste abandono,
então aí a consciência designa-o como `agradável, desejável'\ldots{} e dá"-se
uma cristalização, e o `eu' forma"-se de modo a manter esse estado. Mas por essa
altura o conforto e o espaço já se desvaneceram e, à medida que as contracções
psicológicas começam a ter efeito\ldots{} surge o sofrimento.

De forma que não é apenas um apego à informação dos sentidos ou aos sistemas ou
às opiniões que precisa de ser dissolvido: trata"-se de dissolver a base. Os
factores do Despertar levam"-nos a dissolver esta premissa de ignorância. Eles
Despertam a intenção que não apoia essa base de `nomeação' da mente. Esta
renúncia por completo a ter sido, tido ou sabido fosse o que fosse. É aqui que
acaba o mundo. Não há mundo, nem aqui nem além: o fogo de tais designações
estingue"-se.

\section{Caminho-conhecimento}

Uma das consequências de tudo isto é o conhecimento do Caminho -- a consciência
com discernimento da Génese Dependente. Ou seja, existem estados de aparecimento
dependente que levam ao sofrimento, à consolidação do mundo e `eu'; e existem
estados de aparecimento dependente que levam à libertação. Estes últimos surgem
de acordo com o Dhamma (\emph{dhammatā}).\pagenote{Um exemplo de \emph{dhammatā}
  encontra"-se em \href{https://suttacentral.net/an11.2/en/thanissaro}{AN 11.2},
  onde o Buddha delineia uma sequência de práticas, que iniciam com o treino
  moral e culminam no Despertar. A consciência moral e o cuidado iniciam, de
  igual forma, a sequência causal de factores que conduzem à libertação em
  \href{https://suttacentral.net/an7.65/en/sujato}{AN 7.61}. Consultar também:
  \href{https://suttacentral.net/an7.81/en/sujato}{AN 7.81};
  \href{https://suttacentral.net/an10.1/en/bodhi}{AN 10.1};
  \href{https://suttacentral.net/an10.2/en/bodhi}{AN 10.2}}
Mesmo a realização surgiu `de acordo com o Dhamma' -- depende de factores do
Despertar tais como a consciência ou a concentração. E quando determinados
enviesamentos causais são removidos\ldots{} é aí que o mundo não surge. De forma
que o caminho do conhecimento oferece a visão integradora da causa e do efeito,
bem como de onde a causa e o efeito cessam. Na fruição do Despertar existe um
fim para o kamma.

Mas, de modo a integrar esta renúncia da opinião própria em termos de acção, o
caminho contínuo das nossas vidas envolve manter e dar valores espirituais de
forma a vivermos em harmonia com o mundo e a beneficiá"-lo. Reconhecemos os
benefícios de viver numa sociedade que valoriza a vida e a liberdade de escolha.
Reconhecemos os pais que nos mantiveram vivos e intactos em termos psicológicos,
durante anos. Reconhecemos a grande dádiva dos ensinamentos e de termos um
professor. Será que o cadinho para a libertação estaria preparado, sem termos
tomado os preceitos e assumido um compromisso para com uma convenção e prática?
Sem treinar a mente na meditação, será que ocorria a reacção química que
transmuta o kamma na libertação? Assim, o kamma adequado prepara o caminho para
que a causa e o efeito se dirijam ao transpessoal, transcendente.

O que surge é o desejo de servir. Ter consideração para com o mundo e para com a
cura do seu sofrimento -- isto é compaixão. Ter esta consideração de modo
constante -- isto é desapego. Em conjunto eles conseguem lidar com o surgimento
e o final do mundo. Esta é a acção dos Despertos.
