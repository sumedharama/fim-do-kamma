\chapterNote{Encontrar-se com o seu mundo}

\chapter{Meditação}

\tocChapterNote{Encontrar-se com o seu mundo}

Estabeleça uma presença corporal que lhe dê uma sensação de suporte: uma sensação de estar direito, um eixo que se centra à volta da coluna vertebral. Conecte-se com o chão e com o espaço situado acima e à volta do seu corpo. Tome consciência de se encontrar sentado dentro de um espaço, levando o tempo e o espaço que necessita para se estabelecer. À medida que se aquieta, deixe que os seus olhos se fechem suavemente. Sintonize-se com as sensações físicas através das sensações da respiração: primeiro na barriga, permitindo que a respiração desça através dos tecidos moles... sinta a inflexão da respiração reflectida no natural largar e firmar do abdómen ao respirar.

Esteja ciente do tronco, abrindo os tecidos de ligação situados entre este e a parte superior dos braços, ao deixar cair conscientemente os ombros... permita que a respiração flexione subtilmente o peito. Abra a cabeça ao descontrair os maxilares e pousar a língua na parte inferior da boca. Descontraia a zona em torno dos olhos, a testa e as têmporas. Deixe que o pescoço fique livre e a garganta aberta, como se tirasse um lenço ou desabotoasse um colarinho. Sinta a respiração a mover-se através da garganta, desde a base do pescoço, até à região posterior da boca e, para o exterior, através do nariz e da boca. À medida que estabelece esta referência física, aquiete-se nela, levando a mente apenas periodicamente a estes pontos específicos. Se se sentir instável -- com uma agitação perturbadora, uma quebra da energia ou um esmorecer da disposição -- leve a sua atenção pela coluna abaixo até ao chão, permitindo que a parte anterior do corpo flexione livremente com a respiração. Se sentir-se agitado ou nervoso, sintonize-se com a `respiração descendente', para baixo através do abdómen. Se sentir-se abatido ou entediado, sintonize-se com a `respiração ascendente', para cima, através do peito e da garganta.

À medida que atinge uma sensação de equilíbrio, pense numa situação actual da sua vida. Pode acontecer que, ao interrogar-se: `o que é importante para mim neste momento?' ou `com o que é que actualmente me debato?', lhe ocorra um cenário indicativo. Pode tratar-se de algo do trabalho, ou relacionado com amigos próximos ou familiares, sobre o seu bem-estar ou o seu futuro. Fique apenas com a impressão geral disso, sem entrar na história completa. Isso pode despoletar uma agitação de possibilidades antecipadas, ou mesmo um sentimento pesado de não ter qualquer escolha; pode ser `tanto que fazer...' ou `preciso mesmo disto', ou `e depois faço isto e depois aquilo'. Tente apanhar e destilar o sentimento emotivo: sobrecarga, antecipação, agitação, seja o que for. À medida que se torna mais claro, sinta a energia e o seu movimento (mesmo que não consiga colocá-lo em palavras). Por exemplo, é uma sensação de aceleração, de leveza, de tontura ou de bloqueio? Continue a despoletar esse resultado ao trazer esse cenário ao pensamento, até sentir que já distinguiu a sua tonalidade.

Então contemple esse resultado em termos físicos. Repare se, por exemplo, sente um rubor na face ou em torno do coração, ou um aperto no abdómen, ou uma tensão subtil nas mãos, nos maxilares ou em torno dos olhos. Se o assunto é muito evocativo, poderá sentir uma perturbação e de seguida uma tal inundação de pensamentos e emoções que perde a consciência do corpo.

Se isto acontecer, abra os olhos, expire e inspire lentamente e espere que as coisas aquietem novamente. Então, à medida que se volta a conectar com ou sustem a consciência do corpo, sinta novamente esse efeito emocional... que área do corpo é afectada? E, à medida que se concentra no resultado físico, que disposição é que isso traz à superfície? É positiva, algo que cria uma antecipação, de modo que o corpo fica com a sensação de se elevar e abrir? Ou é negativa, acompanhada por um afundamento ou aperto no corpo? Seja o que for, crie um espaço consciente à volta da experiência: consegue ficar presente com isto durante uns momentos?

Deixe fluir a consciência de `estar presente com', sinta plenamente a tonalidade dessa experiência. Pode tornar-se uma imagem -- tal como uma corrente brilhante de água, ou algo escuro e pesado, ou algo torto e encalhado. Pergunte-se: `O que é que isto me parece ou como é que estou a sentir isto neste momento?' Depois à medida que se aquieta durante alguns segundos, pergunte-se: `O que é que isto precisa?' ou `O que é que isto quer fazer?' Siga com atenção tudo o que acontece a esta noção de estabelecer contacto, ou de se afastar, ou de tensão. Repare se outras partes do seu corpo foram afectadas: digamos que sentiu um aperto na barriga e quando lhe prestou atenção, experienciou linhas de energia no seu peito. Permaneça com essa experiência alargada, notando quaisquer alterações ao nível das emoções. Quando as coisas estiverem mais libertas, interrogue-se, com curiosidade: `O que é esta resposta?'

Repita isto, cuidadosamente, com este aspecto do seu mundo, até sentir que algo mudou na sua resposta, ou que isso lhe proporcionou uma chave para uma compreensão mais profunda. Pode sentir um largar ou um afirmar das suas intenções.

Regresse através do corpo: a estrutura central e os tecidos mais moles que a envolvem, a pele que os envolve, o espaço que envolve tudo. Abra os olhos lentamente, sintonizando-se com o espaço e com a sensação do local onde está sentado/a.
