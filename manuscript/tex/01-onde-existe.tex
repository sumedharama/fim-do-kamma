\chapterNote{Uma panorâmica sobre os ensinamentos do Buddha acerca do kamma}

\chapter{Onde Existe uma Vontade Existe um Caminho}

\tocChapterNote{Uma panorâmica sobre os ensinamentos do Buddha acerca do kamma}

\begin{quote}
  ``E, bhikkhus, como é que uma pessoa vive simultaneamente para o seu próprio
  bem e para o dos outros? Ela própria pratica no sentido da remoção da cobiça,
  do ódio e da ilusão e encoraja igualmente os outros a fazerem o mesmo.''

  \quoteRef{\href{https://suttacentral.net/an4.96/en/thanissaro}{AN 4.96}}
\end{quote}

O que é o `kamma' e qual a sua relação com o Despertar? Ora bem, tomado como uma
palavra, `kamma' é a versão em língua Pali do termo em sânscrito `karma', que
tem vindo a ser usado no inglês coloquial\footnote{E também no português
  coloquial (N.T.).} com um significado semelhante a `destino'. Neste sentido,
esta noção pode apoiar uma aceitação passiva das circunstâncias: se algo corre
mal podemos sempre dizer que `era o meu carma', querendo dizer que tinha de
acontecer. Onde a ideia de base realmente se perde é quando é usada para pactuar
com determinadas acções, como por exemplo `ser ladrão é o meu carma'. Se este
fosse o significado do carma tirar"-nos-ia a responsabilidade sobre as nossas
vidas. Para além disso, não haveria qualquer forma de nos orientarmos de modo a
sairmos das nossas circunstâncias ou da nossa história passada -- que é aquilo
em que consiste o Despertar. Contudo, `kamma', no sentido em que o Buddha
ensinou, significa acção adequada ou inadequada -- algo que fazemos agora. É o
aspecto activo de um processo de causa e efeito conhecido como
\emph{kamma"-vipāka}, no qual \emph{vipāka} ou `kamma antigo' significa o efeito,
o resultado de acções anteriores. E, em larga medida, vemo"-nos envolvidos nos
resultados das nossas acções.

Não obstante, enquanto `acção', o kamma apoia a escolha. Podemos escolher as
acções que vamos empreender. Causa e efeito regulam a actividade dos vulcões,
das plantas e dos sistemas planetários, mas o kamma diz respeito especificamente
aos seres que têm poder de escolha sobre aquilo que causam -- ou seja, o leitor
e eu. Da mesma forma, nem tudo o que experienciamos é devido ao kamma antigo
(para além do que diz respeito a termos nascido).\pagenote{Onde existe uma
  vontade existe um caminho:

  O Buddha sublinhou que tudo aquilo que experienciamos não se encontra
  necessariamente relacionado especificamente com as nossas acções anteriores:

  ``\ldots algumas sensações surgem a partir de mucos\ldots{} a partir de ventos
  internos\ldots{} a partir de desequilíbrios dos humores do corpo\ldots{} da
  mudança das estações\ldots{} do cuidado irregular do corpo\ldots{} de
  agressões\ldots{} do resultado do kamma. Que algumas sensações surgem em
  resultado do kamma, podemos saber por nós próprios e todos compreendemos que
  isso acontece. Agora, quaisquer contemporâneos ou sábios que defendem a
  doutrina ou ponto de vista que seja o que for que o indivíduo experiencie --
  prazer, dor, nem prazer nem dor -- é inteiramente causado por aquilo que foi
  feito anteriormente -- ultrapassam aquilo que eles próprios sabem e aquilo em
  que as pessoas em geral concordam. Desta forma afirmo que esses contemporâneos
  ou sábios estão errados.''

  \href{https://suttacentral.net/sn36.21/en/bodhi}{SN 36.21}, ver também
  \href{https://suttacentral.net/mn136/en/thanissaro}{MN 136}.}
Assim, se adoecemos ou se formos apanhados num terramoto, não é necessariamente
porque tenhamos praticado más acções numa vida anterior. Em vez disso, o kamma
centra"-se na nossa intenção ou `volição'
(\emph{cetanā})\pagenote{Intenção/volição é kamma: por exemplo
  \href{https://suttacentral.net/an6.63/en/thanissaro}{AN 6.63}. É importante
  reconhecer que `intenção' neste contexto não requer necessariamente
  deliberação. \emph{Cetanā} refere"-se à `inclinação' ou à `intenção' do
  coração, que se encontra subjacente ao pensamento e alimenta a emoção.}
actual. Desta forma, os ensinamentos sobre o kamma encorajam um sentido de
responsabilidade relativa à acção -- a responsabilidade de prestar atenção às
muitas escolhas conscientes e semiconscientes que realizamos, em relação ao que
fazemos. Isto significa que, neste preciso momento, temos possibilidade de
escolha sobre a forma como o futuro vai decorrer: se nos vamos sentir contentes
e tranquilos connosco próprios, ou ansiosos e deprimidos, dependerá das nossas
acções neste momento. E, de igual forma, através das nossas acções neste
momento, podemos ser libertados do passado, do presente e do futuro. É isso que
resulta do Despertar para o kamma.

\subsection{Kamma físico, verbal e mental}

`Kamma' significa `acção', num sentido que vai para além do físico. Inclui
também a acção verbal -- quer insultemos e gritemos com as pessoas ou digamos
coisas verdadeiras e fiáveis. E isso inclui o `discurso interno' do pensamento!
Mas, na realidade, o kamma das nossas respostas emocionais -- o kamma `mental'
(ou `do coração') -- é o mais forte.\pagenote{O kamma mental é o mais forte.
  Este é o tema de um debate, registado em M.56, no qual o Buddha explica de
  forma convincente:

  ``Destes três tipos de acção\ldots{} descrevo a acção
  mental como a mais repreensível para a realização de acção malévola, e não
  tanto a acção física ou verbal.''} As respostas -- e as propensões sobre as
quais elas se baseiam -- regem as acções do corpo e do discurso, originando
igualmente resultados no domínio das emoções, das atitudes e dos estados de
espírito. De igual modo, as nossas acções físicas ou verbais resultam apenas de
convicções, pressupostos, interpretações e atitudes (ou seja, todos provenientes
da mente). Por si próprio, o corpo não faz bem nem mal -- estas qualidades
éticas encontram"-se enraizadas na mente que inicia o acto físico. Passa"-se o
mesmo com a fala e o pensamento: a linguagem é neutra -- é a bondade ou a
malícia da mente, que utiliza a linguagem e os conceitos, que produz resultados
felizes ou infelizes.

Considerarmos o kamma sob este prisma motiva"-nos a libertar a mente da má
vontade ou da cobiça, porque estas conduzem a acções verbais ou físicas que
deixam uma marca desagradável: geram rudez, apego e obstinação e, mais tarde,
preocupação, arrependimento e dúvida. Por outro lado, acções e pensamentos
baseados na compaixão proporcionam à mente clareza e afecto. Daí os ensinamentos
sobre causa e efeito: eles são um lembrete para verificarmos, investigarmos e
purificarmos os estados de espírito associados à acção, seja ela qual for. Tal
como as nossas acções trazem conflito ou harmonia ao contexto em que vivemos,
controlar o kamma permite"-nos ter um efeito positivo no mundo que nos rodeia.
Perceber o kamma desta forma também possibilita que percebamos amplamente que o
nosso bem"-estar não é independente do modo como agimos para com os outros.

\subsection{As dinâmicas do kamma}

A lei do kamma define que um efeito ou resultado é inevitável a partir de uma
causa activa. Se eu ofendo e maltrato alguém hoje, isto tem como efeito que essa
pessoa se sente magoada e isso quer dizer que ela, provavelmente, irá ser
desagradável comigo no futuro. É igualmente provável que essa acção tenha
efeitos imediatos sobre a minha própria mente: agitação e remorso. Ou pode
acontecer que me habitue a agir dessa forma: de modo que continuo a agir de
forma ofensiva, desenvolvo uma mentalidade insensível e perco amigos. Desta
forma os efeitos acumulam"-se, tanto em termos de estados de espírito (ofensas e
remorsos), como também de estruturas comportamentais (um padrão ou uma
programação de ser egocêntrico ou de possuir uma língua afiada). O que é
verdadeiramente problemático são os `programas', as `formações' (ou, em
linguagem budista, \emph{saṅkhāra}) em decurso. Estes padrões de comportamento
tornam"-se parte da nossa identidade e, uma vez que não conseguimos ver para além
dos nossos próprios hábitos entranhados, estes padrões e programas sustêm o
girar, o \emph{saṁsāra}, de causa e efeito.

Desta forma, se nos queremos libertar, é importante, controlar o modo como
estamos a funcionar. E isso é possível porque o processo \emph{kamma"-vipāka}
forma circuitos de \emph{feedback} de sensações mentais de tensão ou de
agitação, ou de tranquilidade, que podemos contemplar e ter em consideração.
Para além disso, podemos responder de diferentes maneiras ao resultado das
nossas acções - de forma a que cada efeito não venha a inevitavelmente gerar uma
causa correspondente. Eis a escolha: posso fazer uma pausa, sair do estado de
espírito de irritação ou de imprudência, prestar"-lhe a devida atenção e tentar
fazer melhor no futuro. Este é o primeiro passo no sentido da libertação.

Os ensinamentos do kamma são mais facilmente acessíveis no contexto do
comportamento exterior. O Buddha viu que a clareza relativamente ao
comportamento proporciona uma via pragmática, através da qual o sofrimento e a
tensão podem ser evitados, e a paz, a confiança e a clareza podem ser geradas.
Assim, o Buddha falou de kamma escuro -- acções tais como assassínio, roubo,
falsidade e abuso sexual, que conduzem a maus resultados -- e kamma luminoso --
acções tais como bondade, generosidade e honestidade, que fazem o inverso. O
Buddha falou igualmente de uma mistura de kamma escuro e luminoso -- acções que
possuem algumas boas intenções mas que são conduzidas de forma inadequada. Um
exemplo disto seria termos o objetivo de proteger e cuidar da nossa família, mas
fazê"-lo de um modo prejudicial para os nossos vizinhos.

O kamma é também dinâmico: agimos de acordo com informação que recebemos e, à
medida que recebemos de volta os resultados agradáveis ou desagradáveis, as
nossas acções subsequentes vão sendo moldadas. Contudo, uma vez que parte deste
retorno não ocorre imediatamente e pode mesmo levar anos a suceder, existem
aspectos deste circuito de retorno que são caóticos. Isto significa que o nosso
ritmo de aprendizagem não acompanha necessariamente o ritmo ao qual podemos
desenvolver novas acções. Estivemos a poluir alegremente a atmosfera durante
décadas, até se tornar claro o que se estava a passar e, por essa altura, outras
acções já tinham sido realizadas -- estabeleceram"-se indústrias e estilos de
vida dependentes de recursos não sustentáveis, o que torna a mudança mais
complicada.

Este ponto é significativo: encoraja"-nos a esforçarmo"-nos por clarificar a
consciência da mente e dos seus impulsos. É necessário que investiguemos as
nossas mentes e os programas mentais frequentemente e de forma mais detalhada.
Então aí é possível interromper o circuito de retorno com informação que pára ou
que modera os nossos impulsos. Esta informação é o kamma que leva ao final do
kamma e constitui o eixo dos ensinamentos do Buddha.\pagenote{Escuro, luminoso,
  ambos e o kamma que conduz ao fim do kamma:

  ``E qual é o kamma que não é nem escuro nem luminoso, que não tem resultados
  nem escuros nem luminosos, e que conduz ao fim do kamma? A intenção para
  abandonar aí mesmo este kamma que é escuro e com resultados escuros\ldots{}
  luminoso e com resultados luminosos\ldots{} escuro e luminoso e com resultados
  escuros e luminosos.''

  \href{https://suttacentral.net/an4.232/en/sujato}{AN 4.232}, ver também
  \href{https://suttacentral.net/mn57/en/bodhi}{MN 57.7-11}.}
Ao ser plenamente realizado, pode levar não apenas a mudanças no
comportamento, mas à libertação completa.

\subsection{Nascimento: a herança da causa e efeito}

A libertação completa significa sair por completo do processo de causa e efeito.
Este processo é aquilo que estamos a experienciar neste momento. Nascer é kamma
antigo e trouxe"-nos ao imbróglio da existência no domínio da causa e efeito, com
o potencial de continuar a girar nele. Tendo herdado o efeito de sermos
corpóreos, somos afectados por comida, saúde e clima. E, juntamente com isto,
vem o potencial para defender, procurar comida e procriar. A mente está
sintonizada para responder a tudo isto de forma instintiva, com impulsos de
luta, paralisação ou fuga, que podem surgir num abrir e fechar de olhos. E com a
mente vem a consciência do envelhecimento, da doença e da morte. E, com isso, a
separação daqueles que amamos, com o efeito desagradável que isso tem em nós; e
ainda, o facto de sermos aparentemente impotentes para fazer seja o que for
acerca de tudo isto. Assim, o girar do \emph{saṁsāra} prende"-nos no seu
movimento giratório.

O kamma antigo traduz"-se igualmente em todos os programas sociais e
comportamentais que nos dizem como devemos funcionar em termos de costumes e
atitudes. Psicologicamente, experienciamos a necessidade de pertença, de
compreensão e de nos sentirmos em paz com as circunstâncias nas quais estamos
inseridos. Esta constitui a base para o kamma novo: uma das nossas acções
mentais mais contínuas é a de interpretação e de arquivo de experiências, de
forma a extrairmos significados e propósitos. De igual modo, apanhamos muita
informação das outras pessoas, incluindo preconceitos e informação redundante,
assim como bons conselhos. Este kamma mental forma o programa, o modo de pensar,
através do qual interpretamos a nossa experiência presente. Instalado e
modificado ocasionalmente, este programa aplica, em qualquer momento, uma
interpretação da experiência presente para nos dizer `como algo é'. E isto vai
muito além da definição objectiva: `Será seguro, será amigável, será permitido,
serei valorizado ao adquirir isto' e por aí adiante -- torna"-se um filtro de
impressões que nos ajuda a tomar muitas decisões instantâneas nas nossas vidas.
Contudo, qual a fiabilidade deste processo crucial?

Surgem memórias que nos levam a ter determinada inclinação, surgem condições
físicas que afectam a nossa disposição, a duração da atenção varia e o resultado
de uma conversa que tivemos há uma hora atrás ainda está a agitar os nossos
corações. Entretanto o nosso contexto social liga"-nos aos efeitos das mentes das
outras pessoas, cada uma com as suas próprias interpretações e os seus
mal"-entendidos, de forma que também somos afectados por isso. Os programas são
transferidos, ou desencadeiam outros programas, e as nossas mentes são apanhadas
num turbilhão de impressões, sendo que qualquer delas pode a qualquer momento
despoletar certos impulsos.

Então, a parte verdadeiramente crucial daquilo que herdamos é a `mente' -- uma
consciência afectiva"-receptiva complexa que transporta um elevado potencial para
mais kamma. Através da mente podemos estar conscientes de algo, reflectir e
desenvolvermo"-nos de acordo com os resultados daquilo que fazemos e dizemos.
Podemos aprender. E, à medida que as nossas respostas ampliam e intensificam a
nossa experiência, podemos ser criativos. Podemos concentrar"-nos numa ideia e
desenvolvê"-la numa invenção fantástica ou numa belíssima obra de arte. Contudo,
por outro lado, podemos ser assoberbados por uma emoção desenfreada. Se a
disposição é de desconfiança ou de ressentimento, podemos elaborar pontos de
vista maléficos sobre o mundo e cometer atrocidades. Há uma grande amplitude de
possibilidades: o Buddha refere muitos domínios infelizes e infernais que
resultam de acções imorais ou `kamma escuro'. As boas notícias é que os domínios
felizes e celestiais, onde podemos nascer em resultado de kamma bom ou
`luminoso', são ainda mais numerosos.\pagenote{Tanto
  \href{https://suttacentral.net/mn135/en/bodhi}{MN 135} como
  \href{https://suttacentral.net/mn136/en/thanissaro}{MN 136} apontam para
  diversos destinos localizados em reinos celestiais ou infernais como o
  resultado de actos realizados nesta vida.}
Seja como for não se acaba tudo
numa vida! Hoje em dia as pessoas gostam de interpretar estes `domínios'
cosmológicos como estados psicológicos -- mas mesmo sob este ponto de vista, o
kamma, luminoso ou escuro, é acção que vai dar origem a um estado de existência
futuro. E, à primeira vista, parece que não existe saída possível:
experienciamos as nossas vidas como estando inseridas num contínuo dinâmico que
nos afecta e nos modela -- tal como nós o afectamos e modelamos.

Pensando bem, o processo de \emph{kamma"-vipāka} é como o de um oceano que nos
pode elevar, engolir ou transportar em qualquer direcção. Funciona através de
uma interação contínua entre os efeitos herdados, à medida que eles surgem no
presente, e a gama de respostas consequentes e tendências para agir. O passado
não está morto: os seus efeitos transportam potencial. E o futuro surgirá de
acordo com a forma como agimos com base nisso.

\subsection{Despertar para a causa e efeito}

À primeira vista, esta perspetiva do kamma parece ser, não de libertação, mas
sim de aprisionamento. Mas quando se está na cela de uma prisão, começa"-se a
investigá"-la mais detalhadamente para ver, primeiro, como é que se pode torná"-la
mais habitável e, depois, como sair dela.

Em primeiro lugar, se temos de gerar kamma, podemos, pelo menos, determinar se
este vai ser luminoso ou escuro. Lembre"-se que existe a possibilidade de
escolha: o kamma depende da volição mental, ou da intenção, quer se trate de uma
intenção cuidadosamente ponderada, de uma necessidade compulsiva ou de um
reflexo psicológico. Neste sentido, `intenção' não é apenas um plano deliberado.
Com efeito, lá bem no âmago do nosso enigma encontra"-se o facto de nem sempre
estarmos assim tão claros relativamente ao que fazemos e às razões subjacentes.
Podemos estar a funcionar em modo automático, ou com uma atenção turva ou
desfasada, mas ainda assim a agir ou a sermos movidos pelo `empurrão' de uma
paixão ou hábito. Para muitos de nós, o problema principal não é a intenção
deliberadamente malévola, mas sim a acção baseada quer em confusão, desatenção
ou num mal"-entendido. Muitos dos nossos problemas têm origem no facto de
estarmos `pre"-ocupados' ou emperrados em hábitos. Dessa forma, não trazemos uma
atenção clara às nossas acções. Ou pensamos que as coisas estão bem e, portanto,
não precisamos de nos preocupar; ou que não estão bem, mas que não temos
qualquer poder de decisão sobre o assunto. Mas pelo menos existe sempre a
escolha de investigar o kamma, e é aí que começa a reviravolta.

Isto sucede porque a volição mental pode ser compreendida e bem dirigida, se
estivermos cientes dela com atenção e sabedoria. Podemos olhar para os nossos
pensamentos e impulsos, fazer uma pausa e recebê"-los de uma forma mais plena e,
deste modo, sentir o tom luminoso ou escuro que trazem à mente. Isto ajuda"-nos a
perceber como e quando devemos agir. Por exemplo, o sentimento resultante de
kamma luminoso encoraja alguém que tenha tido o bom fortúnio de herdar riqueza
ou inteligência, a partilhá"-la. Assim, essa pessoa irá fazer a melhor utilização
desse kamma antigo ao torná"-lo kamma novo brilhante. Com efeito, quando alguém
não partilha o seu bom fortúnio, o efeito agradável a ele associado estraga"-se:
as pessoas ricas que são egoístas e complacentes, em geral, querem sempre mais
ou tornam"-se avarentas e obcecadas consigo próprias.

Por outro lado, podemos estar a experienciar o efeito nefasto de sermos
agredidos ou magoados, ou de estarmos ansiosos, mas decidirmos não cobrar isso a
outra pessoa. Ou também podemos começar a investigar e a acalmar essa
disposição, e até a compreender o que causou esses efeitos. Neste tipo de acção,
a volição não progride para um estado futuro, mas desenvolve"-se no sentido de
compreendermos o estado presente. Esta compreensão não é uma questão de
percebermos todos os porquês e motivos do kamma: na realidade os detalhes das
razões pelas quais experienciamos uma determinada disposição, intuição ou
impulso num determinado momento são demasiado complexos para serem
compreendidos, é como tentar compreender que rio ou nuvem deu origem a qual gota
de água no oceano.\pagenote{De acordo com
  \href{https://suttacentral.net/an4.77/en/thanissaro}{AN 4.77}, o funcionamento
  preciso do kamma é um dos quatro `imponderáveis'. Ponderar sobre estes leva à
  `loucura ou enfado'. Os restantes são: a dimensão do poder de um Buddha, a
  dimensão dos poderes de alguém em estado de absorção (\emph{jhāna}) e a origem
  do mundo.}
O aspecto mais importante relativamente a compreendermos o kamma é
entrarmos nas correntes subjacentes da mente e alterarmos a maré dos efeitos. E
para fazermos isso é necessário que cultivemos uma consciência firme, clara e
não agitada por estados de humor e sensações. Este reconhecimento, baseado na
`visão correcta', traz o `propósito correcto', a activação do Caminho Óctuplo
que nos liberta do sofrimento e do stress.\pagenote{A Visão Correta é
  considerada como fundamental em numerosos \emph{suttas}, sendo descrita da
  seguinte forma:

  ``Existe aquilo que é dado, aquilo que é oferecido e aquilo que é sacrificado;
  existe o fruto e o resultado de boas e más ações; existe este mundo e o outro
  mundo; existe mãe e pai; existem seres que renascem espontaneamente; existem
  neste mundo contemplativos bons e virtuosos, e sábios que realizaram por si
  próprios através de conhecimento directo e declaram este mundo e o outro
  mundo.'' \href{https://suttacentral.net/mn117/en/bodhi}{MN 117.5}

  ``O Propósito Correto é um propósito que é desprovido, ou se afasta, de causar
  dano, da crueldade e do desejo sensual.''
  \href{https://suttacentral.net/mn19/en/bodhi}{MN 19}}

Um aspecto que se torna evidente em relação à corrente da mente é que seja em
que sentido for que ela esteja a fluir, temos a tendência a ficar presos nela.
Queremos proteger e manter um estado feliz e sentimo"-nos mal relativamente ao
seu eventual declínio ou desaparecimento -- a nossa identidade fica baseada
nesse estado. Por outro lado, sentimo"-nos encalhados e desesperados
relativamente aos estados de infelicidade. Desta forma, todo o kamma origina,
define e nos enquadra em termos psicológicos. Portanto, até o bom kamma confere
uma certa desvantagem, pois ainda implica um investimento na área do kamma.
Nessa área, a roda da sorte pode girar no sentido descendente: podemos ser
atacados e certamente adoeceremos, sentiremos dor e enfrentaremos separações.
Podemos igualmente ser arrebatados por alguma paixão ou impulso avassalador e
fazermos algo do qual nos arrependamos para o resto da vida. Assim, o Buddha
apontou para uma melhor opção: sair desta arena do kamma, aprofundando a
consciência para além do alcance do kamma e do \emph{vipāka}. Este é o objetivo
que leva ao Despertar e à total libertação.

\subsection{Kamma e o `eu\kern -0.5pt'}

No seu verdadeiro sentido, a libertação do kamma é a libertação da mente do
ciclo de causa e efeito. É um processo de nos distanciarmos mental e
emocionalmente de qualquer estado e de o ver apenas como um estado, sem reacções
e atitudes. Esta capacidade simples, que muitos de nós conseguimos ter
ocasionalmente, é aquilo que desenvolvemos na prática budista. De uma forma mais
radical, significa distanciarmo"-nos do programa que afirma que `a minha vida
encontra"-se preenchida se eu tiver ou estiver num determinado estado\ldots{}
neste ou noutro mundo'. Esta é a perspetiva que perpetua o kamma. Quando esta
perspetiva constitui a lente através da qual vemos, iremos continuar a procurar,
ou a imaginar, um determinado estado subjetivo imutável no qual nos encontramos,
podemos vir a estar, ou que poderemos possuir. Já alguém, alguma vez, encontrou
algum estado com estas características? Esta `perspetiva de si próprio'
agarra"-se aos corpos, aos sentimentos, às noções, aos programas mentais e à
informação sensorial sob a forma `isto sou eu, isto é meu, isto sou eu próprio'.
Mas já encontraram algum estado no qual podem ficar para sempre e que nunca vos
desilude?

E quem é este `eu' que pode possuir alguma coisa e o que é que ele ou ela poderá
ser? Como é que `eu' posso ter posse das emoções, se as agradáveis se vão embora
quando `eu' não quero, e as desagradáveis instalam"-se sem serem convidadas? Como
é que `eu' posso ser os meus impulsos, as minhas ideias ou as minhas disposições
se, quando estes surgem, condicionam"-se entre eles e desaparecem como se de
cúmplices se tratassem? Não podemos dizer que os sentimentos constituem um `eu'
permanente, uma vez que eles vão e vêm, e alteram"-se. Nem sequer podemos chamar
isso de propensão para sentir `um eu', uma vez que também isso se encontra
dependente de estarmos acordados ou adormecidos, atordoados, desatentos ou
hipersensíveis. Para além disso, os significados pessoais que projectamos na
vida e as reacções e as respostas psicológicas que são as nossas impressões
digitais, estão também sujeitas a alterações. A dinâmica deste processo é tão
contínua, que dá origem a uma sensação de solidez, mas não se consegue encontrar
uma essência ou uma entidade central permanente.

Se aprofundarmos este tópico, algo que seja `eu' teria de ser independente --
apenas `eu' e não parte de outra coisa. Mas a existência e constituição do
corpo, por exemplo, encontra"-se dependente dos pais e da comida, bem como de
muitos outros factores. Não surge de forma independente. Quando cortamos o
cabelo ou as unhas, não perdemos o sentido de `eu', de forma que o `eu' não pode
estar associado à corporalidade. E os sentimentos e a propensão para sentir não
estão dependentes de `algo' mais? Vemos algo, sentimos algo -- que tipo de ver e
de sentir poderia acontecer sem um objecto? Para além disso, o `eu' não pode
estar associado a estados e actividades mentais -- estes não surgem dependentes
de estados físicos ou de outros estados ou outras actividades mentais? Então,
mais uma vez, se corpo, sentimentos e informação sensorial fossem o meu `eu',
certamente que eu poderia decidir `que sejam desta forma e não de outra'. Mas,
efetivamente, eles seguem o seu próprio caminho, o caminho da `causa e efeito'.
Então o que é o `eu'? E o que é que se passa aqui?

Podemos não conseguir encontrar algo ao qual chamar `eu', contudo experienciamos
`ser algo' de forma contínua; temos um sentimento contínuo de `eu sou'. Como é
que isto acontece? Devido à consciência -- cuja função normal é discriminar a
experiência, em termos de um sujeito e um objecto separados. Mas sujeito e
objecto constituem inferências e não realidades, e eles estão dependentes um do
outro -- não podemos ter a sensação de um sujeito sem um objecto e vice"-versa.
Agora, quando a consciência mental se dirige para a nossa dimensão interna,
subjectiva, mais uma vez separa sujeito de objecto, com uma inferência de `eu',
como o propagador, e de `mim próprio', como aquele que herda os estados de
espírito, as disposições, os pensamentos e tudo o resto. (Tal como em `aqui
estou eu a sentir"-me confundido comigo próprio, porque esta atitude não parece
minha -- estou"-me a sentir fora de mim neste momento'.)

Cada sensação, cada pensamento, cada sentimento que ocorre é classificado como
`aquilo que sou -- isto é meu' (apesar de mudarem e desaparecerem); ou então
mantemos a noção que `eu não sou assim' (apesar de `eu' ser definido em termos
de informação que recebo e que me afecta). Qualquer uma destas duas actividades
psicológicas instintivas de `eu ser' (`eu sou isto' ou `eu sou algo que não
isto') determina continuamente um sentido de eu próprio, de formas que geram
ainda mais uma autodefinição específica. Contudo, uma vez que `eu' não consigo
agarrar"-me àquilo que quero e não consigo escapar daquilo que não quero, a
disposição subjacente do `eu' fica inquieta e insatisfeita. Persisto em
encontrar o estado bom\ldots{}, mas ainda não é bem este. Assim, existe mal"-estar.
A libertação deste mal"-estar e desta tensão é, assim, sinónimo da nossa
Libertação do `eu' insatisfeito.

\subsection{Desenvolver"-se a si próprio para se libertar do `eu\kern -0.5pt'}

A forte tendência para manter um `eu' -- denominado `tornar"-se/devir' ou `ser'
(\emph{bhava}), encapsula o sentido do `eu' e de `mim' como uma noção, uma
impressão de si próprio que perdura como um ponto de referência para a minha
mente, independentemente daquilo em que ela tenha estado envolvida. Desta forma,
este `eu' não é uma entidade mas sim um padrão, um `programa de construção de si
próprio' (\emph{ahaṁkāra}), constituído por comportamentos emocionais e
psicológicos. Estes dão origem à sensação de que: `esta é a forma como funciono,
estas são as minhas opiniões, esta é a minha história' e, consequentemente,
existe a impressão de ser um `eu' neste e naquele estado, com estratégias no
sentido de continuar dessa forma, de a incrementar ou de sair dela. Mas tudo
isto não constitui uma identidade, algo fixo, sendo sim um processo de `ser isto
e tornar"-se aquilo'. `Tornar"-se' (ou `devir') é o programa central por detrás do
kamma. É a razão pela qual existe tanta actividade mental. Mas devido a ser uma
actividade, não pode parar. Contudo, este `devir' está enraizado e é instintivo:
não pára através do raciocínio. Como instinto que é, tem de ser travado no âmago
do coração, sendo este um processo que envolve conduzir o impulso da intenção a
clarificar o ponto de vista que cria o `eu'.

Primeiro temos de criar força, perícia, capacidade. Assim, muitos dos
ensinamentos do Buddha favorecem o `devir' em termos de nos tornarmos mais
claros, mais estáveis, mais afáveis.\pagenote{Ou seja, aquele que se desenvolve
  faz com que estados positivos, tais como a calma e a bondade, se concretizem.
  Alguém assim tem: ``realização em termos de virtude; de desejo; de si próprio;
  de diligência; de atenção cuidadosa.''
  \href{https://suttacentral.net/sn45.71-75/en/bodhi}{SN 45.71}}
Geramos bom kamma através de actos de generosidade, bondade e através do abandono dos
comportamentos que causam dano. Tornamo"-nos pessoas mais calmas, mais
inteligentes. A visão correcta encoraja consciência dos resultados daquilo que
fazemos, consciência de partilharmos este mundo com outros e consciência da
importância do nosso domínio interior, psicológico, sobre o domínio do contacto
dos sentidos. Se o sentido mais profundo de realização, através da compaixão e
da calma, se estabelece, podemos então largar o `conseguir, ganhar e agarrar' --
principais causas de tensão. Podemos divergir da necessidade de eficiência a
curto prazo, de conveniência, de sucesso, de conforto e de atracção sexual.
Estaremos, então, no caminho certo para a libertação.

Consideremos o cenário social: no ocidente, muitos de nós podemos viver com
conforto físico, contudo, por estamos constantemente a ser confrontados com
comodidades mais refinadas ou com alterações constantes nos padrões de
comparação, não existe um grande contentamento. E existem também pressões
sociais e de grupo. Uma pessoa pode muito bem sentir que o seu emprego está em
risco se a roupa que está a usar não for a `correcta', de forma que tem de ter
este aspecto em consideração. As pessoas podem tornar"-se deprimidas, mesmo
neuróticas, se os seus corpos não estiverem à altura dos padrões actuais de
beleza, ou se as suas personalidades não forem suficientemente vivas, cínicas ou
sórdidas (dependendo da moda). Queremos evitar perder boas oportunidades e
receamos a solidão de não ter amigos. Por isso pode existir um sentimento de
nervosismo ou de insegurança relativos a falta de adequação, que nos privam de
um sentimento de confiança no nosso valor inato enquanto seres humanos.

Então, somente por causa disto, é importante que sintamos e nos definamos como
`estando' afastados destas correntes, mais que não seja para chegarmos a uma
base mais firme. E o que realmente nos ajuda é acalmar e apaziguar a mente, e
desenvolvermo"-nos naquilo que confere maiores benefícios, vivendo a nossa vida
com autenticidade. Podemos cultivar uma simplicidade de necessidades e um
sentido de sinceridade e de integridade. Podemos ter contentamento através do
reconhecimento do bem em nós próprios e nos outros. As pessoas têm problemas e
defeitos, mas é sábio honrar a bondade que todos temos. É igualmente sensato
encarar com compaixão as nossas tendências no sentido da reactividade e da
confusão, isto porque a forma como dispomos a nossa atenção cria o âmbito no
qual a mente habita. De modo que, se pudermos começar por experienciar clareza e
empatia relativamente a nós próprios e aos outros, damos por nós a viver de uma
forma mais grata e equilibrada, encorajando o desenvolvimento da bondade. Assim,
podemos dirigirmo"-nos para o bom kamma, desenvolvendo uma base que constitui um
real apoio para o nosso bem"-estar.

Trabalhar desta forma com as impressões da mente pode trazer mudanças radicais
na vida. Descobrimos que o factor de `saber bem', de agradabilidade, requerido
pelo nosso sentido de `eu' é adquirido de modo muito mais completo através da
propensão para um comportamento ético e de compaixão. As qualidades que surgem
destas propensões são altamente nutritivas. Trata"-se de cultivar o `eu' de uma
forma correcta e constitui um aspecto essencial da prática do Dhamma. A partir
daqui compreendemos que podemos realizar escolhas com significado nas nossas
vidas, o que nos confere a capacidade de sentir o potencial da nossa condição
humana e nos encoraja a investigá"-la mais profundamente.

\subsection{Discernimento e `não"-eu\kern -0.5pt'}

Todo este bom kamma baseia"-se no cultivo da mente, sendo que a capacidade de
acalmar e serenar a mente através da meditação é outra forma de kamma `bom' ou
`luminoso'. De uma forma geral, existem dois temas principais de meditação para
estabilizar a mente: o de manter uma atenção firme e o de trazer bem"-estar ao
coração. Estes temas focam"-se em efeitos benéficos e alegres, que proporcionam
conforto imediato à mente, e desligam os programas verdadeiramente destrutivos
de ressentimento, depressão, ansiedade, etc. De modo que nos tornamos um `bom
eu', com uma mente calma e aberta. Isso torna possível a investigação sobre como
o `eu' ocorre e como se pode abdicar totalmente do `devir'. Este é o cultivo da
`realização interior'.

\enlargethispage{2\baselineskip}

Estes dois aspectos do cultivo mental -- desenvolver um `bom eu' e
libertarmo"-nos da perspectiva do `eu' -- andam lado a lado. Quando conseguimos
desenvolver um `bom eu', podemos investigar o que está na sua base e compreender
que se fundamenta em estados que dependem de bons programas, tais como bondade,
determinação ou concentração. Não são inerentemente seus ou de qualquer outra
pessoa. Esta é a visão da compreensão: as causas e condições dão origem a
efeitos felizes ou lamentáveis. Para além disso, os resultados felizes surgem de
forma mais pronta e constante se a mente não se encontrar preocupada em afirmar
ou negar um `eu' que tem ou não esses resultados. Esta visão do `eu' como
sucesso ou fracasso cria uma pré"-concepção, na qual as mentes ficam encalhadas
e, quando isto acontece, sente"-se tensão, surgindo as condições para agitação,
incerteza, anseio, desânimo, etc.

Mas temos de lidar com a pespectiva do `eu', compreendê"-la e revelá"-la como uma
série de programas que criam tensão. Tentarmo"-nos livrar de um `eu' requer
intenção (e até uma atitude de indiferença gera efeitos). Qualquer abordagem
niilista transporta sementes de kamma escuro: querer ser nada (\emph{vibhava})
continua a partir do princípio de que somos algo, e envolve a intenção de
aniquilação. Tudo isto torna"-nos menos confiantes e generosos nas nossas acções
e relacionamentos. Pelo contrário, o processo de Despertar altera a enfâse na
construção do `eu' para uma enfâse no apoio e no apreço pelo equilíbrio mental.
Temos necessidade de sentir essa estabilidade interior, no presente, de forma a
sermos capazes de largar os pressupostos e preconceitos do passado e a
necessidade de devir, que luta ou anseia pelo futuro. E é através deste abandono
do devir que encontramos um ponto de tranquilidade -- onde o kamma antigo deixa
de afectar a mente.

\subsection{Prática diária}

Tal como todos os que praticam não tardam a compreender, este cultivo
introspectivo tem de remar contra a corrente de grande parte da cultura
dominante. Temos uma enorme indústria de distracção que nos encoraja a
empregarmos o tempo a evadirmo"-nos de onde estamos -- ler alguma coisa, comer
alguma coisa, olhar distraidamente para a televisão -- e isto afasta"-nos do acto
de observar e cultivar a mente. E~também nos coloca frequentemente em situações
nas quais não recebemos qualquer tipo de informação responsável. As distrações
parecem oferecer uma saída fácil da tensão, mas não nos fazem muito bem.
Trata"-se de uma forma de ignorar os efeitos negativos com os quais não queremos
ou não sabemos lidar, mas que não os remove totalmente. Por exemplo, digamos que
depois de ter tido um dia stressante, teve de conduzir no meio do ambiente
irritante e agressivo do trânsito: todo aquele contacto aleatório rápido e
potencialmente arriscado agita o sistema nervoso. De forma que chega a casa a
sentir"-se tenso e desgastado, e então surge o reflexo para contactar algo
agradável ou fácil. Talvez apenas se atire para o sofá e coma qualquer coisa,
beba algo ou veja qualquer coisa na televisão, fique vagamente descontraído ou
divertido e dependente de apoios exteriores. Neste cenário a mente torna"-se
fraca e subdesenvolvida e, gradualmente, torna"-se necessário que as formas de
distracção sejam cada vez mais fortes. Se permitirmos que este padrão persista,
podemos passar uma vida a tornarmo"-nos psicológica e emocionalmente fracos.

A maior parte de nós precisa de lembretes de forma a não sermos contagiados pela
disposição do contexto social indiscriminadamente e para nos lembrarmos que
podemos conseguir sair do programa da rotina diária. Para começar, tenho sempre
uma imagem do Buddha no espaço onde habito, num pequeno altar, ou em qualquer
sítio onde possa estar em contacto com ela. É algo que me lembra o valor de
estar plenamente presente -- encoraja"-me a fazer uma pausa, reconhecer e reagir
ao estado em que me encontro, seja ele qual for. Recorda"-me que a primeira
prioridade é criar algum espaço na mente, em vez de fazer com que o cultivo da
mente seja ainda mais outra coisa para me ocupar. Se o dia é uma roda"-viva a
fazer e a resolver coisas, é bom aprender a moderar essas mesmas volições.

Quando estamos num estado emocional instável, a resposta mais adequada pode ser
apenas receber o que estamos a sentir no momento presente com alguma clareza e
simpatia -- sentarmo"-nos tranquilamente e permitirmos que as coisas passem
através de nós. Seja qual for o estado, a reacção inicial tem de ser de
permanecer presente e cultivar espaço. Mas causa e efeito funcionam de uma forma
que faz com que quando ficamos sem agir ou suprimir o nosso estado mental actual
(mesmo que seja só por cinco minutos), o resultado é um determinado tipo de
tranquilidade e de diminuição de pressão. Então começamos a reconhecer uma
sanidade natural, uma semente do Despertar, que está presente quando `o fazer'
cessa. Não está muito distante. Mas nós necessitamos de entrar em contacto com
ela e encorajá"-la.

\subsection{É sempre possível}

A constatação mais significativa que resulta de uma pausa de cinco ou dez
minutos dos aspectos de ser e devir, é que as coisas param por elas próprias.
Não quero com isto dizer que uns minutos de descontração sejam o fim do kamma e
a alvorada da Derradeira Verdade, mas mostram"-nos que a mente pode ter uma
direcção diferente, em vez de avançar (ou recuar) aos ziguezagues. A mente pode
ser aberta, e nessa abertura, todo o cenário se altera: experiencia"-se a
consciência mental como um campo no qual pensamentos, disposições e sensações
vêm e vão. E podemos testemunhar esse conteúdo mental, em vez de agirmos em
função dele. Para além disso, uma boa parte do conteúdo mental cessa quando não
existe a perspectiva de que temos de agir, resolver ou mesmo parar, e não existe
o `eu' atarefado a tentar urgentemente manter esse conteúdo mental
activado.\pagenote{De acordo com \href{https://suttacentral.net/mn57/en/bodhi}{MN 57.7-11},
  os quatro tipos de kamma -- luminoso, escuro e uma mistura de
  luminoso e escuro -- dão origem e fomentam uma noção de `eu':

  ``Assim, o aparecimento de um ser é devido a um ser.'' Contudo, o kamma que
  leva ao fim do kamma é descrito simplesmente como ``acção que leva ao fim da
  acção''.}
Esta é uma descrição resumida da forma como o kamma do desenvolvimento leva ao
fim do kamma.

Mas trata"-se de um processo subtil. Os pensamentos e as disposições não param
por tentarmos fazê"-los desaparecer -- essa tentativa constitui mais volição,
mais kamma. Eles param quando o programa de identificação, a base do `devir',
não é posto em funcionamento. Vale a pena lembrarmo"-nos disto porque quando
consideramos a complexidade dos circuitos de \emph{feedback} de causa e efeito,
é fácil perceber que seria necessário um muito complexo processo de desenlace e
acalmia. Mas um vislumbre de como ocorre a cessação ao não darmos apoio à
energia e perspectiva da identificação, mostra"-nos que a forma de sairmos do
mal"-estar e da tensão é simples e directa. E isso encoraja"-nos a criar ocasiões
nas quais podemos levar a não"-identificação a níveis mais profundos da nossa
actividade psicológica. Isto é conseguido através do processo contínuo de
meditação, mas todos podemos e precisamos de apoiar esse processo no dia a dia.
Não precisamos de beber a água onde nadamos.

Não bebermos do oceano do \emph{saṁsāra} significa controlar e restringir a
atracção dos sentidos, controlar e pôr de lado os programas normais dominantes e
cultivar a atenção e a consciência plenas. Ou seja, trata"-se de um caminho para
a vida, o Caminho Óctuplo. E isto decorre da Visão Correcta e da Intenção
Correcta, não das perspectivas do `eu', do destino ou através de sistemas e
técnicas automáticas. Ao seguir esse caminho, constatamos que não estamos tão
incorporados e presos no \emph{saṁsāra} como pareceria. Para começar, na
realidade nunca conseguimos tornarmo"-nos algo durante muito tempo. Claro que
atravessamos períodos de agitação e de tensão mas, com a prática, existem
períodos de alegria e de bom"-humor e, à medida que nos tornamos mais astutos a
levarmos a mente em consideração, o hábito de nos agarrarmos a determinados
estados abranda. Identificamo"-nos cada vez menos com este ou com aquele estado e
isso reduz a tensão e a agitação.

Deste ponto de vista, a vida humana constitui uma grande oportunidade.
Independentemente dos efeitos que herdamos, podemos sempre agir de forma
adequada e cultivar a mente. Podemos sempre encaminharmo"-nos no sentido da
bondade, da felicidade e da libertação.

\clearpage

\section[Meditação: sentar-se tranquilamente]{Meditação}

{\centering
\subSectionFont\selectfont
\textit{Sentar-se tranquilamente}
\par}

\bigskip

Sente"-se confortavelmente num local tranquilo. Descontraia os olhos mas
deixe"-os abertos ou semicerrados, com um olhar descontraído. Esteja consciente
da sensação dos globos oculares a descansar nas respetivas cavidades (em vez de
focar o olhar naquilo que vê). Esteja sensível à tendência dos olhos para não
pararem quietos e descontraia"-os constantemente. Como alternativa, pode achar
útil deixar o olhar descansar, de forma descontraída, num objecto propício, tal
como uma paisagem distante.

Seguidamente leve a atenção para as sensações nas suas mãos, depois para o
maxilar e para a língua. Veja se também estas podem deixar, por um momento, de
estarem sempre apostos para entrar em acção ou para se defender. Deixe a língua
descansar na parte inferior da boca. Seguidamente percorra, com essa atenção
relaxante, uma área desde os cantos dos olhos até à volta da cabeça, como se
estivesse a tirar um lenço. Deixe que o couro cabeludo se sinta liberto.

Deixe os olhos fecharem"-se. À medida que descontrai a cabeça e a cara a toda à
volta, traga essa qualidade de atenção, lentamente, gradualmente, mais abaixo
para a garganta. Solte essa zona, como se permitisse que cada exalação soasse
como um zumbido inaudível.

Mantendo o contacto com essas zonas do corpo, ciente do fluxo de pensamentos e
de emoções que passam pela mente. Oiça"-os como se estivesse a ouvir a água a
correr ou o mar. Se sentir que está a reagir"-lhes, leve a atenção para a
próxima expiração, continuando a descontrair através dos olhos, da garganta e
das mãos.

Se quiser expandir o processo, percorra com a sua atenção o corpo até às plantas
dos pés e, desta forma, construa toda uma sensação de à-vontade no corpo.

Mantendo a consciência da presença do seu corpo como um todo, pratique o
distanciamento ou largue quaisquer pensamentos e emoções que surjam. Não lhes
acrescente nada, deixe"-os passar. Sempre que fizer isso, repare na sensação de
espaço, mesmo que breve, que parece estar aí presente, por detrás dos
pensamentos e dos sentimentos. Entre em sintonia com essa tranquilidade.

Ao sentir essa tranquilidade, acolha-a. Em vez de exigir ou tentar atingir um
estado de calma, crie uma prática na qual serenamente oferece paz às energias
que o atravessam.
