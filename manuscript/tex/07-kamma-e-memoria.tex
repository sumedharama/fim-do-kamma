\chapterNote{Limpar o passado}

\chapter{Kamma e Memória}

\tocChapterNote{Limpar o passado}

\begin{quote}

``Quem foi imprudente no passado, mas deixou de o ser,\\
ilumina o mundo tal lua liberta de nuvens.''

\href{https://suttacentral.net/dhp167-178/en/buddharakkhita}{Dhammapada 172}

\end{quote}

Lembra-se alguma vez de coisas que fez e que gostaria de não ter feito? Ou, por
vezes, quando depois de proferir um comentário acutilante ou de se ver a
exagerar de forma a conseguir o que quer, tem uma sensação desagradável pesada
como: `Aí, lá estou eu outra vez!... Provavelmente vai levar-me várias vidas até
conseguir controlar a minha mente\ldots{} Tenho um grande kamma\ldots{} Como é
que eu saio desta?\textquotesingle{}

A determinado ponto da nossa vida, todos nós dissemos ou fizemos coisas das
quais nos arrependemos. Ou não fizemos: não dissemos as palavras de generosidade
e de amizade que gostaríamos de ter dito, não realizámos as acções nobres ou
atenciosas que gostaríamos de ter realizado. Por outro lado, podemos ter sido
alvo de acções desagradáveis; outras pessoas podem ter-se aproveitado ou abusado
de nós, pessoas nas quais confiávamos podem ter-nos desiludido. Talvez a falta
de confiança nas outras pessoas continue a \mbox{ensombrar-nos} e a fazer-nos
sentir arredios e isolados. De forma que podem ocorrer na mente estas
`percepções' (ou `significados sentidos') desagradáveis, que diminuem o nosso
respeito por nós próprios e a nossa confiança em fazer as coisas e em estar com
os outros. Estes significados sentidos podem formar padrões de comportamento ou
`afirmações de vida' relativamente a quem somos: a vítima, aquele que é deixado
de fora, o danificado, não amado ou impuro. E estes padrões podem continuar a
desenvolver-se e a influenciar diferentes cenários: é como se os actores e o
pano de fundo mudassem, mas as tonalidades de desconfiança continuassem a
persistir nos nossos corações.

\section{O kamma antigo não morre}

Isto constitui \emph{vipāka}, `kamma antigo'. E não morre -- não sem que haja
alguma ajuda. Mas nem tudo resulta de acções que tenhamos realizado. O kamma
antigo mais básico é o de termos nascido; de termos herdado consciência
sensorial (\emph{viññaṇa}). A consciência sensorial é aquilo que interioriza a
experiência dos sentidos -- dá origem à sensação `isto está a acontecer-me'. Uma
galinha ou um lagarto também têm isso -- trata-se do resultado de se nascer com
consciência sensorial. Ou seja, alguns dos objectos dos sentidos ocorrem na
presença de um órgão dos sentidos em funcionamento e, com este `contacto
perturbador' (\emph{paṭigha-phassa}), surge a consciência sensorial.
Seguidamente a consciência mental forma a `percepção', que constitui uma
impressão daquilo que foi visto, e assim sucessivamente. Esta é a forma como a
experiência é interiorizada: seja onde for que a consciência mental ocorra,
dá-se uma impressão ou `contacto designação'.\pagenote{As duas formas de
  contacto estão expostas em \href{https://suttacentral.net/dn15/en/bodhi}{DN
    15.20}.} Esta impressão afecta-nos e nós reagimos a ela.

`Contacto' constitui o acto de registar a experiência. Consideremos agora uma
situação em que estamos profundamente concentrados na leitura de um livro ou a
ver um filme: a consciência sensorial dos nossos corpos, da pressão sobre a
cadeira e talvez mesmo de alguma ligeira dor ou de desconforto, desaparecem. A
atenção da mente é absorvida no acto de ver e de processar o que está a ser
visto, de forma que as impressões tácteis da cadeira e de estarmos sentados, não
causam uma impressão interna.\pagenote{Tal como em
  \href{https://suttacentral.net/mn18/en/nyanamoli-thera}{MN 18.18}: ``Quando
  não existe visão, forma ou consciência visual, é impossível assinalar a
  manifestação de contacto. Quando não existe manifestação de contacto, é
  impossível assinalar a manifestação de sensação. Quando não existe\ldots{}
  sensação, é impossível assinalar\ldots{} percepção\ldots{} sem\ldots{}
  percepção, é impossível assinalar a manifestação do pensamento\ldots{} sem
  pensamento, é impossível assinalar a manifestação do ser assolado pelas
  percepções e noções tingidas pela proliferação mental.''}
Neste caso, não se pode dizer que tenha havido contacto -- este depende de onde
se encontra a atenção.

Quando experienciamos contacto, dá-se uma mudança -- algo reverbera. O que
ocorre é que determinados padrões de energia, a nível físico, conceptual ou
emocional, reagem. Deste modo, estarmos conscientes de algo envolve uma mudança
energética -- concentramo-nos, animamo-nos, descontraímo-nos ou retraímo-nos --
somos movidos interiormente. E a forma como somos interiormente tocados depende
da atenção e do contacto, dos programas e padrões das formas (\emph{saṅkhāra})
tais como activação, descontracção ou defesa. A consciência sensorial prende-se
com estes últimos: encontra-se programada para formar e reter programas e
padrões adicionais. Por sua vez, estes transportam intenção sob a forma de uma
resposta momentânea ou de um propósito continuado. A~intenção mais fundamental
envolve determinar se algo que vimos, ouvimos, tocamos, etc., constitui um
indício de ameaça ou de prazer. A um nível muito básico isto é aquilo que
programa as nossas vidas: procuramos o que é agradável e procuramos evitar o que
é desagradável.

A natureza da intenção afecta igualmente a atenção: quando um ladrão olha para
um santo, está a notar os bolsos deste. E a intenção é dirigida por aquilo que a
mente designa como algo que proporciona prazer (tal como adquirir algum
dinheiro) ou, claro, desprazer. Mas, como no exemplo anterior, o que é agradável
não consiste na sensação táctil de ter as moedas na mão, mas sim no seu
significado. Assim, o `significado sentido' associado a uma experiência -- esta
`impressão no coração' ou `contacto designação' -- constitui o aspecto mais
importante do contacto. Construímos as nossas cognições a partir destes
significados sentidos e aprendemos o significado dos conteúdos: o sorriso
significa afabilidade, esta pessoa é de confiança, etc.. Depois `reconhecemos'
estes sinais, quando nos deparamos com eles, bem como outros semelhantes. Esta é
a forma como os significados sentidos, os apetites e as necessidades do coração
têm o seu efeito sobre aquilo que procuramos ou recordamos. Assim, aquilo que
capta a atenção da mente são os sinais que representam estes significados
agradáveis ou desagradáveis. Estes sinais tornam-se pontos de referência para a
forma como nos sentimos bem, ou não, no presente. No caso anterior, isto
significa que o ladrão avalia as suas oportunidades e, depois, quando a atenção
do santo se encontra dirigida para outro sítio, esvazia-lhe o bolso. Dito de
outra forma, o contacto-designação depende dos padrões, do kamma antigo, e
constitui uma base para a acção, o kamma novo.

Este processo de formação de padrões é mais do que um processo mental. O corpo
também, `aprende' e até se `lembra'. Por exemplo, aprendeu a levantar-se e a
manter o equilíbrio. O corpo sabe quais as sensações, as pressões, etc., que são
indícios de que está a manter um bom equilíbrio. De igual forma, esta
inteligência corporal proporciona-nos os reflexos de fuga, paralisação ou luta.
Quando estamos numa discussão, o corpo fica tenso, um som estridente pode
fazer-nos `saltar', um sorriso `amigável' pode despoletar uma aceleração na
pulsação, e por aí adiante. O corpo e a mente-base, em termos essenciais, não
são separados e, ao nível instintivo dos reflexos, a inteligência corporal pode
sobrepor-se à racional. É importante ter em conta este dado, porque mesmo quando
uma memória, ou a consequência do kamma, é analisada de forma racional e posta
de lado, esquecida ou suprimida, ainda pode restar uma memória corporal e
emotiva dos acontecimentos significativos. E, tal como qualquer outra memória,
pode surgir em qualquer momento inesperado.

Aquilo que designamos, em termos globais, como `memória' é, então, a referência
da consciência para estabelecer padrões. E quando a memória entra em acção, age
como um chamamento súbito: `Foi isto que aconteceu e no momento presente fazemos
parte disso'. Esta memória pode ser algo que podemos fazer surgir
deliberadamente ou pode tratar-se de uma ocorrência involuntária, como os
\emph{flashbacks} ou traumas com forte carga emocional, com as suas alterações
súbitas na energia física. Com base neste despoletar de antigos padrões
avaliamos uma situação actual em termos do nosso passado e determinamos o que
fazer. Avaliamos a situação presente de acordo com a forma como se assemelha a
algo do passado e decidimos o que fazer.

Apesar de por vezes ter um caráter neutro, o processo da memória é emotivo,
sendo por vezes intensamente evocativo. Se no passado um acontecimento nos moveu
emocionalmente, a impressão no coração é forte. Então, com a memória, surge a
emoção. Trata-se de um processo involuntário e pode ser assustador, quando o
presente se torna subitamente num intenso reviver do passado `emocional':
estamos a falar com alguém, a conversa dá algumas voltas e, subitamente,
voltamos a estar a discutir com o nosso pai, ou a sentirmo-nos rejeitados por
alguém de quem gostamos... novamente. Este reviver dos acontecimentos passados
acontece mesmo quando não fizemos fosse o que fosse, mas fomos recipientes das
boas ou más acções de outras pessoas. Porquê? Porque o programa da mente afetiva
ou `coração' consiste em reter um afecto com carga emocional como uma impressão
do coração e, depois, reciclá-la. Este constitui o `kamma antigo' resultante de
termos uma mente afectiva. E as impressões do coração, quer sejam de prazer ou
de medo, passam a ser referidas como `eu', `meu' e `eu próprio'.

Assim, o nosso sentido de nós próprios não é apenas gerado pela forma como
agimos, surgindo igualmente a partir de uma aceitação instintiva e de um reviver
daquilo que nos foi dito que somos ou do que nos fizeram `sentir' que somos. A
nossa personalidade, desta forma, encontra-se em evolução contínua. E, assim, se
não cultivarmos o abandono dos padrões antigos e não deixarmos de aceitar as
impressões do coração como uma verdade imparcial e como sendo `nós próprios', os
assuntos do passado constituirão a base para mais kamma.

\section{Colher -- e limpar -- resultados do passado}

Os padrões de sofrimento, como a depressão, a ansiedade, o ressentimento e a
culpa actuam como um peso morto no potencial da mente. Assim, quando o processo
da memória os faz surgir, precisamos de os limpar -- para o bem das outras
pessoas, bem como para o nosso. Este processo de limpeza, tal como é delineado
pelo Buddha, é duplo: primeiro reconhecer os resultados da acção e resolver não
voltar a agir dessa forma; em segundo lugar, disseminar intenções de boa vontade
através de todo o sistema e para com qualquer outra pessoa ligada àquela acção.

O passado não se vai embora por ele próprio. Quando cometemos uma acção menos
benigna, isso deixa uma impressão forte e persistente. Assim, quando existe uma
memória, esta faz surgir a disposição e a imagem dessa acção. É como se o
passado voltasse de repente -- mas como o poderia fazer se fosse realmente
passado? O que na realidade aconteceu foi que o acontecimento criou um padrão de
\emph{saṅkhāra}, uma espécie de trajectória no fluxo da mente. Então um programa
no presente envia energia mental por essa trajectória e o passado `regressa'. É
a forma como a mente funciona: os padrões e os programas de \emph{saṅkhāra},
bons e maus, são os meios através dos quais a mente trabalha. Estes produzem a
nossa própria imagem e podem trazer o passado à tona, independentemente da nossa
vontade. É assim que o passado e a imagem que temos de nós próprios permanecem
connosco.

Qualquer forma de agressão -- física, verbal, psicológica -- dos outros ou de
nós próprios, provoca uma barreira ou até mesmo uma perversão da sensibilidade
da mente. E tudo isto deixa a sua trajectória enquanto `memória' ou
\emph{vipāka}. Mesmo os pensamentos inadequados têm este efeito --
principalmente porque os podemos repetir muito mais vezes do que os actos
físicos. Se permitimos que a mente, repetidamente, enuncie falsidade, inveja ou
mesmo culpa, isso cria uma trajectória ao longo da qual as energias emocionais e
psicológicas vão decorrer. É muito provável que mais tarde ou mais cedo as
acções físicas ou verbais se desloquem ao longo dessa trajectória mas, mesmo que
isso não aconteça, vão ocorrer atitudes e formas de pensamento que irão ter
fortes efeitos sobre a mente. As pessoas podem guardar rancores e recitar os
agravos que sofreram durante anos depois dos acontecimentos terem ocorrido. E
podem sentir uma culpa e um desânimo crónicos devido às acções que praticaram.
Podem correr programas de auto-depreciação e falta de valor próprio devido às
acções e às atitudes a que foram sujeitos. Mais grave ainda: quando se encontra
ligado à imagem que temos de nós próprios, nem nos damos conta, pois encontra-se
tão entranhado que se torna normal: eu sou este ser inadequado. E esta percepção
constitui, então, o programa básico que afecta todas as acções da nossa vida.

Então, se aquilo do qual nos apercebemos no presente, aquilo que recordamos do
passado, aquilo que imaginamos do futuro e mesmo a forma como nos vemos a nós
próprios, não são uma verdade imaculada mas dependem do kamma... o que devemos
fazer para deixar de viver o kamma antigo, com os seus hábitos e predisposições?

O Buddha define kamma como intenção. Ou seja: a intenção não cria o kamma; a
intenção -- `energia impulso' -- em si própria, é kamma. O kamma não é uma
divindade sem rosto e remota, nem um sistema automático que soma o bem que
fazemos, subtrai todo o mal e nos dá o resultado dessa operação aritmética. O
kamma é a descarga no nosso sistema nervoso, a expansão da luminosidade no
coração ou a acutilância do olhar. Uma das primeiras coisas que aprendemos
através da prática do Dhamma é o tipo de potencial que transportamos nestes
\emph{saṅkhārā}; e a segunda é que não temos de agir em sob esses
\emph{saṅkhārā} ou reagir-lhes. E por último, aprendemos que é através do acesso
ao padrão do nosso kamma adequado que começamos a limpar o passado.

O que precisa ser limpo situa-se em três níveis: existem padrões activos, os
programas que estão a correr; existem tendências involuntárias, programas que
estão adormecidos mas que vêm à superfície quando estamos sob tensão ou quando a
mente se revela na meditação; e por último, existe a imagem que temos de nós
próprios. Em qualquer dos casos, o método envolve o acesso a padrões e programas
do kamma antigo da mente, bem como a descoberta das suas trajectórias. Assim, a
boa notícia é que, uma vez que o kamma se desloca nestas trajectórias, limpar o
passado não implica passar por todos os nossos actos específicos de ações menos
benfazejas, mas sim corrigir, desenraizar ou abandonar a trajectória.

No nível inicial e mais óbvio (o reconhecimento das nossas acções e a alteração
da forma como iremos agir no futuro) admitimos quaisquer acções inadequadas que
sentimos ter realizado e reflectimos sobre o padrão subjacente. Não basta tentar
alterar sem olharmos para a forma como agimos, bem como para as tendências que
nos animam. Mas, se o fazemos, é provável que comecemos a revelar as tendências
subjacentes -- por vezes é apenas aquela tendência para a ignorância que nos
torna descuidados, ou com falta de consideração relativamente à forma como
afectamos os outros. Seja como for, quando nos debruçamos sobre isto e
compreendemos que esse padrão não é agradável e não nos leva a lado nenhum,
podemos decidir a um nível profundo, ou até prometer a nós próprios,
refrearmo-nos de agir de forma semelhante no futuro. E, seguidamente, o tema
geral da prática consiste em espalhar benquerença (\emph{mettā}), compaixão
(\emph{karunā}), alegria abnegada (\emph{muditā}) e equanimidade
(\emph{upekkhā}) relativamente aos outros seres que sentimos poder ter
prejudicado.

De igual forma devemos cultivar as mesmas qualidades para com os nossos corações
quando estes ficam infectados com violência, falsidade ou algo semelhante. A
prática abrange tanto a nós próprios como aos outros, uma vez que no coração `o
eu e os outros' são interdependentes. Ou seja, a nossa personalidade baseia-se e
é moldada de acordo com aqueles com os quais interagimos. Isto certamente
acontece quando estamos na presença de alguém que nos é hostil ou acolhedor:
podemos nos sentir e agir como uma vítima ou como um amigo de longa data. Assim,
quando recordamos uma acção inadequada que praticámos relativamente a alguém,
podemos igualmente ter presente a personalidade insensível que podemos ter tido
na altura. E quando `nos lembramos' de ter sido o objecto da agressão ou falta
de empatia por parte de outros, fazemos algo semelhante. Temos de tomar em
consideração todo o cenário de quem sentimos que fomos, bem como de como
sentimos que o outro foi, e espargir boa vontade em tudo isso.

O Buddha usa a analogia de alguém a soprar num búzio para descrever a
disseminação de bondade, compaixão, alegria abnegada ou equanimidade -- quer se
trate de uma destas qualidades ou de todas.\pagenote{Soprador da concha de búzio:
  \href{https://suttacentral.net/sn42.8/en/sujato}{SN 42.8}} No seu conjunto,
são chamadas `a intenção incomensurável', uma vez que o seu som segue sem
restrições em todas as direções: para os outros tal como para nós próprios; para
o coração que agiu sob a acção dessas energias e para quaisquer outros que
tenham sido afectados por elas. A `melodia' exacta que entoamos é algo que surge
dependendo da distorção que nos encontramos a sarar. Existem dores que trazem à
tona a noção da necessidade básica de nutrir a qualidade da bondade; enquanto
que por outro lado, a consciência de como todos somos tão voláteis e
vulneráveis, pode apelar à compaixão, a energia protectora. Por vezes trata-se
de reconhecer o dano associado à negligência para com aquilo que é bom em nós e
nos outros, ou mesmo à negligência em relação aos outros de forma geral. Então o
sentimento de alegria abnegada, mesmo que obscurecido, pode surgir. É importante
não negligenciar o seguinte -- a cadeia de boas acções que praticámos, as
palavras gentis que saíram naturalmente e que foram o acto certo na altura
certa. É importante não descurar isto, pois com tanta frequência o fazemos.

A equanimidade contém o espaço empático e permite que as coisas se revelem. Não
exige resultados, mas sintoniza-se com a realidade tal como ela é no momento
presente. É aqui que o assunto do kamma chega ao seu fim, pois a equanimidade
está impregnada da compreensão que, em última análise, ninguém `fez' fosse o que
fosse. Existiu um padrão, baseado em acções anteriores e naquilo que cada pessoa
fez com essas acções. No mundo, em termos gerais, há uma grande herança de
padrões de agressão, baseados na violência e na privação -- e quem sabe onde
tudo isto começou? Mas ao invés de nos agonizarmos e culpabilizarmos, podemos
encarar as nossas acções e as das outras pessoas em termos de causa e efeito.
Este olhar traduz a equanimidade, a base mais fiável para a acção.

\section{O grande coração}

No decurso do nosso trabalho com os padrões cármicos, precisamos desenvolver um
`propósito incomensurável', bem como outros tipos de robustez para nos ajudar
tanto com as tendências involuntárias como com a imagem que temos de nós
próprios -- a forma como habitualmente nos vemos. Isto envolve o desenvolvimento
de um grande coração e de um discernimento profundo.

Este desenvolvimento duplo baseia-se na consciência dos padrões. Isto é o que
ocorre a partir da meditação: podemos reconhecer um padrão negativo, tal como
uma tensão residual, uma irritabilidade, uma sensação de inadequação ou um peso
no coração. As disposições negativas podem surgir, a mente pode sentir-se
toldada e cansada e as memórias, as tonalidades das disposições e os
\emph{flashbacks} podem surgir com uma intensidade pungente. Isto produz um peso
desagradável, um sentimento de sermos alguém que carrega anos de história e de
hábitos acumulados... O padrão é sentido como uma quantidade enorme de bagagem
-- como largá-la? O que existirá ainda para além disto? De igual forma, se
agarrámos nalguma bagagem apenas por estarmos vivos, é provável que vamos
continuar a apanhar mais! Então como é que nos livramos do peso e da viscosidade
associados a estarmos vivos?

Bem, se sentimos aversão a isto, esta apenas vai contribuir para aumentar o
peso. Se mantemos a opinião que somos como somos devido ao que os outros nos
fizeram (ou não fizeram) e nos resignamos a isso, essa resignação vai fixar
ainda mais os padrões antigos, em vez de libertar. Simplesmente dizermos a nós
próprios para sairmos desse estado, não vai limpar seja o que for. Se ignorarmos
a natureza dos nossos padrões ao nos absorvermos no que vemos, ouvimos,
saboreamos e pensamos no momento, podemos não estar conscientes dos padrões
durante algum tempo, mas quando a música pára... voltamos novamente a nós, com
as nossas mudanças de estado de espírito e a nossa auto-imagem desgastada.
Entretanto, as acções que empreendemos para fugir de nós próprios, bem como os
actos de negligência e de distracção, constituem kamma com os respetivos
efeitos.

Largar este peso acontece quando vamos ao encontro destes padrões com um coração
grande.\pagenote{Coração grande:

  ``Anteriormente a minha mente era estreita e subdesenvolvida; mas agora a
  minha mente não tem limites e é bem desenvolvida. Nenhum kamma possível
  permanecerá nela, nenhum kamma persistirá nela.''
  \href{https://suttacentral.net/an10.219/en/bodhi}{AN 10.208}}
Isto envolve o cultivo de uma forte corrente de determinação. Fazemo-lo a partir
do cultivo dos três fluxos dos padrões (corpo, coração/mente e
pensamento/discurso) de forma a abandonarmos a trajectória negativa e
estabelecemos uma trajectória baseada na clareza e na bondade.

Quando fazemos isto na meditação, a consciência da respiração pode espalhar-se e
refinar qualquer efeito positivo através de todo o sistema nervoso. Quando a
mente se afasta, podemos dirigi-la de volta com um simples pensamento: `onde
estou neste momento?' ou `onde está a minha respiração no meio disto?' Então a
forma energética do padrão negativo -- a sua constricção obstructiva ou o seu
empurrão -- dissolve-se gradualmente na corrente da presença constante. É desta
forma que se desenvolve \emph{samādhi} -- concentração. Este permeia a base
emotiva/impulsiva da mente com correntes mais profundas do que as do contacto
dos sentidos e do pensamento discursivo, de forma que uma firme sensação de
bem-estar age como uma quilha de um barco, com o intuito de evitar que as
memórias e as disposições tenham um efeito assoberbante sobre a mente. Este
processo torna a mente grandiosa em termos das suas fronteiras energéticas e da
sua capacidade, pois temos alguma gravidade que não é apenas tensão interna.

Neste mesmo alinhamento, desenvolvemos um coração grande ao cuidarmos deste com
determinação incomensurável. A partir desta perspectiva, se a mágoa, a agitação
ou o medo surgem, em vez de revivermos os antigos padrões de sentimentos de
desolação, de tentarmos arranjar uma solução ou de analisarmos o problema,
podemos interrogarmo-nos: `Como é que eu estou com isto, neste momento?' O
objectivo não é fugir ao assunto, mas sim ter uma perspectiva destacada sobre
este - permitir que uma consciência cuidadosa chegue à história que leva à
emoção. Pode ocorrer um estado emocional entorpecido, tenso ou agitado
juntamente com uma tensão no peito ou palpitações no coração. Não vá por aí --
em vez disso, encontre uma zona no seu corpo onde sinta conforto ou firmeza e
espalhe a consciência a partir dessa zona até à fronteira da área difícil.
Posicione-se interiormente de forma a que o seu coração possa ser observador e
compassivo, tendo o sentimento presente mas sem se identificar com ele. Se
sustivermos esta empatia e firmeza, o coração grande desenvolve-se. Possui uma
correnteza positiva que pode ajudar a alinhar, elevar e refrescar o corpo e a
mente -- e nós apenas nos sentamos nesta correnteza e nos banhamos, assim como
banhamos também as zonas atingidas, até que o sistema entra num equilíbrio e se
sente refrescado e renovado. Isto constitui a sanidade básica. Se nos vamos
inserir num mundo de causa e efeito aleatórios, quando estamos desconfortáveis,
tensos ou deprimidos, estaremos a expor-nos à possibilidade de criarmos kamma
inadequado. Mas com o coração grande não nos tornamos brutalizados, defensivos
ou reactivos.

\section{Destituir o `Tirano Interno'}

Normalmente, quando uma disposição negativa surge, instala-se e infecta a mente
inteira -- tornamo-nos essa disposição, com a sua forma característica. Isto
constitui a grande fraqueza da mente não desenvolvida: torna aquilo `que eu
sinto' naquilo `que eu sou'. Dá-se um apego, uma contracção, e somos sugados
para a história, ficamos hipnotizados por ela e fazemos repetidamente novas
versões da mesma. Fixamos os detalhes: `ela disse isto há cinco anos e ontem ele
fez isto', ou embarcamos novamente no `estou sempre ansioso e não consigo ser
bem-sucedido'. Mas, quando existe um coração grande, este pode confrontar-se com
essa fracção de narrativa sem ser sugado para dentro dela. E, a partir daí, pode
ocorrer uma resposta adequada, ao invés de um envolvimento ou de uma reacção.

\enlargethispage{\baselineskip}

Isto é crucial, porque tentar modificar o nosso estado de espírito negativo de
forma mais directa nem sempre constitui a solução, pois antes de mais, nem
sempre se trata do nosso kamma. Podemos estar a carregar os padrões psicológicos
que não resultam daquilo que fizemos mas sim daquilo que nos foi feito, ou da
forma como fomos educados. Se sofremos perseguições dos nossos colegas na
escola, ou fomos discriminados devido à nossa etnia ou género, a única coisa que
podemos ter feito é carregar o mau kamma de outras pessoas. Neste caso o que
precisamos fazer é não somente abordar os estados de insegurança, a sensação de
intimidação ou de ressentimento, mas fundamentalmente, abordar a noção de que
este padrão cármico é quem `eu sou'. Se abordamos a forma como compreendemos
esse kamma e não continuamos a vê-lo como um aspecto da nossa identidade, o
estado de humor desvanece-se por si próprio.

\enlargethispage{\baselineskip}

Qualquer análise dos padrões psicológicos tende a ter em conta o `Tirano
Interior'. Provavelmente já o conheceu: trata-se do parceiro da nossa `visão
própria' afligida. O Tirano é a voz incómoda que irá sempre exigir que atinjamos
objectivos de perfeição impossíveis, que nunca felicita nem exprime gratidão,
que exagera os pontos fracos, que nos atribui a total responsabilidade dos
acontecimentos mesmo que apenas tenhamos sido parte deles e, com base nisto,
mostra-nos indiferença, censura-nos e castiga-nos. Por vezes, o Tirano apenas
nos proporciona uma autoestima fria e condescendente. Outras vezes, o Tirano
incita-nos constantemente a que façamos mais, a perdoar os outros, a controlar
as nossas emoções e assumir a responsabilidade -- conselhos que podem ter o seu
lugar, mas que são completamente inadequados no que diz respeito à mudança da
nossa perspectiva sobre nós próprios. Simplesmente ainda enraíza mais a crença
de que `eu sou assim'. Trata-se do peso, do peso dos padrões e dos programas que
estamos a tentar mudar. E resulta da acção involuntária em adoptar padrões
psicológicos, tais como o sentido de eu próprio. Não faz sentido, mas todos nós
o fazemos (existe sempre a crença que hei-de encontrar um que será satisfatório
e que vai servir!)

As acções do Tirano, que nos incita a seguirmos programas de punição, resultam
da falta de empatia. Os cenários são exagerados, os veredictos severos, os
castigos apenas tornam as coisas piores e não saram seja o que for -- mas o
Tirano não consegue agir de outra forma. O Tirano encontra-se preso -- trata-se
de uma parte do \emph{vipāka} que ficou emperrada. Não nascemos com isto, mas
desenvolvemo-lo devido ao ambiente humano confuso e pouco empático. A
necessidade social de competir com os outros e de evitar sermos de segunda
categoria, não nos permite ter empatia face àquilo que nós ou que os outros
estão, na realidade, a experienciar. Debaixo desta pressão, a mente divide-se
entre aquilo que `estou a sentir' e aquilo que é suposto que `eu seja'. Desta
forma, a empatia e a integridade são abandonadas em prol do sucesso e do
desempenho. A pressão social é mantida no lugar ao ser interiorizada sob a forma
de dois `eus': o `Tirano Interior' constitui o agente de pressão e a sua vítima
é o `Pequeno Eu'.

Desde que continuemos a ser o Pequeno Eu, a vítima, apoiamos a fragmentação e,
claro, o Tirano. Por vezes o Pequeno Eu revolta-se, ou procura afirmar-se de
forma a tornar-se no Grande Eu. E desta maneira o Tirano levou-nos a criar outra
imagem de nós próprios -- uma imagem que não consegue auto-sustentar-se sem um
fluxo contínuo de nutrição do ego. De forma que temos de abandonar esta
tendência para criar autoimagens e em vez disso, restaurar os padrões de empatia
de uma mente equilibrada. E isto faz-se através do kamma adequado de termos
presente e sentirmos a energia e a sensação de um pensamento, disposição ou
padrão, ao invés de os seguir ou acreditar neles. É por isso que precisamos de
um coração grande.

É particularmente importante cultivar a alegria abnegada. Quando o coração é
grande nesse sentido, consegue manter o Tirano sob controlo e perscrutar as suas
narrativas até atingir um significado mais profundo da sua própria consciência
saudável. Pode trazer à mente a sensação `Sou maior do que este Tirano, não lhe
devo quaisquer favores, não preciso disto.' `Valorizo estar aqui, mesmo com a
minha tristeza ou insegurança. Posso estar presente e sentir compaixão por isto
sem precisar de alterar nada.' Isto porque o simples `estar' numa consciência
compassiva, sem tentar resolver, sem culpar e sem alterar seja o que for, só por
si é benéfico. Não estamos a agir a partir de um `Pequeno Eu' contraído e
carente. Depois a transformação pode ocorrer. Deixamos de ser o `Pequeno Eu',
libertamo-nos das suas histórias e podemos conscientemente e compassivamente
ouvir o Tirano a vociferar e a resmungar -- eventualmente até nos rimos.

Esta forma de desconstruir o Tirano é completada quando o incorporamos. Ou seja,
tendo estabelecido o coração grande, de forma a estarmos aptos a testemunhar as
queixas e a dureza, mudamos para a experienciação dos programas do Tirano na
perspectiva da primeira pessoa. Em vez de termos o Tirano a referir-se a nós,
sentimo-lo e integramos o programa. Escutamos realmente a voz do Tirano e
imaginamos o seu aspecto. Imaginamos como seria ser aquele Tirano. De seguida,
adoptando o ponto de vista do Tirano, descobrimos aquilo que queremos. Detesta
todas aquelas disposições idiotas e fraquezas? Pois bem. Esteja presente com
elas e sinta através dessa energia.

Quer controlar tudo e todos? Pois bem. Esteja presente com e sinta através
disso, sinta essa energia no seu corpo -- até que ocorra uma empatia que inunde
o programa por completo. Com isto, o `Pequeno Eu' desaparece e, à medida que
isso acontece, os objectos dos desejos do próprio Tirano e, finalmente o próprio
Tirano, desconstroem-se. Por estranho que pareça, este processo traz uma
aprendizagem muito profunda e poderosa relativamente aos nossos reflexos, uma
experiência de largar as formas do `eu' pelo Dhamma. E, no meio disto, aquilo
que frequentemente é necessário não é sermos alguém com uma resposta, mas sim
unificarmo-nos em redor da empatia -- porque a perda de empatia e de unidade
foram as causas prioritárias de todo este cenário problemático. Então, quando
oferecemos a nós próprios uma energia tranquila e empática, o nosso próprio
coração, naturalmente grande, é-nos devolvido.

\section{Opinião, discernimento e sentido de si próprio}

Assim, para limpar o kamma antigo temos de ir ao seu encontro e lidarmos com ele
de forma a este revelar os seus padrões. Mas isso necessita de uma consciência
matura, do coração grande. De outro modo, sempre que a forma como nos sentimos
se torna em quem somos, o kamma novo segue o padrão antigo. E, com base nisto,
surge um `sentido de mim próprio': eu enquanto inteligente, eu enquanto carente,
eu enquanto incompreendido -- começo a ficar preso nestas personagens. Mas lutar
com elas ou qualquer outra atitude que adopte estas personagens como verdadeiras
e reais, não as liberta. Porque a noção `eu sou' constitui o resultado (não o
agente) do kamma com o qual estamos a tentar lidar. Trata-se de uma cauda a
perseguir o cão. E o abanar da cauda apenas cria mais kamma.

Esta auto-imagem é uma ilusionista. Já alguma vez planeou o ambiente ideal para
a meditação, ou ensaiou uma dezena de diferentes estratégias sobre como irá
lidar com uma situação na sua vida... para descobrir simplesmente que o fluxo da
actualidade sofre sempre alterações relativamente àquilo que antecipava?
Subjacente a estes planos encontra-se a auto-imagem, que precisa de saber e de
sentir que tem tudo sob controlo. Se seguirmos esta ideia de quem somos, daquilo
que queremos e daquilo que receamos, ao tentarmos criar um `eu seguro' no
futuro, estaremos a criar um `eu agitado' no presente. É por isso que o Buddha
não enfatizava o `eu'. O interesse do Buddha era visar e reconhecer o sofrimento
e tensão subjacentes.

Deste modo, o conselho é não nos metermos com o `eu' e a sua história, mas em
vez disso aplicarmos a atenção adequada no sentido de atenuarmos as energias de
contracção, suavizarmos o impulso da postura defensiva ou iluminarmos a
insipiência da resignação. Não quer dizer que não se justifique termos de nos
conformar com determinada injustiça, porém, tudo o que por vezes podemos fazer,
é ter em mente os padrões e a auto-imagem que isso cria. Conheço pessoas com
cancro que se recusaram a entrar em desespero e resignação e a tornarem-se
vítimas. E pessoas que sobreviveram a regimes totalitários por nunca aceitarem o
cenário da perda de liberdade. Este é o tipo de intenção que pode parar os
padrões psicológicos. E quando o padrão é retido, o `eu afligido', o `eu'
enquanto vítima, não é criado. Esta é a compreensão-sabedoria: começamos a
conhecer os nossos muitos `eus' como não sendo `eu próprio'. Começamos a ficar
cada vez menos interessados nos seus programas e não os alimentamos. E com isto,
não necessitamos de continuar a reviver a calamidade do \emph{saṁsāra}.

Contudo, por outro lado, aqui estamos nós: com o surgir da consciência, com o
pensamento, a visão e o acto de alimentar os miúdos, surge o contacto, e com
isso surge a noção de `eu aqui' a ser afectado por `aquilo'. Esse sentido de
`eu', a impressão ou o impulso subjetivo, constitui uma referência relativa à
onda de sentimentos e de intenções em mudança, à medida que estes surgem com o
contacto. Mostra-nos como o kamma nos está a afectar e como lhe estamos a
reagir. E sim, precisamos de um sistema operativo com um operador coerente. É
esta a razão pela qual o Buddha não negava categoricamente o eu: constitui um
\emph{locus} de consciência que pode ser desenvolvido de forma a transportar
programas adequados. E precisa de ser investigado. Precisamos de ser claros e
adequados no que diz respeito a esse eu aparente. Essas energias, esses
\emph{saṅkhāra} que transportam o processo da consciência, desempenham funções
que são relevantes para a encarnação. Temos de agir e de nos relacionarmos com o
nosso contexto do quotidiano. É bom acordar a cada manhã como, mais ou menos, a
mesma pessoa, conhecer a linguagem e ter um corpo que consegue funcionar no seu
ambiente. Por isso, sim, novamente eu. Mas ao ter conhecimento que qualquer
sentido de nós próprios é uma função da consciência, ao invés de uma identidade,
aprendemos a ser indivíduos no mundo da natureza e das outras pessoas, em vez de
ignorarmos as preocupações dos outros, de sermos obcecados connosco próprios ou
intolerantes. Assim, as energias que produzem um centro pessoal e um objetivo
estão bem, se puderem ser usadas para responder de forma clara às coisas tal
como são. Tudo o que precisa de ser abolido é o pressuposto subjacente que
procura e forma as opiniões que fazem do `eu' uma entidade sólida e duradoura.

Os pontos de vista são o instigador do kamma: agimos consoante aquilo em que
acreditamos. Os pontos de vista são um íman que atrai a energia da vontade e das
tendências: desenvolva uma determinada atitude e pode estar certo de que a sua
mente irá interpretar a realidade com base nisso. Mas se notarmos a forma como
os pontos de vista atraem a energia... e como a energia cria um padrão... e como
um padrão mental se torna uma convicção... e como uma convicção se torna num
`facto inquestionável' -- é assim que surge o `eu'. De modo que, desde o momento
em que exista a necessidade de um ponto de vista, a necessidade de ser e de
provar, então essa necessidade irá apoiar uma autoimagem. Aí, se nos agarramos a
esse ponto de vista, a seu tempo irá surgir o conflito com os outros e o não
conseguir aquilo que queremos. Mas se a energia puder seguir outro caminho,
gerando um padrão de solidez de base, de empatia e de coração grande, então o
ponto de vista e a opinião podem mudar. Desanuvia-se com a descoberta de que
`tudo isto, toda esta energia, é invocada pelo \emph{saṅkhāra}, modelada pela
consciência, com um significado atribuído pela percepção, reverberado pelos
sentimentos, produtora de intenção, resultando em efeitos... Tudo isto está em
mudança, é insubstancial, não existe um `eu' nisto, nem qualquer `eu' pode ser
estabelecido para além disto'. Consequentemente, agimos com integridade e não
nos agarramos. E aí não existe nem peso nem tensão.
