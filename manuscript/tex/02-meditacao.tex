\chapterNote{Sentar-se tranquilamente}

\chapter{Meditação}

\tocChapterNote{Sentar-se tranquilamente}

Sente"-se confortavelmente num local tranquilo. Descontraia os olhos mas
deixe"-os abertos ou semicerrados, com um olhar descontraído. Esteja consciente
da sensação dos globos oculares a descansar nas respetivas cavidades (em vez de
focar o olhar naquilo que vê). Esteja sensível à tendência dos olhos para não
pararem quietos e descontraia"-os constantemente. Como alternativa, pode achar
útil deixar o olhar descansar, de forma descontraída, num objecto propício, tal
como uma paisagem distante.

Seguidamente leve a atenção para as sensações nas suas mãos, depois para o
maxilar e para a língua. Veja se também estas podem deixar, por um momento, de
estarem sempre apostos para entrar em acção ou para se defender. Deixe a língua
descansar na parte inferior da boca. Seguidamente percorra, com essa atenção
relaxante, uma área desde os cantos dos olhos até à volta da cabeça, como se
estivesse a tirar um lenço. Deixe que o couro cabeludo se sinta liberto.

Deixe os olhos fecharem"-se. À medida que descontrai a cabeça e a cara a toda à
volta, traga essa qualidade de atenção, lentamente, gradualmente, mais abaixo
para a garganta. Solte essa zona, como se permitisse que cada exalação soasse
como um zumbido inaudível.

Mantendo o contacto com essas zonas do corpo, ciente do fluxo de pensamentos e
de emoções que passam pela mente. Oiça"-os como se estivesse a ouvir a água a
correr ou o mar. Se sentir que está a reagir"-lhes, leve a atenção para a
próxima expiração, continuando a descontrair através dos olhos, da garganta e
das mãos.

Se quiser expandir o processo, percorra com a sua atenção o corpo até às plantas
dos pés e, desta forma, construa toda uma sensação de à-vontade no corpo.

Mantendo a consciência da presença do seu corpo como um todo, pratique o
distanciamento ou largue quaisquer pensamentos e emoções que surjam. Não lhes
acrescente nada, deixe"-os passar. Sempre que fizer isso, repare na sensação de
espaço, mesmo que breve, que parece estar aí presente, por detrás dos
pensamentos e dos sentimentos. Entre em sintonia com essa tranquilidade.

Ao sentir essa tranquilidade, acolha-a. Em vez de exigir ou tentar atingir um
estado de calma, crie uma prática na qual serenamente oferece paz às energias
que o atravessam.
