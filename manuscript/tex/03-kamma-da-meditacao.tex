\chapterNote{Limpar os programas}

\chapter{O Kamma da Meditação}

\tocChapterNote{Limpar os programas}

\begin{quote}
  ``A sabedoria surge da meditação. Sem meditação a sabedoria diminui.
  Tendo em conta estas duas formas de progresso ou de declínio, oriente a sua conduta
  de forma a aumentar a sabedoria.''

  \quoteRef{\href{https://suttacentral.net/dhp273-289/en/buddharakkhita}{Dhammapada 282}}
\end{quote}

A meditação é uma actividade profundamente transformativa. Isto pode soar
estranho, uma vez que a meditação não parece algo muito activo: implica
frequentemente estarmos sentados, quietos e, para mais, em silêncio. Quanto a
fazer seja o que for com a mente\ldots{} tudo o que a meditação aparentemente
envolve são algumas coisas supostamente inconsequentes, tais como levar a
atenção a sensações associadas com a respiração, ou talvez testemunhar os
pensamentos, à medida que passam. A meditação não parece, de todo, um processo
muito significativo. Os principiantes perguntam: `O que é que eu devo fazer
enquanto estou aqui sentado? O que devo fazer com a minha mente de forma a
torná"-la melhor\ldots{} Devo pensar sobre o quê?' Com efeito, um dos aspectos da
meditação trata de moderarmos esta volição de `fazer' e, consequentemente,
sermos mais receptivos.

O ensinamento consiste no seguinte: quanto mais moderarmos desta forma a nossa
energia, maior luminosidade, confiança e clareza iremos atingir. Então, a
inquietude, a preocupação e os impulsos, deixam de surgir como forma de
distracção. E, devido a isto, a meditação pode gerar efeitos de grande alcance
na nossa vida: temos oportunidade de saborear e valorizar a quietude e a
simplicidade, sendo que isso nos leva a querermos menos e a mais facilmente
abrirmos mão.

A meditação centra"-se à volta de duas funções. A primeira é uma espécie de cura,
um tónico. Denomina"-se `acalmar' (\emph{samatha}): o aquietar e abrandar das
energias físicas e mentais. A segunda função é o `discernimento'
(\emph{vipassanā}), que é mais uma questão de olhar para o corpo/mente que se
tornou calmo e compreender as coisas tal como elas são realmente. As duas
funções trabalham em conjunto: à medida que aquietamos, a nossa atenção torna"-se
mais clara; e, à medida que vemos as coisas de forma mais clara, existe menos
agitação, confusão ou coisas para resolver. E a conclusão destes dois processos
centra"-se na condução da mente -- ou melhor das disposições, atitudes e memórias
que nos impulsionam -- para um local de resolução. A meditação diz respeito a
uma acção que leva ao fim da acção.

\subsection{Programar: corpo, mente e racionalidade}

Começamos a meditação expandindo uma consciência firme no corpo enquanto este
está sentado, a andar, em pé, reclinado\ldots{} e no seu processo contínuo de
inspiração e expiração. Pomos de lado preocupações e circunstâncias temporárias,
e damos atenção ao nosso sistema de corpo"-mente.

Algo que se torna claro com a introspecção é o quão este sistema é dinâmico: as
sensações físicas pulsam e mudam, e as suas energias afluem e fluem. No plano
mental, as disposições alteram"-se, os pensamentos correm e despoletam memórias,
planos\ldots{} -- o que quer dizer que raramente estamos completamente presentes com
aquilo que estamos a fazer no momento. Aquilo que se encontra aglomerado como `o
meu corpo' e `a minha mente' é, na realidade, uma dinâmica incessante de
sensações, disposições e impulsos que abrandam, aceleram e se alteram
constantemente. 

A inteligência do corpo proporciona"-nos um sentido de
localização, contudo estamos presentes com ele apenas durante os momentos
necessários pois, entretanto, a inteligência emocional está a dizer"-nos como é
que nos sentimos e a faculdade racional diz"-nos aquilo o que devemos ou o que
não podemos fazer. Estas inteligências interagem -- disposições e pensamentos
enviam afluxos, ou mesmo choques, ao sistema energético do corpo e vice"-versa.
Por vezes um surto de irritação ou de medo vai causar uma certa contracção, ou a
noção de ter imensas coisas para fazer gera um rodopio, no qual perdemos a
consciência do corpo. E, apesar disto parecer o `eu', não possui uma substância
duradoura. A substancialidade é apenas criada pela névoa e interacção constantes
das energias física e mental, semelhante ao disco aparente que é criado pelas
lâminas de uma ventoinha em funcionamento.

Estas energias em interacção são os nossos programas físicos, mentais/emocionais
e racionais (\emph{saṅkhāra}). Os programas são instruções codificadas que
associamos ao \emph{software} informático. Mas eles não são apenas uma invenção
moderna. Longe disso. A capacidade de raciocinar e de usar a lógica é um
programa. E, tal como a nossa mente racional é programada para delinear planos e
definir razões, a mente afectiva está programada para ser afectada por
sentimentos e impressões e para formular impulsos e respostas. O corpo é
igualmente programado para funcionar da forma como funciona, e para produzir e
fazer circular energia, interligada com a respiração. Estes são programas
funcionais que são estabelecidos pela força vital (\emph{ayusaṅkhāra}).

Sobre estas bases elementares, são construídos programas mais complexos -- mais
\emph{saṅkhāra}. Ou seja: o \emph{saṅkhāra} do acto de pensar vai ficar
programado em atitudes e formas de pensar específicas. E o nosso programa
emocional de gostar e de não gostar fica afinado especificamente para uma gama
de respostas. Desta natureza constante de estabelecimento de padrões, um outro
nível de programação interpreta tudo isto como `eu', `meu'. Essa interpretação
cria, então, um centro para preconceitos subjectivos, tendências e aversão, que
dá origem ao comportamento complexo. Todos estes comportamentos, padrões e
programas, misturados com a sobrevivência, os preconceitos e a tensão são
denominados `\emph{saṅkhāra'}.

Os programas \emph{saṅkhāra} são simultaneamente activos (a intenção, o impulso
para fazer, põe o processo em andamento) e resultantes, na medida em que, uma
vez estabelecidos, tornam"-se nos padrões normais de cada indivíduo, a sua forma
de pensar, de responder em termos emocionais e de sentir a energia do corpo.
Desta forma, são os \emph{saṅkhāra} que programam o nosso comportamento e que
transportam o \emph{kamma}.

Para utilizar uma analogia, se limparmos um trilho no mato, é criado um padrão
que tem como resultado encorajar os outros a seguir esse caminho. Se esse
percurso for seguido vezes suficientes temos uma autoestrada, um programa
estabelecido. Tal como os carros numa estrada, em breve haverá uma grande
quantidade de trânsito a seguir por essa estrada, que não irá por outro
percurso. De igual forma, na vida, como resultado de atitudes e de preconceitos,
a nossa forma de pensar e de agir segue um percurso habitual. Se temos reagido
sempre de uma determinada forma -- por exemplo relativamente a cães ou a
multidões ou a não conseguirmos o que queremos -- então essa reacção fica fixada
como um facto real e inalterável que os `cães são horríveis' ou que nós temos um
problema no que diz respeito aos cães. As nossas emoções não seguem outro
percurso perante os cães -- é esse o programa no que diz respeito a caninos. À
medida que este programa se fixa, estabelece"-se a impressão de que `eu tenho'
estas atitudes e comportamentos. Para além disso, se este programa fica
realmente solidificado, dá origem ao pressuposto `isto sou eu e eu não consigo
mudar, ou não mudo'. 

Esse padrão resultante, essa opinião sobre nós próprios, 
torna"-se uma identidade. E essa identidade é uma parte enorme e significativa da
programação. Torna"-se a base para nova acção -- digamos que eu evito cães ou que
fico tenso se estou perto de um cão. E isso é apenas uma pequena parte de `mim'.
Assim, numa série de diferentes níveis, os padrões e os programas são o meio
para as acções e para os efeitos que nos caracterizam: `isto sou eu, esta é a
minha maneira de ser e eu rejo"-me por isso'.

O objectivo da meditação, e na realidade de toda a prática do \emph{Dhamma}, é a
libertação dos programas e o desenvolvimento de nova programação -- é esse o
programa da meditação! Como o paradoxo sugere, a prática envolve a utilização da
mente de formas particulares, de modo a neutralizar os programas negativos e
gerar programas mais adequados. Envolve igualmente uma maior consciência
relativamente aos programas que se baseiam na ignorância, bem como na sua
desinstalação. O apoio para a meditação reside no facto de, à medida que o
\emph{saṅkhāra} superficial dos pensamentos e das emoções acalma, serem
revelados os preconceitos subjacentes, as ânsias, a resistência e a confusão.
Com a continuação da estabilização e da renúncia face aos preconceitos, a mente
pode descansar e ficar mais pura. É essa a prática. Contudo, trata"-se de um
processo profundo: alguns destes programas angustiantes estão latentes, ao invés
de activos, e, tal como sementes no solo seco, apenas germinam quando a chuva
cai. Quando o nosso sistema interno se encontra agradável e solarengo pode
parecer que não temos quaisquer tendências no sentido da má vontade ou dos
desejos sensoriais, mas pode acontecer que estas estejam adormecidas. Assim,
para ser realmente benéfica, a meditação é algo que empreendemos como um
processo, de forma a expor e resolver esta rede de padrões, quer faça sol ou
quer faça chuva\ldots{}

O processo meditativo tem de trabalhar através de três programas: corpo,
pensamento/discurso e coração.\pagenote{Três tipos de \emph{saṅkhāra}:

  ``Existem três formações\ldots: a formação corpórea, a formação verbal e a
  formação mental\ldots{}''

  ``A inspiração e a expiração\ldots{} constituem a formação corpórea; o
  pensamento aplicado e o pensamento continuado constituem a formação verbal; a
  percepção e a sensação constituem a formação mental.''

  \href{https://suttacentral.net/mn44/en/sujato}{MN 44.13-15}}
O programa crucial é \emph{citta"-saṅkhāra}, que ocorre com \emph{citta} -- a
mente como `coração'. Esta mente é o sentido afectivo que experiencia o
significado e os sentimentos, e produz respostas e atitudes. Devido à
\emph{citta}, devido ao facto de sermos sensíveis, interpretamos e atribuímos
significados -- somos levados no sentido da alegria ou da tristeza. Então, a
partir do sentimento de prazer ou de desapontamento, etc., relativamente àquilo
que nos tocou, são formulados mais propósitos deliberados -- decidimos agir com
base num pensamento ou num impulso. E assim temos novo kamma e, como resultado
deste, certos padrões: favorecemos e desenvolvemos gostos que se tornam `no meu
estilo, nas minhas atitudes, na minha visão do mundo'.   

A habitual natureza de
tudo isto significa que vemos as situações de acordo com a forma como estamos
habituados a vê"-las e respondemos a elas de modos padronizados. Então parece
que: `Este tipo de situação faz"-me sempre sentir desta forma.' Faz-`me'
realmente, não faz? O sentido de `eu sou' surge de forma muito poderosa quando
os nossos sentimentos são despoletados. Não nos apercebemos nem penetramos este
despoletar, que nos leva para aquilo que é por vezes doloroso, por vezes
aprazível, mas que, de forma geral, é território emocional familiar. É familiar
porque se trata de um padrão ou de um programa, frequentemente com um `eu' no
seu centro, reconhecível como independente, vitimizado ou folião. E, há medida
que o `eu' é aí padronizado, agimos de acordo com a sua programação, somos
controlados (e por vezes esmagados) pelas suas disposições e pelos seus
impulsos. Há uma névoa indistinta\ldots{} uma aceleração\ldots{}, uma reacção\ldots{}, e
algo foi dito ou feito a partir dessa pré"-concepção. De acordo com tudo isto, as
nossas vidas fazem viragens através de pressupostos, pré"-concepções subjectivas
e impulsos, que uma lembrança consciente não teria deixado acontecer. Este
funcionamento automático encontra"-se ancorado no preconceito subjacente da
ignorância (\emph{avijja}), que constitui a programação mais fundamental para o
nosso sofrimento e para a nossa tensão.

A ignorância cega"-nos, e assim deixamos de ver. Contudo, podemos reconhecer a
tendência para lidar com ou julgar antecipadamente as situações através de
percepções distorcidas: `eu acho que devias ser\ldots', `as mulheres são\ldots'.
E depois torna"-se `o meu problema é ser\ldots' Mas quando uma análise e uma
opinião antecedem (ou até substituem) a ocasião da qual deveriam resultar, algo
tem de estar errado: não estamos, realmente, plenamente presentes com aquilo que
está diante de nossos olhos; encontramo"-nos dentro de um programa, a falar para
as nossas percepções e impressões já previamente formuladas. Contudo, gostamos
muito destas atitudes fabricadas pois elas são confortáveis, e evitam o
`processo incómodo' de estar perante as situações de uma forma `nova e fresca'.
Elas são kamma antigo, armazenado como \emph{citta"-saṅkhāra}.

Mesmo se a nossa atitude exclui ou suprime as nossas emoções, isso também
constitui um padrão e um programa. Tal como quando nos deleitamos na emoção e a
afirmamos como `isto é verdadeiro e real'. Desta forma, aparentemente sem
fazermos grande coisa, existe kamma mental. Pode ser bom, mau ou misto, mas
forma um ponto de partida mental, a partir de ressonâncias, momento a momento.
E, através disto, sentimo"-nos muito preocupados, muito atarefados a sermos `eu'.

O programa para o kamma verbal, \emph{vacī"-saṅkhāra}, formula pensamentos e
discurso. Opera através de um processo dual. Em primeiro lugar, a mente racional
examina a impressão sensorial e dá"-lhe um nome: `vaca', `sino' -- isto constitui
`trazer à mente' ou `pensamento dirigido', \emph{vitakkā}. Em conjunto com isto,
ela verifica como é que esse conceito se aplica (afina"-se ou avalia) e pode dar
origem a mais conceitos, como por exemplo `a vaca parece doente/zangada' -- Isto
é avaliação, \emph{vicāra}. Toda esta dinâmica é alimentada por um impulso para
definir, ser claro e planear, mas quando o programa se torna compulsivo, a mente
fica cheia com circuitos de \emph{feedback} de concepção e avaliação,
planeamento e ponderação.

\enlargethispage{\baselineskip}

Podemos ser pensadores incessantes ou preocupados, ou alguém que gosta de
pensar, que aprecia a capacidade para gerar ideias. Ou o nosso pensamento pode
sair em golfadas, ir para a frente e para trás, confuso, irregular e
desconfortável -- pensar sobre como parar de pensar. Podemos ficar agitados,
absorvidos no nosso diálogo interno e não ver as coisas directamente como elas
são. De forma que os programas verbais afectam a mente: ficamos agradados,
fascinados ou deprimidos com os nossos pensamentos e a nossa capacidade para
pensar. Logo o kamma verbal alimenta o kamma mental e torna"-se uma fonte para a
acção.

\enlargethispage{\baselineskip}

Por último, o programa para o kamma físico é \emph{kaya"-saṅkhāra}, o fluxo de
energia física relacionada com a respiração. Devido à respiração, o corpo, em
estado enérgico ou de descanso, é um processo dinâmico. E a sua vitalidade (ou a
sua ausência) constitui para nós uma satisfação, um fascínio ou um
desapontamento. De forma que este processo afecta a mente. Assim, o programa
crucial, aquele que todos nós alimentamos, é \emph{citta- saṅkhāra}. Para além
disso, é através das suas interpretações e respostas que é gerado o kamma novo.
Essa nossa opinião transporta igualmente o preconceito subjacente da ignorância.
Desta forma, para alcançarmos clareza e nos libertarmos, trabalhamos sobre os
padrões e programas da mente, recorrendo a dois programas de meditação
\emph{samatha} e \emph{vipassanā}. Com \emph{samatha} acalmamos e estabilizamos
todo o sistema -- clarificando e dirigindo a faculdade do pensamento, iluminando
e estabilizando a energia física e suavizando o coração. E, à medida que fazemos
isto, revemos os programas \emph{saṅkhāra} com discernimento: quando interpreto
este pensamento ou esta energia como sendo minha ou sendo eu, isso resulta em
tensão e stress? E como é que esta disposição é sentida como `minha'? O
objecctivo do discernimento é, concretamente, limpar a nossa auto"-imagem, os
seus preconceitos, apegos, defesas e necessidades daí resultantes.

\subsection{Trabalhar com a programação na meditação}

A forma como damos atenção e aquilo em que reparamos afecta"-nos -- trata"-se do
programa \emph{citta} básico. Por isso é importante que na prática da meditação,
o coração esteja envolvido e participe de forma adequada, de forma a dar"-se a si
próprio tempo para aproximar"-se da sua experiência de forma amigável. Despenda
algum tempo a reparar na sensação de espaço ao redor do seu corpo -- tempo
observar e estabelecer"-se. Faça-o de forma a que não haja nada que `tenha de
ser', desenvolver ou arranjar. Dê a si próprio tempo para estar presente e
aprofundar"-se em atitudes simples, momento a momento, com benquerença: `que eu
esteja bem', `que os outros estejam bem'.

Tudo isto afecta a intenção, que comanda o processo de programação. Quando o
hábito da sua vida activa implica apressar"-se, fazer coisas, resolver coisas e
obter resultados, sair desse tipo de intenção é uma mudança significativa. Para
muitas pessoas, a energia volitiva, particularmente através da mente que pensa,
é largamente preponderante. As nossas mentes aceleram e existe a ansiedade de
conseguir algo. É toda uma atitude relativamente à vida: `A vida é uma luta.
Tenho de trabalhar arduamente. Ir à luta e conseguir que as coisas corram bem
para mim'. Noções como estas dão à nossa vida uma qualidade de ambição, no
sentido de alcançarmos os nossos propósitos, sucesso, etc.. Esta é, basicamente,
a forma de pensar do mundo actual.

Mas se somos demasiado tensos não limpamos nada, e se somos demasiado frouxos
também não limpamos nada. Situado algures no meio encontra"-se o melhor percurso
-- da atenção consciente. Conceda a si próprio todo o tempo do mundo para entrar
em sintonia, de forma simples e calma, com o padrão mais estável, seja ele qual
for, de sensações físicas que ocorrem quando o seu corpo está sentado: a pressão
do corpo naquilo sobre o qual está sentado, a sensação da postura vertical e,
talvez, o fluxo da expiração. Com isto monitorizamos as nossas atitudes e a
nossa intenção. Passamos de `tenho de fazer isto bem' para `vamos simplesmente
considerar isto um momento de cada vez'. Ajustamos o modo e a atitude de
atenção, de forma a estar de acordo com aquilo que torna a mente passível de ser
trabalhada, fluida, interessada -- até brincalhona. Procuramos, no coração, um
tema que pode constituir um apoio.

Quando trabalhamos na meditação com programas verbais/racionais, treinamos a
mente a estar consciente da forma como pensamos e dos resultados que daí
obtemos. Treinamos a `mente pensante' a fazer o que é relevante e suficiente: a
ajudar a atenção e a desconstruir as complexidades do pensamento especulativo ou
analítico. Desta forma, simplesmente notamos `isto é inspirar', `isto é
expirar', `isto é andar'. E avaliamos: `esta sensação é de suavidade', `esta
respiração é longa'. Não temos de fazer sempre uma nota mental, mas pelo menos
dirigimos a nossa atenção. É como se definíssemos aquilo que experienciamos de
forma demarcada, momento a momento: `O corpo está assim. Sinto-o desta forma.'
Sabemos como está quente, como é sólido, quais as suas pressões, e por aí fora.

Para trabalhar com \emph{kaya"-saṅkhāra} levamos em conta a energia física.
Podemos sentir"-nos muito energizados e contentes pelo seu brilho ou pelo seu
vigor, ou aborrecidos com a sua indolência ou desequilíbrios, as suas alterações
hormonais, a sua energia sexual, bem como todas as coisas que constituem a
sensação de estar neste corpo. A forma como este corpo é directamente
experienciado e como eu me sinto neste corpo constitui a área de
\emph{kaya"-saṅkhāra}. É toda a experiência formativa, activa, impulsiva do corpo
-- não se trata de algo que possamos ver com os olhos, não é carne e ossos.
Assim, em vez da sua experiência exterior, sintonizamo"-nos com as pressões e
rubores, peso e temperatura. Termos em conta o corpo deste modo, ajuda"-nos a
estarmos conscientes de como este é afectado e como podemos usar a sua
sensibilidade. E podemos beneficiar do efeito de base de suporte do corpo.

\enlargethispage{\baselineskip}

Vale a pena despender algum tempo a estabilizar o corpo na posição sentada: a
colocar o corpo numa postura verticalizada e a descontrair o que está tenso.
Isto faz com que tenhamos de nos sintonizar com a forma como o corpo está neste
momento, bem como descobrir a melhor forma deste estar sentado, de modo a
manter"-se em vigília, mas sem tensão. Pode levar algum tempo a encontrar um
equilíbrio, devido a hábitos de más posturas ou tensões residuais no corpo.
Pratique igualmente a busca deste equilíbrio quando se encontra em pé ou a
andar. Tenha em conta dois pontos de referência: a coluna vertebral -- tente
estar sentado, em pé ou a andar de forma que o conjunto da coluna esteja
alinhado, desde o alto da cabeça até ao cóccix, como se estivesse pendurado de
cabeça para baixo. Procure um equilíbrio que cause a menor tensão possível. Em
segundo lugar, deixe que o seu corpo sinta o espaço à sua volta. Isto ajuda a
descontrair a zona anterior do corpo. Se se perder em pensamentos ou se sentir
desconfortável, volte simplesmente a estes dois pontos de referência.

\subsection{Relações entre corpo, discurso e mente}

\enlargethispage{\baselineskip}

A forma como posicionamos o corpo tem um significado mais profundo para além de
atingirmos o conforto. Apesar do kamma ser gerado através da \emph{citta},
através das atitudes e emoções, é o corpo que fornece energia para tal. É
difícil manter uma prática feliz de meditação se a energia do seu corpo se
encontra abatida. Também é difícil permanecermos zangados se sentimos a energia
do corpo como descontraída e luminosa: a depressão não perdura. Enquanto que, se
se sente a energia do corpo como enfadonha, a afundar"-se e errática, isso
rapidamente conduz à tonalidade emocional da depressão e da apatia. Se a energia
do seu corpo está acelerada e truculenta, então isso vai fazer com que o seu
coração se sinta agitado. O corpo e a mente estão muito interligados.

Por outro lado, se o seu coração se encontra arrebatado ou tenso, então o seu
corpo vai receber sinais no sentido de lhe proporcionar mais energia, de forma
que o seu sistema nervoso começa a aumentar as rotações. O corpo recebe este
sinal a partir da \emph{citta} -- `mais energia, mais energia, mais energia' --
de modo que ficamos em tensão, prontos a agir. Repare simplesmente a quantidade
de energia nervosa que pode utilizar quando fica emocionalmente excitado acerca
das situações. Repare como isso pode ser extenuante. Treine"-se a encontrar um
bom equilíbrio entre determinação e sensibilidade, que significa que não vai
estar a manter ideais ou imperativos sob formas que não poderá sustentar em
termos de energia e capacidade física. `Vou sentar"-me aqui até conseguir atingir
a iluminação completa' tem maior probabilidade em resultar na ruptura dos
ligamentos dos joelhos e despoletar um conflito ao nível dos pensamentos e das
disposições, do que de levar ao resultado desejado.

Contudo, podemos aquietar a mente afectiva ao sintonizá"-la com os ritmos simples
da respiração do corpo. Podemos trazê"-la para um sítio confortável e íntimo, de
forma a que ela não precise de andar ansiosamente a correr atrás de ou a
agarrar"-se a algo. Podemos respirar através das nossas disposições e descobrir
onde é que elas são mantidas no corpo -- peito tenso, diafragma hirto, garganta
apertada\ldots{} Levarmos aí a nossa atenção, como se fosse uma massagem,
oferece à \emph{citta} um bom sítio para estar, no qual pode fazer um trabalho
inteligente e sentir"-se contente com esse facto. Desta forma, a interconexão
resulta em algo positivo e podemos esvaziar a tensão do coração ao mantê"-lo no
fluxo da energia física. Deste modo, a ligação ao corpo é kamma adequado, uma
vez que, quando estabelecemos contacto com as energias e os ritmos do corpo, o
sistema nervoso, no seu todo, fica tonificado e massajado por um contacto que é
simples e estável.

Os padrões verbais e físicos estão igualmente relacionados. O Buddha disse que
pensar demasiado cansa o corpo.\pagenote{Pensar demasiado cansa o corpo:

  ``Mas, com pensamentos e ponderações excessivos, o corpo poderá ficar cansado;
  quando o corpo está cansado, a mente torna"-se tensa, e quando a mente se torna
  tensa, fica afastada da concentração.'

  \href{https://suttacentral.net/mn19/en/bodhi}{MN 19.8}}
Pensar afecta o sistema nervoso. As pessoas frequentemente sofrem desgaste por
uma sobrecarga de verbalização -- o trabalho administrativo pode ser extenuante
e se não conseguirmos ajustar esta energia para uma frequência mais baixa, o
sistema queima"-se através do sistema nervoso. Mau \emph{kamma}: não através de
más intenções deliberadas, mas através da negligência do sistema operativo da
nossa vida. Assim, preste atenção à energia do discurso e do pensamento: esta
afecta tudo.

Por exemplo, se o leitor pensa que devia parar de pensar\ldots{} vai estar a
lutar. Mas ao notar o efeito energético que pensar tem, a velocidade ou a
contracção, ocorre um distanciamento\ldots{} e se continuar com esta perspetiva,
caracterizada por um maior espaço, pára a energia que faz mover o pensamento. A
sua mente aquieta e o leitor pode discernir o que está subjacente a esse
pensamento, em termos emocionais, quer o leitor esteja a ansiar algo, a
funcionar baseado na força de vontade, na ansiedade\ldots{} seja o que for.
Assim que tiver colocado isso a descoberto, pode reflectir sobre em que medida
esse formato emocional é relevante ou útil. Isso ajuda os pensamentos e as
disposições a encontrar determinação: podem ficar mais firmes ou dissolver"-se,
de forma constante e com discernimento. Assim, o leitor recorre à energia
emocional para rever e soltar os padrões de pensamento, e ao corpo para rever e
soltar as emoções. Desta forma a meditação liberta a tensão. E, acima de tudo, o
leitor aprende a relacionar"-se com a sua mente com clareza e empatia.

\subsection{A respiração na meditação}

Na prática de ser consciente da respiração (\emph{ānāpānasati}) o leitor revê e
liberta programas para atingir liberdade, que podemos denominar `o
Não"-programado' (\emph{asaṅkhatā}). Mente, discurso e corpo estão todos
presentes neste processo meditativo. Ou seja, o leitor começa com a atitude de
ter `todo o tempo do mundo para estar apenas a respirar', de forma a colocar a
mente à-vontade e a estabilizá"-la. Então, à medida que se aquieta na postura
sentada, vai moderar a sua capacidade de pensamento ao dirigir a atenção da
mente para a inspiração e a expiração. Ser claro e estar atento à respiração
durante o período de uma inspiração e de uma expiração inteiras revela e coloca
a descoberto, definitivamente, padrões de pensamento compulsivos.

Então, como é que o pensamento constitui um suporte a essa prática? Bem, o
leitor pode utilizar um mantra, tal como `Buddho', pensando `Bud-', à medida que
inspira, e fazer com que o `som' dessa sílaba se estenda pela totalidade desse
processo físico. E o mesmo com `-dho' durante a expiração. Contudo, pode achar
que, ao fim de algum tempo, esta verbalização constitui um obstáculo.
Pessoalmente, recomendo estabelecer um pensamento inicial, perguntando: `Como é
que sei que estou a respirar?' E depois: `Como é que isso é?' E utilizar isto
apenas o suficiente para manter um questionamento centrado. Não envolve um
grande grau de raciocínio, mas existe ponderação. Estamos a ter algo em
consideração, a pegar"-lhe: `Onde é que está agora?' E a reparar que sabemos que
a respiração está a acontecer devido a uma sensação de aumento, aperto e alívio,
no peito ou no diafragma. Este é kamma verbal adequado, porque ilumina e
clarifica, assim como acalma a mente. Leva a que os padrões de pensamento párem,
não através da aniquilação, mas através da sintonização.

Existem vários tipos de padrões físicos que podem ser sintonizados quando
estamos a inspirar e a expirar. Primeiro, podemos reconhecer os aspectos
puramente físicos, corpóreos da respiração, como por exemplo o aumento repetido
do tórax ou do abdómen e a dilatação e a distensão da pele. Em segundo lugar,
podemos sentir o fluxo do ar através do nariz e a descer pela parte de trás da
garganta. Em terceiro lugar, existe o efeito de promover energia: quando
inspiramos, temos um efeito que ilumina e, quando expiramos, temos um efeito
calmante. Estes são os três estratos da experiência de respirar. Com o tempo
conseguimos distingui"-los a todos.

\enlargethispage{\baselineskip}

Coloco alguma enfâse no efeito energético porque podemos observá"-lo em qualquer
parte do sistema, em qualquer local que nos seja conveniente, e porque confere
brilho e calma à mente. Contudo, se concebemos o nosso corpo e a nossa
respiração em termos puramente físicos -- inspirar, encher os pulmões de ar, e
depois expirar novamente -- descuramos este aspecto da respiração. E se
procuramos estar atentos a um ponto físico onde aparentemente não temos energia,
torna"-se difícil ficarmos descontraídos e confortáveis. Mas se, pura e
simplesmente, colocarmos de lado o conceito do corpo e nos interrogarmos: `Como
é que eu estou neste momento a experienciar o meu corpo?', podemos sentir o
corpo de forma mais dinâmica. Verificamos que existem todos os tipos de
tremores, ímpetos, rubores, formigueiros e palpitações. De igual forma, o corpo
é bastante inteligente e parece saber o que fazer: quando está tenso, mas a
mente saí da equação, ele solta"-se; quando precisa de inspirar, inspira, e nunca
expira quando precisa de inspirar, nunca confunde estas duas acções. Tem um
sistema inteligente que cuida de si próprio. Todo este processo constitui o
padrão físico, \emph{kaya"-saṅkhāra}, sendo que a inspiração e a expiração
encontram"-se precisamente no seu centro, enquanto experiência energética.

\enlargethispage{\baselineskip}

Com efeito, o próprio Buddha não diz para centrarmos a atenção num ponto do
corpo ou mesmo na inspiração ou na expiração, mas sim para estarmos cientes do
processo de respiração, como um todo. O Buddha diz simplesmente: `Saiba que está
a inspirar e saiba que está a expirar.'\pagenote{Citando do Ānāpānasati sutta:

  `Fazendo uma inspiração longa, ele compreende `Eu faço uma inspiração longa';
  Fazendo uma inspiração curta, ele compreende `Eu faço uma inspiração curta'\,'
  \href{https://suttacentral.net/mn118/en/bodhi}{MN 118.18}

  Aqui, a palavra `compreende' (pajanati) é muito próxima da palavra `sampajano'
  e implica uma total consciência da experiência, ao invés de pensar sobre a
  experiência.}
Não refere seja o que for sobre onde devemos focar a nossa atenção -- apenas nos
encoraja a estarmos conscientes deste ritmo da `inspiração/expiração'. Para mim
isto é significativo pois o ritmo tem um efeito no coração. Qualquer músico,
qualquer pai ou mãe a embalar uma criança sabem disso. Se a concentração leva à
tensão, tente apenas receber o ritmo -- digamos, sentir o pequeno inchaço no
peito ou até o pequeno aperto do cinto na cintura e depois o soltar; por vezes
estas sensações repetem"-se e por isso reparamos nelas com facilidade. Esteja
simplesmente consciente do corpo como um padrão de sensações repetidas que
ocorrem com a respiração. Quando nos apercebemos desta qualidade repetitiva,
vamos tomar consciência da energia, uma vez que essa é a fonte dessa vitalidade
que flui: \emph{kaya"-saṅkhāra}.

A prática é tornar as coisas simples. Dê"-se a si próprio todo o tempo que
necessita para tornar tudo simples -- isto por si só, inverte as tendências de
uma vida. E quando perder a concentração, não faça disso um problema, pois dessa
forma pode transformar outro hábito da \emph{citta}. A prática fica mais
acessível se não fizer mais nada para além de reparar quando se dispersa e,
nesse momento, apenas perguntar: `O que se passa neste momento com a
respiração?'\ldots{} e aperceber"-se da sensação, seja ela qual for, que surge ligada
à respiração. Estará, provavelmente, a alterar programas profundos, pelo simples
facto de não estar a forçar. Seguidamente, o resto da prática prossegue à medida
que o praticante fica mais leve e simples.

À medida que a sua mente se estabelece, pode refinar o processo ao sintonizar"-se
com a duração total da respiração. Isto põe"-nos em contato com o final, o
libertar e a quietude no final da respiração, bem como com a plenitude e a
quietude totais quando a inalação está completa. Este completar constante, este
atingir da quietude, constitui um aspecto da energia do corpo que, com
frequência, não seguimos na nossa forma de vida habitual. Mas ao dar"-se a si
próprio `todo o tempo do mundo' para se sintonizar com a respiração, permite"-se
ficar presente com esse movimento em direcção à quietude. E pode sintonizar"-se
com isso através da sua receptividade às sensações tácteis: o que, por si,
constitui uma mudança de atenção importante, relativamente às bases racionais ou
visuais que, normalmente, dominam as nossas vidas. O sentido do tacto é
altamente sensível e reactivo, de uma forma não"-verbal. É igualmente íntimo:
quando eu toco em algo, isso também me toca\ldots{} por isso desenvolve"-se confiança.
Inspirar e expirar constitui um processo agradável e seguro, que encoraja um
aprofundamento dessa confiança. E quando confiamos, a energia descontrai e o
coração fica mais luminoso. Deste modo, estarmos em sintonia com a respiração
resulta em sensibilidade e descontração: kamma luminoso.

Assim, existem efeitos somáticos e emotivos que acompanham esta prática.
Sentimo"-nos profundamente descontraídos, suaves, realizados, frescos. Esta é a
experiência de êxtase (\emph{pīti}), um estado de frescura e entusiasmo, bem
como de bem"-estar (\emph{sukhā}). Estes estados trazem consigo a sensação de
estarmos em sintonia com algo. Não se trata apenas de estarmos a fazer o bem,
mas sim de estarem a acontecer coisas boas e, à medida que tomamos consciência
disso, a \emph{citta} e o corpo acalmam, a respiração suaviza e estes efeitos
combinados estendem"-se a todo o sistema. A mente pensante, o coração e o corpo
encontram"-se e começam a unificar"-se, sendo que esta unificação tem
simultaneamente um carácter de luminosidade e de quietude. Trata"-se da
`concentração correcta' (\emph{samādhi}).

\subsection{Os factores de samādhi}

\emph{Samādhi} vai muito para além da concentração que podemos estabelecer para
resolver problemas, ou quando estamos absortos num entretenimento emocionante.
Estes são exemplos de algo que funciona através da nossa absorção e não do
treino da atenção: não desenvolvemos grandes capacidades quando assistimos à
final do campeonato do mundo! Contudo, uma vez que \emph{samādhi} ao mesmo tempo
depende e afecta a forma como aplicamos a nossa mente, requer uma inspecção do
modo como a mente funciona. É necessário desenvolver a intenção direcionada de
forma a manter a mente interessada, envolvida e contida. Contudo, se a intenção
tem um carácter demasiado forçado e impaciente, então deixa de ter receptividade
e capacidade para apreciar e disfrutar. A concentração depende de modificarmos a
intenção e a atenção: temos de aprender a forma de encorajar o interesse, como
disfrutar, como largar e como apreciar. Só a aprendizagem destas capacidades já
é razão suficiente para praticar.

É útil levar em consideração que o \emph{Samādhi} depende da interacção de cinco
factores: trazer à mente (\emph{vitakkā}), avaliação (\emph{vicāra}), êxtase,
à-vontade e unidireccionalidade (\emph{ekaggatā}). Em primeiro lugar surge o
trazer à mente e a avaliação: estabelecer uma base para a respiração e estudar
as suas sensações. Por outras palavras, orientamos a atenção com estímulos
apropriados, tais como: `Como é que sei que estou a respirar?', `Como é que está
a ser isto agora?' Seguidamente podemos explorar efeitos mais subtis: a duração
da respiração, o efeito imediato de qualquer impacto e as ressonâncias em termos
de sentimentos. Podemos utilizar a capacidade de pensamento para dirigir a
atenção para a forma como os padrões se interrelacionam: quando nos surge uma
enxurrada de pensamentos, em vez de pensarmos sobre eles, perguntarmo"-nos: `Como
é que estou a sentir isto no corpo?' ou `Como é que estou a sentir isto no
coração?' O pensamento discursivo habitual, em geral, traz consigo um certo
aperto no campo físico de energia: podemos senti"-lo como mais energizado ou com
arestas mais vivas. Pode dar"-se um aumento de energia nos ombros, nas mãos ou na
cara -- as partes `criativas' do corpo. Pode dar"-se uma ligeira contracção no
diafragma -- a parte de `protecção' do corpo. Mas seguidamente: `Onde está agora
a respiração?' Também a respiração vai estar afectada -- frequentemente a sua
amplitude é reduzida. Assim: `E se eu esperar até à próxima expiração e deixar
que flua através de todo o corpo?' Depois, deixe acontecer.

De forma semelhante com o coração: pensar faz com que surja um turbilhão. Mas em
vez de reagir ao tópico do pensamento, sinta a brusquidão, o tom de urgência, o
seu borbulhar ou remoer: `Qual o sentido emotivo disto?' Por vezes existe
ansiedade ou o impulso para fazer qualquer coisa; ou pode existir a sensação de
dor subjacente aos pensamentos; ou o turbilhão de alegria que acompanha uma
grande ideia. (O que me acontece com frequência quando tento meditar!) Então:
`Qual o efeito disto em mim?' Mas em vez de nos analisarmos e admoestarmos por
termos novamente vagueado e `Quantas vezes\ldots{}!' etc., fazemos uma pausa\ldots{} Então
podemos trazer ao coração a sensação `Que eu possa estar bem\ldots{}'. Seguidamente:
`Para já, porque não fluir com a respiração?' Se o pensamento é simples e
cuidador, ilumina a atenção e vai ao encontro do bem"-estar subtil associado à
calma do corpo e da mente. Isto constitui êxtase e à-vontade, o segundo par de
factores de \emph{samādhi}. Estes factores de bem"-estar, por sua vez, fazem com
que a mente saia da ânsia e da crítica, deixe de ser abafada pelo torpor, de se
afligir ou de estar presa na dúvida. Estes estados afastam as dificuldades ao
tornar mais suaves as energias de tensão, iniquidade e torpor, que suportam a má
vontade, a dúvida, a agitação e a ânsia por algo. É este o seu principal
objectivo, o seu efeito medicinal.

\enlargethispage{\baselineskip}

A unidireccionalidade constitui o último dos factores envolvidos. Resulta do
facto da mente apreciar e ser dirigida para a energia física e mental quando o
sistema não está a ser atirado de um lado para o outro pelas dificuldades e
pelas distracções. Apesar de o termo ser unidireccionalidade, surge através do
reconhecimento do à-vontade global da totalidade do campo de energia física. Com
frequência, a atenção pode pousar na zona de um ponto no corpo, digamos a parte
posterior das vias nasais, ou o diafragma, ou onde quer que seja confortável à
medida que os desequilíbrios, os apertos ou as dormências na energia do corpo
vão sendo libertos. Mas quando a centragem é feita com êxtase dá"-se um efeito
irradiante e a energia da respiração penetra em todo o corpo. As arestas vivas e
a rigidez do corpo são sentidas mais como um campo energético. Então o à-vontade
estabiliza aí a atenção, de forma a neutralizar alguma euforia ou apreensão.
Como resultado sentimos que ficamos suspensos numa energia que estabelece uma
base sólida: isto constitui a unidireccionalidade. Quando isto se desenvolve
como um efeito duradouro torna"-se a concentração conhecida como `absorção'
(\emph{jhāna}).

\subsection{Parar o kamma através do discernimento}

\emph{Samādhi} tem a natureza do kamma, das causas e efeitos gerados a partir de
programas relacionados com a intenção e a atenção situadas no presente. Também
tem como base as disposições -- programas estabelecidos no passado. Sendo que,
naturalmente, cria programas para o futuro: desenvolvemos a tendência no sentido
de modos de vida mais simples e pacíficos. Com tudo isto, é benéfico termos em
mente que o propósito permanente da meditação é a libertação relativamente a
programas antigos, enquanto se desenvolvem programas novos. \emph{Samādhi}
proporciona"-nos uma libertação temporária de alguns temas cármicos (tais como os
desejos dos sentidos, a preocupação ou a má vontade) no presente,
proporcionando"-nos uma mente firme e bem apoiada, que se sente luminosa. Mas
mesmo o próprio \emph{samādhi} é formulado.

\enlargethispage{\baselineskip}

De igual forma, o seu desenvolvimento leva tempo e, entretanto, a própria noção
de `alcançar \emph{Samādhi'} pode despoletar pensamentos stressantes, tais como:
`Não consigo', `Sou um desastre', e por aí fora. De igual modo para quem
desenvolve, e para quem não desenvolve, um forte \emph{samādhi}, a aprendizagem
reside em lidar com a programação e revê"-la: `Quanto `\emph{eu'} está envolvido
nisto; quanto apego ainda existe?' O processo de discernimento consiste nisto e
é sempre relevante.

Apegarmo"-nos, ganharmos, obtermos, perdermos: a energia de formulação de
\emph{saṅkhāra} -- estável ou agitado, dirigido ou à deriva -- é algo que pode
ser testemunhado. E podemos moderá"-la ao termos o corpo como referência em
relação à fala e à mente. Por exemplo, quando uma troca de palavras começa a
ficar demasiado aquecida, ajuda muito ter a noção de como o corpo e a mente se
encontram interrelacionados. Podemos sintonizar"-nos com o que está a acontecer
no corpo: as palmas das mãos, as têmporas e os olhos são fontes de informação
acessível relativa à energia. Será que esta energia precisa de ser descontraída?
Por vezes descubro que apenas tomar consciência e ajustar a velocidade do meu
andar altera as atitudes e as disposições; ou suavizar e tornar o olhar mais
difuso. Ou quando nos sentimos entorpecidos ou deprimidos: será que o corpo está
realmente presente? Peito\ldots{} garganta\ldots{}? Talvez dar gentilmente alguma atenção
ajude a energia a iluminar"-se e altere a disposição mental.

Podemos reparar no aparecimento de contentamento ou de desânimo, do engodo do
sucesso e da ansiedade em querer mais. Mas ao contemplar as disposições e os
instintos que surgem como realmente são, podemos concentrar"-nos nos seus padrões
e programas enquanto tal, em vez de `isto sou eu', `isto é meu', `tomo a minha
posição sobre isto', ou mesmo, `sou diferente disto'. É este o objectivo do
discernimento. Trata"-se de testemunhar os programas: como estão dependentes da
forma como nos vemos a nós próprios; como surgem com uma contracção, uma
sofreguidão; e como levam à criação de ideias e de noções. Com a meditação,
contemplamos toda a ladainha do sucesso e do fracasso, aquilo que sou e aquilo
que serei: tudo isto constitui mais formulação. Constitui tudo mais kamma, mais
`forma como nos vemos a nós próprios', mais coisas com as quais nos ocuparmos.
Mas se virmos a futilidade de tudo isto, libertamo"-nos do programa. E esta é a
única forma de nos vermos livres do kamma.

Quando este aspecto se torna claro, não há muito mais a fazer do que
mantermo"-nos atentos ao que surge e trespassa a nossa consciência. Porque quando
nos referimos às energias emocionais, físicas e conceptuais como programas, isso
já não apoia a noção de `eu sou'. Não sendo apoiadas por esta noção, aquietam.
Então podemos lidar com a vida, sem sermos atirados por ela de um lado para o
outro. Não necessitamos de estar continuamente a mostrar o nosso valor, a
defendermo"-nos e a criarmo"-nos a cada momento. O kamma pode cessar.

Mas é como coçar uma comichão ou fumar um cigarro: mesmo que tenhamos a noção
que seria bom parar, o nosso sistema não o irá fazer antes de sentirmos
realmente os benefícios de pararmos e nos sentirmos suficientemente firmes para
o realizar. Estas características de tranquilidade e de firmeza são aquilo que
\emph{samatha} proporciona na meditação: abre e cura os nossos sistemas e
permite que a clareza se estenda a áreas e a aspectos da programação que, com
frequência, passam despercebidos na vida do quotidiano. Então, com essa calma e
tranquilidade, o discernimento permite que o kamma cesse.

\clearpage

\section[Meditação: dar corpo à mente]{Meditação}

{\centering
\subSectionFont\selectfont
\textit{Dar corpo à mente}
\par}

\bigskip

Sente"-se numa postura verticalizada e dirija a consciência para a experiência no
momento presente. Interrogue"-se: `Como é que eu sei que tenho um corpo?' Por
outras palavras, procure a experiência directa da noção do corpo -- as pressões,
energias, pulsações e vitalidade que traduzem a consciência do corpo.
Seguidamente, a partir dessa situação de sensibilidade directa, procure mais
detalhes.

Empurre um pouco para baixo o seu cóccix e o fundo pélvico. Repare como isso
ajuda a colocar a coluna num equilíbrio, no qual o sacro está direito e a região
lombar forma um arco com alguma tensão. Evite bloquear ou forçar. Faça um
ligeiro movimento no sentido descendente para formar o arco, em vez de forçar
uma curva exagerada com um ímpeto no sentido ascendente com os músculos da zona
lombar. Isto fornece à postura a sua fundação crucial: permite ao corpo ser
sustido por uma mola que transfere o seu peso para aquilo sobre o qual se
encontra sentado.

\enlargethispage{\baselineskip}

Desloque a sua consciência de forma gradual e delicada pela sua coluna vertebral
acima, desde a ponta do cóccix, através do sacro e das vértebras lombares e
dorsais. Estique o corpo ligeiramente no sentido ascendente, a partir das ancas.
Verifique o centro das costas, entre as extremidades inferiores das omoplatas:
dê vida a esta zona ao trazê"-la para dentro, no sentido do coração. Indo para
cima, certifique"-se que os ombros estão para baixo e descontraídos e faça um
varrimento com uma consciência descontraída desde a base do crânio até às faces
laterais do pescoço e ao longo do topo dos ombros. Leve a consciência para as
vértebras do pescoço -- conscientize"-se de que existe uma sensação de espaço
entre a face posterior do crânio e a zona superior do pescoço. Isto pode ajudar
se colocar o seu queixo para dentro e o inclinar ligeiramente para baixo.
Verifique o equilíbrio geral -- se a cabeça se encontra equilibrada na coluna,
alinhada directamente acima da pélvis. Certifique"-se que a coluna se encontra
descontraída. Descontraia igualmente os ombros, os maxilares e deixe que o peito
se abra. Permaneça algum tempo a sentir a estrutura óssea, dando a sugestão às
articulações situadas entre os braços e os ombros, por exemplo, para que estas
fiquem soltas e com uma sensação de abertura. Deixe os braços alongarem"-se.
Descontraia"-se no equilíbrio.

\enlargethispage{\baselineskip}

Esteja atento às sensações em termos físicos: por exemplo, a forma como o peso
do corpo se encontra distribuído; ou o grau de vitalidade e de calor interno que
se encontra presente. Sinta os movimentos subtis do corpo, mesmo quando este
está quieto -- o pulsar, o palpitar e as sensações rítmicas associadas à
inspiração e à expiração. Sinta"-se confortável: avalie as sensações físicas em
termos de tranquilidade. Uma certa pressão numa determinada zona pode ser
sentida como sólida e proporcionar uma boa base, noutra zona pode sentir"-se
aperto ou rigidez. As energias e as sensações internas que se movem pelo corpo
podem ser sentidas como agitadas ou vibrantes. Liberte"-se das interpretações
mentais relativas às suas causas, bem como de quaisquer reacções relativas ao
facto de serem correctas ou incorrectas. Ao invés, espalhe a consciência
homogeneamente por todo o corpo, com uma intenção de harmonia e estabilidade.
Deixe que essa atitude seja sentida como uma energia que se espalha por todo o
corpo. Isto vai permitir que qualquer contracção se descontraia e vai trazer
luminosidade às áreas mais folgadas ou entorpecidas.

\enlargethispage{\baselineskip}

À medida que tudo se harmoniza, as sensações da respiração vão ficar mais
nítidas, profundas e estáveis. Poderá notar não apenas que a respiração flui até
ao abdómen, como também que esta produz uma sensação subtil de rubor ou de
formigueiro na cara, nas palmas das mãos e no peito. Permaneça com estas
sensações, explorando"-as. É provável que a mente vagueie, mas acima de tudo
assegure"-se que mantem a intenção no sentido da harmonia e da estabilidade.
Assim, quando se apercebe que a mente vagueou, nesse momento de
consciencialização -- faça uma pausa. Não reaja. Enquanto a mente paira durante
esse momento, introduza a questão: `Como é que eu sei que estou a respirar
agora?' Ou simplesmente: `A respirar?' Entre em sintonia com a sensação que
surge (seja qual for) que lhe indica que se encontra a respirar e siga a próxima
expiração, deixando a mente repousar nessa expiração. Veja se consegue
permanecer com a expiração até à última sensação, entrando depois na pausa que
antecede a inspiração. Depois, siga a inspiração de forma semelhante, até à
última sensação. Desta forma, deixe que o ritmo da respiração dirija a mente --
em vez de impor a ideia do `estar consciente' ao processo natural de respiração.

Explore a forma como experiencia a respiração em diferentes partes do corpo,
começando na barriga. `Como é que a barriga conhece a respiração?' Pode
experienciar isto como uma sensação de dilatação fluida. Permaneça aí durante
alguns minutos, deixando que a mente absorva. Depois, `Como é que o plexo solar
conhece a respiração?' Aqui pode ter uma sensação mais concreta, como um abrir e
fechar. Depois o peito, onde predominam as sensações `aéreas' de dilatação.
Verifique a garganta e a zona entre as sobrancelhas. Repare como a respiração
não constitui um modo único de sensações ou de energias mas que, em termos de
energia, a distinção entre inspiração e expiração é sempre reconhecível.

\enlargethispage{\baselineskip}

Por fim a sua mente vai querer aquietar e centrar"-se num ponto do corpo --
permita"-lhe escolher o que for mais confortável. Pode ser no peito ou nas
passagens aéreas do nariz, por exemplo. Depois continue a seguir e a estudar a
respiração, como anteriormente. À medida que a mente entra em fusão com a
respiração"-energia, espalhe a sua consciência sobre todas as sensações do corpo,
como uma inundação ou instilação. As diferentes sensações da respiração podem
difundir"-se e dissolver"-se nessa energia. Permita alguma confiança, deixando que
a atenção cognitiva se descontraia e apoiando"-se na fruição da energia subtil de
forma a manter"-se consciente. Esteja presente, mas não envolvido em tudo o que
surge.

Quando quiser parar, leve a sua atenção de volta para as texturas da carne e
para a firmeza da estrutura óssea. À medida que sente esta presença sustentada,
permita que os seus olhos se abram sem olhar para nada em particular. Ao invés,
deixe que a luz e as formas ganhem configuração por elas próprias.
