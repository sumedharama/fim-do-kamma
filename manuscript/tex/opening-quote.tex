\clearpage

\begin{quote}

  ``O kamma deve ser conhecido. A causa através da qual o kamma entra em acção
  deve ser conhecida. A diversidade no kamma deve ser conhecida. O resultado do
  kamma deve ser conhecido. A cessação do kamma deve ser conhecida. O caminho da
  prática que leva à cessação do kamma deve ser conhecido. Assim foi dito. Por
  que razão foi dito?

  A intenção, digo"-vos, é kamma. Ao termos intenção, produzimos kamma através do
  corpo, da fala e da mente.

  E qual a causa através da qual o kamma entra em acção?

  O contacto, bhikkhus.

  E o que é a diversidade no kamma? Existe kamma a ser experienciado no inferno,
  kamma a ser experienciado no reino dos animais vulgares, kamma a ser
  experienciado no reino das sombras famintas, kamma a ser experienciado no
  mundo humano, kamma a ser experienciado nos mundos celestiais.

  E qual é o resultado do kamma? O resultado do kamma, digo"-vos, tem três tipos:
  aquele que surge aqui e agora, aquele que surge mais tarde (nesta vida) e
  aquele que surge depois disso\ldots{}

  E o que é a cessação do kamma? A partir da cessação do contacto dá"-se a
  cessação do kamma e apenas este nobre caminho óctuplo -- visão correcta,
  propósito correcto, discurso correcto, acção correcta, meio de subsistência
  correcto, esforço correcto, consciência correcta, concentração correcta --
  constitui o caminho da prática que conduz à cessação do kamma.

  Assim, quando um discípulo dos seres nobres compreende desta forma o kamma, a
  causa através da qual o kamma entra em acção, a diversidade no kamma, o
  resultado do kamma, a cessação do kamma e o caminho da prática que conduz à
  cessação do kamma, então compreende esta vida sagrada penetrante como a
  cessação do kamma.

  ``O kamma deve ser conhecido. A causa através da qual o kamma entra em acção
  \ldots{} A diversidade no kamma \ldots{} O resultado do kamma \ldots{} A
  cessação do kamma \ldots{} O caminho da prática que leva à cessação do kamma
  \ldots{}'' Assim foi dito e esta é a razão pela qual foi dito.''

  \href{https://suttacentral.net/an6.63/en/thanissaro}{AN 6.63}

\end{quote}
