\chapterNote{Apoio para a atenção}

\chapter{Kamma Luminoso}

\tocChapterNote{Apoio para a atenção}

\begin{quote}
  ``Aquele que faz o bem, alegra"-se no presente e no futuro, em ambos os mundos. Relembrando as suas acções puras, ele alegra"-se e exulta''

  \quoteRef{\href{https://suttacentral.net/dhp1-20/en/buddharakkhita}{Dhammapada 16}}
\end{quote}

Nas duas últimas semanas, uma imagem do Buddha tem sido criada neste mosteiro
pelo Ajahn Nonti, um escultor tailandês que veio cá praticar este acto de
generosidade. Tem sido um bonito acontecimento, pois a imagem do Buddha tem sido
feita de forma amigável e agradável. Muitas pessoas têm tido a oportunidade de
participar e ajudar. Ontem estavam nove pessoas a trabalhar na imagem, a
alisá"-la. Não é uma imagem assim tão grande, contudo estavam nove pessoas a
trabalhar nela, sem chocarem umas nas outras e a disfrutarem o trabalho em
conjunto.

\subsection{Kamma brilhante e kamma escuro surgem no coração}

Nove pessoas a trabalhar em conjunto de forma amigável e agradável, é um bom
acontecimento. Para além disso, tudo isto tem um carácter voluntário e não
aconteceu devido a qualquer combinação prévia: as pessoas interessaram"-se pelo
projecto e reuniram"-se para colaborar nele. Neste caso, isto deve"-se ao que o
Buddha representa e ao facto de as pessoas gostarem de aderir a causas
beneméritas. É essa a magia do bom kamma: surge quando fazemos algo com
significado a longo prazo e, também, por agirmos de uma forma que nos traz
felicidade, em vez de ser algo intenso ou compulsivo. O kamma (acção intencional
ou volitiva) tem sempre um resultado ou um resíduo e, neste caso, é óbvio que o
bom kamma está a ter bons resultados. Há um resultado imediato pois as pessoas
sentem"-se contentes ao trabalharem em conjunto. E existe um resultado a longo
prazo: estão a fazer algo que irá beneficiar outras pessoas.

Dentro de alguns dias contamos instalar a imagem do Buddha na sala de meditação.
É uma imagem que me faz sentir bem quando a contemplo. Tem uma qualidade suave e
convidativa, que cria um sentimento de boas"-vindas e de descontração - é um
lembrete muito bom para a meditação. Por vezes as pessoas podem ficar bastante
tensas relativamente à `iluminação' e isso faz surgir preocupações, necessidades
e exigências. Mas, com frequência, o que necessitamos é de nos sentirmos
bem"-vindos e abençoados. Isto constitui uma reviravolta significativa da nossa
forma habitual de pensar. Mas quando estamos sentados num sítio no qual sentimos
que confiam em nós e onde estamos rodeados por benevolência, podemos nos abrir à
experiência. E, à medida que abrimos os nossos corações, podemos sentir a
clareza da presença e consolidarmo"-nos à sua volta. Esta firmeza que surge da
suavidade é aquilo que a imagem do Buddha representa. Ela lembra"-nos que existiu
um Buddha histórico, cujo Despertar ainda brilha através dos tempos - mas quando
isto se apresenta como uma impressão do coração e não apenas como um excerto de
história, traz consigo muito mais ressonância. Então a imagem serve como uma
reflexão directa sobre o sentimento associado ao bom kamma.

Nas escrituras, o bom kamma é geralmente denominado `kamma claro', por oposição
ao `kamma escuro'. `Claro' quer dizer que experienciamos clareza e elevação.
Mais do que uma ideia, a claridade é uma tonalidade e não um julgamento (bom ou
mau, certo ou errado). Tem beleza. A palavra `claridade' tem o sentido de algo a
abrir, de suavidade e de alegria -- tem estas tonalidades. Enquanto que `escuro'
implica estar fechado, contraído e sem esperança. De forma que isto é algo a ser
investigado internamente: a tonalidade das acções que empreendemos e o contexto
que geramos à nossa volta são luminosos? Mesmo quando admitimos alguma verdade
dolorosa acerca das nossas acções, não existe uma certa claridade, uma certa
dignidade, quando o fazemos de bom grado? Procurem a claridade em situações nas
quais nos apresentamos assim, e não em termos de facilidade superficial ou de
serem bons por obrigação: a atitude subjacente àquilo que dizemos e pensamos --
é clara ou é escura? Mais do que o encanto ou a obediência, essa tonalidade é a
atmosfera, o sítio onde os nossos corações residem.

\subsection{Mente"-órgão e mente"-base}

A energia do kamma circula através de três canais. O primeiro é o corpo: agimos
fisicamente. O segundo é a faculdade da fala (que inclui o `diálogo interno' do
pensamento). O terceiro é a mente - o sentimento de ser afectado e de responder.
Em inglês (e em português), o termo `mente' abrange tanto a actividade
conceptual como o sentimento afectivo que dá à mente o estado ou disposição
global. Mas na língua Pali existem duas palavras:

\enlargethispage{\baselineskip}

\begin{itemize}

  \item \emph{mano:} refere"-se à mente enquanto órgão que se concentra na
        informação recebida através de qualquer um dos sentidos. Esta função
        denomina"-se `atenção' (\emph{manasikārā}). A mente"-órgão pode igualmente
        definir e articular -- faz"-nos recordar e produz conceitos. Em termos de
        tonalidade é bastante neutra, não é alegre nem triste. É a racionalidade
        que define: `Aquilo é aquela tal coisa, desta forma, assim e assado'.

  \item \emph{citta:} refere"-se à mente enquanto `coração', a base que recebe
        as impressões que a atenção lhe proporcionou. É afectada em termos de
        prazer e de dor, o que se manifesta em estados de espírito de variáveis
        graus de alegria e de tristeza. De igual forma, \emph{citta} forma
        sinais, percepções ou `significados sentidos' (\emph{saññā}) relativos
        às impressões que recebeu e que a afectaram. Então, podemos avaliar nova
        informação com base nos sinais que já estão definidos: uma bola cor de
        laranja de uma determinada dimensão e textura é provavelmente uma
        laranja. Certamente não será uma pessoa ou um tigre. Este significado
        possui nuances que assim determinam outra função mental: a intenção -- a
        mente move"-se no sentido do objecto voluntariamente ou com o interesse
        de o comer. A intenção ou volição ocorre como uma resposta a sermos
        afectados e é assim que surge o kamma mental.

\end{itemize}

\enlargethispage{\baselineskip}

A mente"-base ou a mente"-órgão funcionam da seguinte forma: quando a \emph{citta}
é afectada, o seu órgão, \emph{mano}, pode então produzir um conceito mental
adequado, de modo que, tendo reconhecido um objecto, podemos então dizer `Isto é
um cão; isto é uma campainha'. A mente"-órgão pode igualmente inspecionar a
mente"-base e definir as suas disposições. Tudo isto constitui a acção da
\emph{mano}. O problema é que as pessoas podem pensar em praticamente tudo, com
base no que vêem, ouvem e em ideias, sem reflectirem necessariamente sobre a
forma como o coração foi afectado. Existem bastantes discussões sobre a verdade,
a paz, o amor, a liberdade e outros grandes ideais, porque as paixões ou os
receios se misturam com estas noções. Contudo, não se sente seja o que for ao
nível da faculdade \emph{mano}. E, assim, este não reconhecimento do preconceito
subjectivo é denominado como `verdade objectiva'. (!) Mas para sabermos
plenamente, não apenas pensarmos ou termos alguém que nos diga, e sim sentirmos
realmente a qualidade de bondade, amor e por aí fora, temos de entrar neste
coração e purificá"-lo. Desta forma, o \emph{kamma} mais importante para
aprofundarmos a nossa verdade, a nossa paz e a nossa liberdade, começa ao darmos
uma reviravolta à nossa mente -- fazendo com que a \emph{mano} examine a
\emph{citta}.

\subsection{Significados sentidos}

Quando olhamos para aquilo que provoca os nossos impulsos e acções, percebemos
que estes surgem a partir de sentimentos e tendências do coração. Estes
sentimentos e tendências mentais estão ligados às percepções ou `significados
sentidos', tais como `sentir"-se só' ou `sentir"-se bem"-vindo', e com base nisso
`sentimos que nos apetece dar uma volta' ou `sentimos vontade de visitar
determinada pessoa': há uma propensão. Alguém nos diz algo e podemos pensar,
`Bem, isto soou"-me mesmo agressivo.' Trata"-se de um `significado sentido', uma
percepção mental. Existe uma interpretação emotiva das palavras que alguém
proferiu e é provável que isso vá constituir a base para a intenção: as nossas
tendências, acções e reacções. O que `sentimos' nisso, a percepção, é uma
impressão do coração (denominada `contacto"-designação --
\emph{adhivacana"-phassa}). Apesar de se basear num contacto externo, este
contacto designação (ao contrário do contacto com algo externo) é o que vai
formular, de forma significativa, as impressões que nos animam. Uma parte da
intenção tem como base os reflexos do corpo mas, na sua maior parte, trata"-se da
impressão do coração daquilo que é visto, ouvido, cheirado, tocado ou pensado,
que nos põe em movimento -- para o bem e para o mal.

O significado sentido torna"-se mais poderoso se nós `sentimos' que alguém nos
fez algo propositadamente, ao invés de o ter feito acidentalmente ou por acaso.
Imaginemos um caso em que uma pessoa foi mal"-educada para connosco catorze vezes
este ano. Se o tivesse feito uma vez, pensaríamos que tinha sido um engano, mas
catorze vezes? A acção actual foi sentida de forma mais intensa devido às acções
que ocorreram anteriormente. O `significado sentido' desenvolve algum peso de
forma dependente de uma inferência emotiva. Podemos inferir intenção: `Ele fez
aquilo propositadamente'. Ou fatalismo: `Tenho sempre de aturar pessoas
estouvadas'. Podemos reagir em conformidade a nível psicológico, verbal ou
físico. Esta é a forma como impressões, atitudes e tomadas de posição prévias
moldam a impressão do coração. Desta maneira, as impressões e as atitudes que
transportamos do passado tornam"-se uma base para a intenção, uma base para mais
\emph{kamma}.

Do mesmo modo, quanto mais nos centramos nas nossas impressões e lhes damos
atenção, ou vemos o mundo e os outros através de impressões antigas, mais essas
impressões podem ficar potentes e firmemente estabelecidas. Mas não podemos
confiar nas impressões do coração. Isto acontece porque temos a tendência para
reparar naquilo que nos habituamos a reparar: a conduta graciosa dela, os
maneirismos irritantes dele, etc. E, à medida que revisitamos o mundo desta
forma, acrescentamos mais interpretações. Então, eu `sinto' que `ele é sempre
assim' ou eu `vejo"-te' de uma determinada forma, ou eu apenas reparo nos meus
maus hábitos. Deste modo o meu foco, a minha atenção, fica preparada para
procurar impressões antigas. E eu trago"-as à mente, matuto sobre elas, sou
afectado por elas e ajo em conformidade. De forma que a mente que perscruta
consegue estar sempre a selecionar impressões de acordo com a tendência do
coração e, desta maneira, a contribuir para essa tendência, intensificando"-a. A
atenção está, deste modo, também ligada à intenção e à geração do kamma. Isto
tudo significa que não podemos confiar apenas no coração ou apenas na atenção.
Temos de cultivar uma atenção cuidadosa, atentos ao coração, de forma a chegar
ao fim das tendências preconcebidas.

\subsection{Compreensão, atenção cuidadosa e consciência plena}

Tendo tudo isto em consideração, como é que podemos examinar e responder de
forma mais adequada? Como é que podemos reconhecer que sentimos que nos estão a
prejudicar ou a maltratar e não simplesmente reagirmos ou suprimirmos? Talvez
precisemos de dizer umas quantas coisas a algumas pessoas\ldots{} ou talvez seja
apenas uma questão de corrigirmos as nossas próprias percepções erróneas\ldots{}
Em qualquer dos casos, a melhor forma de começar é por examinar o coração. Este
processo inicia"-se com a `atenção compreensiva' (\emph{yoniso manasikārā}), que
se traduz na atenção reforçada pela intenção em considerar a experiência em
termos da forma como esta nos afecta. É uma abordagem feita de coração, através
da qual, ao invés de apenas seguirmos o tópico de um pensamento, o ouvimos
profundamente. Passamos a pente fino a torrente de interpretações ou de
divagações à volta dos tópicos com um sentido inquisitório que interroga: `Qual
o significado subjacente a este pensamento ou a esta atitude? Qual a suposição e
até que ponto é realista? Em que estado me encontro com a minha mente desta
forma?' Trata"-se de um inquérito compassivo e não crítico. E este inquérito
pede"-nos para termos um sentimento preciso sobre as psicologias que dirigem a
nossa vida. Seguidamente: `Gera stress ou não?'

Este processo revela os impulsos e as impressões subjacentes do coração -- se se
trata de sentimentos de ameaça ou alienação, ou de encorajamento e confiança.
Este material subjacente é o motor que fornece energia para a forma como
pensamos e para aquilo sobre o qual pensamos. É importante sabermos o que nos
faz agir em qualquer situação, de modo que controlar este processo não constitui
uma supressão -- trata"-se mais de permitirmo"-nos examinar o nosso território
interior. Isto ajuda"-nos a ver para além das fronteiras da nossa percepção de
nós próprios. Mas colocamos a análise e a acção sob pausa: não tentamos corrigir
as coisas; não vamos cair compulsivamente numa opinião sobre nós próprios
baseada neste inquérito. E a beleza simples deste processo é que, quando
suspendemos as reacções sobre o que devíamos e não devíamos estar a sentir,
existe clareza e amplitude. Com isto voltamos a conectar"-nos com a nossa
sensibilidade ética inata -- o bom kamma que constitui um suporte à clareza e à
compaixão.

Estas, felizmente, são as qualidades básicas que todos possuímos enquanto seres
humanos. Mas, porque a nossa forma de abordagem é muitas vezes superficial, ou
sujeita aos nossos objetivos, estas qualidades nem sempre nos são acessíveis.
Assim, elas surgem em virtude de uma consideração abnegada, uma consideração sem
pressões, opiniões ou juízos de valor. Esta consideração é a atenção
compreensiva. Apenas tenta ver aquilo que é stressante e aquilo que precisamos
de largar. E esta simples franqueza interior é, muitas vezes, tudo o que na
realidade precisamos -- em geral, quando temos este ponto de vista, podemos
destrinçar os detalhes do que fazer e como fazer, ou de não fazermos seja o que
for.

Um outro desenvolvimento da atenção é a consciência estabelecida (\emph{sati})
-- a capacidade para manter em mente um tema, uma disposição, um pensamento ou
uma sensação. Trata"-se de uma utilização adequada da \emph{mano}, a mente"-órgão.
Enquanto que a atenção compreensiva é uma atenção activa que passa os tópicos da
mente a pente fino, a consciência estabelecida mantém a atenção num ponto -- num
pensamento ou numa sensação -- de forma a olhar para a natureza deste ou desta
enquanto fenómeno. Por exemplo, a consciência estabelecida tem em conta uma
emoção enquanto emoção e não deixa que esta se consolide numa atitude ou numa
acção. Mantém a fronteira do momento presente, de forma que conseguimos
realmente discernir o que é um sentimento e o que é uma disposição, em vez de
agirmos com base neles, de os tentar explicar ou suprimir. A consciência
estabelecida é vital, uma vez que ao nível das sensações não existem fronteiras
-- as sensações mentais vão a todo o lado. E se esse sentimento começa a
proliferar, torna"-se `Eu sou. Sempre serei. As pessoas não gostam de mim. Sou
terrível\ldots' -- e vai continuando a ressoar. Mesmo no caso de uma disposição
positiva, se a consciência estabelecida se encontra ausente, podemos partir do
princípio de que tudo é fantástico e sermos bastante insensíveis às disposições
dos outros. Por isso é sempre bom estabilizar o domínio da \emph{citta} com a
consciência estabelecida. Assim, não nos prendemos à percepção e ao sentimento,
e não proliferamos à volta das impressões do coração ou de estados da mente que
podem surgir subsequentemente.

Um complemento para a consciência estabelecida é a `consciência plena'
(\emph{sampajañña}). A consciência plena é a capacidade de estarmos atentos e
receptivos, a capacidade de termos noção e compreendermos aquilo a que somos
sensíveis. É baseado na \emph{citta}. A consciência estabelecida mantém uma
fronteira, de forma a não ficarmos assoberbados, fechados ou a reagirmos aos
sentimentos que temos. Então, com a consciência plena, compreendemos o todo,
compreendemos como as impressões surgem e o que fazem. Podemos então perceber:
`este sentimento ou esta impressão baseia"-se nesta percepção e neste pensamento,
e desaparece quando esse pensamento ou essa percepção são retirados.' `Esta
impressão negativa surge com aquela percepção ou aquela memória, e desaparece
quando eu ponho em prática a benquerença, ou até mesmo quando consigo apenas
observá"-la e deixá"-la desaparecer.' A consciência estabelecida e a consciência
plena, em conjunto, reconhecem o que está a acontecer e onde acaba. Elas não
trazem o `eu sou' nem o `eu devia ser' para a equação.

Se instituirmos estas capacidades conscientes, elas libertam a mente de agir na
sequência dos resultados do passado, ou de reagir a estes. Se prestamos atenção
às impressões actuais, às disposições e sensações actuais, e cortarmos as
proliferações e as projecções, não vivemos na névoa do ressentimento, da
fantasia, do romance e de outras ideias preconcebidas. Isto significa que a
nossa atenção e, consequentemente, as nossas disposições, acções e discurso, vão
ser mais claros e luminosos. Devido a isto, podemos ficar mais libertos da nossa
acção habitual -- ou inacção. (Privarmo"-nos de agir é, ainda assim, uma acção --
e isso torna"-se igualmente um hábito!) Mas se estamos cuidadosamente atentos ao
coração, podemos falar sobre como as coisas parecem, que incidentes deram origem
ao `sentimento' de sermos maltratados, e ter uma sensação de que,
independentemente de mais alguém escutar ou responder ou não, pelo menos
trouxemos alguma clareza às nossas vidas. Não temos de estar a criar novo kamma
com base nos hábitos antigos -- a atenção sábia é o kamma que conduz ao fim do
kamma.

\subsection{Atenção que guarda e que tranquiliza}

Estabelecer presença de mente e total consciência na vida quotidiana requer uma
filtragem cuidada da informação que nos chega de todas as direcções, uma vez que
o simples dilúvio do contacto pode ser avassalador. O contacto é uma forma de
kamma: aquilo a que damos atenção recebe a nossa energia e entra nos nossos
corações, onde estimula a acção e a reacção.\pagenote{``\ldots{} seja o que for
  que um bhikkhu pensa e pondera com frequência, isso irá tornar"-se a tendência
  da sua mente. Se ele pensa com frequência e pondera sobre pensamentos de
  renúncia, boa vontade\ldots{} inofensividade, então a sua mente tende para a
  renúncia, boa vontade\ldots{} inofensividade.''
  \href{https://suttacentral.net/mn19/en/bodhi}{MN 19.8}}
Uma vez que, em consequência disto, desenvolvemos hábitos claros ou escuros,
temos de ser responsáveis relativamente àquilo a que damos atenção. Parte deste
cultivo envolve, desta forma, virar costas a informação e material que leva a
mente para o anseio, para a aversão ou para a distração. Por isso, uma outra
função do discernimento é ser discriminativo: ter intenção, verificar, passar a
pente fino, deitar fora a escória e reter o ouro.

Na realidade, em vez de ter a mente absorvida seja no que for que os meios de
informação estão a despejar, existem temas aos quais é bom dar
atenção.\pagenote{``E quais são as coisas dignas de atenção às quais ele se
  dedica? Trata"-se de coisas cuja natureza faz com que (quando ele se dedica a
  elas): a mácula ausente do desejo sensual não surge e a mácula presente do
  desejo sensual é abandonada, a mácula ausente do devir não surge\ldots{} e a
  mácula presente do devir é abandonada, a mácula ausente da ignorância não
  surge\ldots{} e a mácula presente da ignorância é abandonada.''
  \href{https://suttacentral.net/mn2/en/bodhi}{MN 2.10}}
A compreensão tem a ver com a reflexão. Existem variadas reflexões, mas
reflectir é algo que podemos fazer durante o dia. Antes de mais existe a
mortalidade, e ao considerarmos o facto da mortalidade de forma cuidadosa e
abnegada, estamos a ajudar a mente a manter"-se calma e estável -- não ficamos
imprudentes ou egoístas e não mantemos ressentimentos. A percepção da
mortalidade leva a que algumas das coisas que nos prendem percam a sua força.
Onde é que está a pressão para obter ou para ser algo quando perdemos tudo o que
alcançamos? Ao que é que vale realmente a pena dar atenção e tempo? A recordação
da mortalidade também nos lembra que os nossos recursos, a nossa energia, a
nossa capacidade mental e a nossa saúde são finitos e em declínio. Podemos usar
os nossos recursos de uma forma que vá potenciar ou libertar as nossas vidas ou
podemos desperdiçar tempo com fantasias e frustrações. Então, usada de forma
sábia, a lembrança e reflecção sobre a morte mantém a mente em forma, limpa e
presente. Diz"-nos que é tempo de largar o nosso fardo.

Outra característica positiva que decorre ao reflectirmos sobre a mortalidade é
a empatia. Uma das maiores fontes de sofrimento, e base para o \emph{kamma}
negativo, é a perda de empatia para com os outros. Na vida urbana moderna,
\mbox{podemos} experienciar muitas pessoas através dos estereótipos mediáticos, ou na
`terra de ninguém' das ruas agitadas e locais públicos. As pessoas tornam"-se,
desta forma, `outros' - outras nacionalidades, outras religiões, etc. -- e nós
podemos sentir indiferença ou desconfiança em relação a elas. Num campo
emocional com este tipo de enviesamento, a indiferença e até a brutalidade
encontram terreno fértil para proliferar. Mas se tivermos em consideração o
nosso terreno comum -- que, como nós, também os outros passam por tensão,
doença, perda e morte -- é mais fácil gerar"-se empatia. Por exemplo, um dos
monges mencionou que sempre que a vida está a tornar"-se um pouco tensa e ele
começa a sentir"-se irritável ou a perder a perspetiva, ele olha para imagens de
vítimas de fome e de pessoas com doenças e deformidades terríveis. Então
experiencia um sentimento de compaixão pelo reino humano, bem como gratidão pela
enorme bênção que é ser saudável, livre de penúria, bem alimentado e cuidado. A
reflexão evoca um estado que ao ser mantido conscientemente, pode vir a
tornar"-se um local estável de permanência para o coração. Então a rudez, a
indiferença e a autocomiseração não prevalecem.

Podemos igualmente alargar a empatia de forma a nos lembrarmos que os outros
também têm alegria e desespero, humor e medo, nascimento, famílias e o seu
kamma\ldots{} Então, porque é que eu não percebo os outros da mesma forma como
eu gostaria que eles me percebessem? A moralidade, na verdade, resume"-se à
empatia traduzida em formas de comportamento.

Considero muito útil meditar sobre os `outros' e sobre aquilo que suscitam em
mim. E reparar que qualquer efeito que surge, surge na minha própria mente
afectiva -- porque sou eu quem tem de viver com essa indiferença, rudez ou
empatia. Quando o coração é defensivo ou displicente, fica apertado, oprimido, e
não consegue aceder à energia que me dá suporte. E quanto mais me sinto pesado e
contraído relativamente aos outros, mais pesada e contraída fica a minha vida.
Evidentemente que abrir o coração traz todo o tipo de irritações e de medos
condicionados mas, se existe atenção compreensiva, o coração também tem acesso à
coragem e à compaixão que constituem o seu potencial. E, à medida que me
sintonizo com o tema do bom kamma da condição humana, posso realmente apreciar e
saborear a nutrição proporcionada pela bondade, pelo cuidado protector da
compaixão, pela alegria do reconhecimento e pela equanimidade, de forma a manter
o espaço que permite às emoções movimentarem"-se. A empatia dá"-me acesso à minha
sanidade inata.

\subsection{Imagens do Despertar}

De forma a alegrar o coração, é bom trazer à mente uma imagem ou um tópico.
Geralmente o mais útil é a recordação das qualidades das pessoas que fazem parte
da nossa vida, porque nós aprendemos muito sobre o kamma luminoso através da
observação das acções dos outros. Assim, um dos maiores apoios para o Despertar
é ter relacionamentos significativos com as outras pessoas. Isto pode incluir os
nossos pais, amigos ou pares, que representam ou invocam os nossos sentimentos
de gratidão, integridade, compaixão -- valor. Sem pontos de referência humanos,
vivos ou já falecidos, a mente está a lidar com abstracções, inclusive em
relação a si própria. As pessoas isoladas ficam presas a noções irrealistas
delas próprias, ou a passatempos, planos, aparelhos ou várias formas de
entretenimento. Não existe um sentimento de ligação ou de fazer parte de algo
maior do que nós. Isso constitui uma perda enorme.

Para trabalhar contra isto, a lembrança do Sangha leva"-nos à humanidade da
prática: não se trata apenas de algo que envolve um manual e algumas ideias. Um
dos principais benefícios de uma linhagem e de uma tradição é despertar"-nos para
um sentido maior de nós próprios -- para a partilha da camaradagem espiritual
com pessoas boas, pelo mundo inteiro, através dos tempos. Podemos igualmente
recordar a partilha de um sistema de valores que dá um grande significado ao
kamma: é esta a lembrança do Dhamma. De modo que recordamos a aspiração e o
Despertar como a nossa referência comum, e o sofrimento e o mal"-estar como o
nosso desafio comum. Então deixamos de nos sentir tão sós com os nossos estados
mentais difíceis e conseguimos lidar com eles de uma forma mais aberta e
consciente. A recordação do Dhamma e do Sangha lembra"-nos que, apesar de
existirem avidez, zanga e confusão, existe sempre uma forma de lidar com elas,
que nos leva além desse âmbito. E existem pessoas que percorreram esse caminho.

O próprio contexto da prática pode ser elevado através da utilização de altares,
de oferendas a uma imagem do Buddha e da entoação de cânticos. Isto constitui o
\emph{puja}: o acto de prestar homenagem ao Buddha, trazendo à mente o milagre
do Despertar numa forma corpórea. Mas não se trata de adorar uma imagem. Nós
utilizamos o ritual porque isso nos dá a oportunidade de agir, ao invés de
pensar, e podemos fazê"-lo em conjunto, através do corpo, do pensamento e do
coração. Uma acção de grupo eleva o sentimento de participação no significado do
Despertar. Uma sintonização e uma participação plenas levam"-nos para fora do
nosso pequeno `eu' e proporcionam uma ressonância profunda.

É por esta razão que, num mosteiro, temos uma imagem tangível e manifesta do
Buddha. É algo que podemos tratar respeitosamente -- limpá"-la fisicamente,
iluminá"-la com luzes, oferecer"-lhe flores. Luzes suaves, flores e gestos de
oferendas encorajam a atenção a permanecer no sentimento do coração em relação
ao altar e, desta forma, a mente é tocada pelo sentimento de estabilidade, de
tranquilidade ou de radiância, e pode permanecer aí. Se estas impressões e
significados sentidos forem estabelecidos de forma regular, atinge"-se um ponto
no qual a simples visão de uma imagem do Buddha eleva o espírito ou acalma a
mente.

Entoar cânticos, particularmente em grupo, pode ter um efeito de harmonização,
de acalmia: de forma sonora e pausada, pode ajudar"-nos verdadeiramente a
apreciar os nossos companheiros de prática. Aqui estamos nós -- pelo menos desta
vez sem os nossos nomes e histórias -- seres humanos que têm como intenção estar
plenamente conscientes. Então temos a noção da nossa própria presença inserida
numa perspetiva mais abrangente. De certa forma, continua a ser somente o nosso
corpo e a nossa mente com todas as suas idiossincrasias, mas a
recordação/reflexão leva a um conhecimento empático de tudo isso.

\subsection{O não envolvimento necessita de apoio}

A atenção compreensiva é, desta forma, uma acção que nos faz parar e que nos
leva, mais profundamente, para as nossas mentes e para os nossos corações. Isto
prepara"-nos para a meditação. Se começamos a meditar a partir de um ponto escuro
ou turvo, a consciência estabelecida e a consciência plenas são fracas. Podemos
dizer a nós próprios que ser escuro ou turvo é sermos autênticos e que devemos
apenas estar conscientes disso. O que, de certa forma, é verdade. Contudo, as
memórias, os planos, as preocupações e os ressentimentos, em geral, são tão
poderosos que em vez de estarmos conscientes deles, eles capturam a nossa
atenção e tornam"-se obsessivos. Assim, é importante estabelecermos um foco com a
visão correcta -- termos em mente como estamos a ser afectados por aquilo ao
qual estamos a dar atenção. É inútil despendermos tempo com a nossa atenção
aprisionada por preocupações ou ressentimentos. Sentarmo"-nos sem recursos para o
coração não é meditação.

É mais produtivo entrar na meditação através da compreensão, até mesmo
ponderando e considerando de que forma a mente está a ser afectada pelas coisas.
Ou seja, nós lidamos com a mente de modo cordial, examinamos e discernimos: `O
que traz sofrimento e tensão? O que os faz soltar?' Esta intenção traz uma
consciência que apoia a sabedoria do Despertar: olhar para o mal"-estar e para a
libertação do mal"-estar. A consciência apenas mantém as coisas em mente. Assim,
para o Despertar, o coração"-base precisa de ser apoiado pela intenção correcta
-- que resulta da atenção compreensiva e da consciência plena.

\enlargethispage{\baselineskip}

A prática não irá longe se a acção da mente não estiver a agir com base em
compromisso, ética e destreza. Mas o cultivo da mente de acordo com estas
directrizes, é um apoio para a compreensão, a atenção e a consciência. Então
conseguimos enfrentar estados positivos, estados negativos ou estados `assim
assim', e encontrar sabedoria e libertação. Porque o que é essencial em todos os
casos é que exista uma distanciação consciente do impulso de um hábito ou da
inclinação de uma disposição emotiva. Isto leva a um apaziguamento da
intensidade e do impulso do estado de espírito, de forma que existe um
enfraquecimento momentâneo ou o finalizar do kamma. Esta mudança acontece quando
somos claros e honestos: não acontece se estamos a esconder alguma coisa ou a
tentar fazer com que algo aconteça -- incluindo estarmos a tentar ser
desapegados! Isto é assim porque a mudança implica afastarmo"-nos da tentativa de
encontrarmos ou de sermos algo, e de sermos íntegros e claros perante as nossas
questões. E o bem que já fizemos, a nossa paciência e a nossa honestidade,
contribuem para fortalecermos a mente"-base de forma a tornar esse distanciamento
possível.

Assim, quando existe escuridão no coração, sabemos como lidar com ela de forma
sensata. Não temos de descobrir qual a sua origem e quem tem responsabilidade.
Talvez provenha de alguma acção passada ou talvez seja um hábito crítico e
negativo, que cria no presente uma memória ou uma interpretação. Mas tudo o que
temos de saber é que isso é \emph{vipāka} escuro e onde é que ele se desvanece.
Diria que o processo é quase como colocar uma peça de roupa suja num lago. A
limpeza é realizada simultaneamente pela acção de colocar a peça de roupa no
lago e pela ausência de acção -- uma vez que a água trata da limpeza. Pegamos
nesse resíduo escuro e colocamo"-lo seja em que clareza ou pureza possa existir
e, apesar de termos de esfregar bem os bocados mais encardidos, é a nossa
sanidade básica que lava a sujidade.

Estabelecemo"-nos num estado desperto e então a consciência continua a
experienciá"-lo, a senti"-lo, e a deixar ir aquilo que surge. Quando algum resíduo
escuro é removido, a plena consciência sente a leveza, a luminosidade. E podemos
sintonizar"-nos com isso. Ao longo do tempo, à medida que vamos cultivando isto,
desenvolve"-se uma crescente base de bem"-estar, uma luminosidade na qual podemos
permanecer. Mas devido ao facto de não existir um sentimento de `eu fiz isto' ou
de `eu vou conseguir isto', a mente não fica enfatuada, permanecendo tranquila e
receptiva.

\enlargethispage{\baselineskip}

De facto, qualquer tipo de opinião sobre nós próprios só confunde pois, como nos
apercebemos na meditação, ficarmos presos nos pensamentos e disposições não
depende de uma decisão pessoal. E também não se trata de `eu não estou apegado'.
Imaginemos que a mente se encontra ampla e estável e nisto surgem pensamentos
sobre o nosso futuro, ou relativamente a alguém que nos está a dificultar a vida
e, de repente, lá estamos, a apertarmo"-nos, a apressarmo"-nos e a proliferarmos
acerca de tudo isto. E surge a opinião relativamente a quem é o culpado\ldots{}
e sobre o que devemos fazer\ldots{} e porquê eu\ldots{} Isto começa com uma
tendência, avança para uma acção e depois torna"-se uma pessoa. De forma que a
libertação não pode ser atingida através de um eu aparente que tem, ou devia
ter, o controlo. Ao invés, requer as capacidades de uma atenção que consegue
lidar com o kamma antigo à medida que este vem à tona.

A nossa prática é assim conduzida pelo Dhamma e não pelas nossas opiniões sobre
nós próprios, direccionando"-se tendencialmente para o término do que é antigo,
em vez de nos tornar em algo novo. Trata"-se de um cultivo que nos liberta e
protege, e nos faz convergir para um espaço livre no centro da vida. É o kamma
que leva ao fim do kamma e tem o sabor da liberdade.

\clearpage

\section[Meditação: recordar]{Meditação}

{\centering
\subSectionFont\selectfont
\textit{Recordar}
\par}

\bigskip

Sente"-se com uma postura alerta e verticalizada, que permita ao corpo estar
confortável e sem se remexer, mas que encoraje a vigília. Deixe os seus olhos se
fecharem ou semicerrarem. Traga o foco da mente conscientemente ao corpo,
sentindo o seu peso, as suas pressões, pulsações e ritmos. Traga à mente a
sugestão de se estabelecer onde se encontra neste momento e pôr de lado, por
agora, outras preocupações ou interesses.

Realize algumas expirações lentas e longas, a sentir a sua respiração a sair
para o espaço que o rodeia. Deixe que a inspiração comece por si própria. Sinta
como a inspiração vai buscar o ar ao espaço à sua volta. Entre em sintonia com o
ritmo desse processo e interrompa quaisquer pensamentos que o distraiam através
do restabelecimento da sua atenção em cada expiração.

Traga à mente qualquer exemplo de acções de pessoas que o tenham tocado de uma
forma positiva, em termos de bondade, paciência ou compreensão. De forma
repetida, toque o coração com alguns exemplos específicos, estabelecendo"-se no
sentimento que evocam.

Permaneça, durante um minuto ou dois, com a sua recordação mais profunda,
cultivando uma atitude de curiosidade: `Como é que isto me afecta?' Aperceba"-se
de qualquer efeito no coração: pode haver uma qualidade de elevação, de acalmia
ou de firmeza. Pode mesmo detectar uma alteração no tónus global do corpo. Dê"-se
a si próprio todo o tempo para estar aqui, sem qualquer objectivo específico
para além de sentir como se relaciona com o momento, numa atitude de compaixão
observadora.

Estabeleça"-se nesse sentimento e concentre"-se em particular na tonalidade da
disposição, que pode ser de luminosidade, de estabilidade ou de elevação. Ponha
de lado o pensamento analítico. Permita que, na mente, quaisquer imagens surjam
e desapareçam. Permaneça e expanda a consciência da sensação de vitalidade ou de
tranquilidade, conforto, espaço ou luz.

De acordo com o tempo e a energia, conclua o processo sentindo plenamente quem
`você é' nesse estado. Primeiro sinta como está em termos físicos. Seguidamente
note quais as tendências e as atitudes que parecem naturais e importantes,
quando permanece nesse seu local meritório. Seguidamente leve essas tendências e
atitudes para a sua situação do dia a dia interrogando"-se: `O que é importante
para mim neste momento?', 'O que tem maior importância?' Depois dê"-se tempo a si
próprio para que as prioridades da acção se estabeleçam de acordo com a
resposta.
