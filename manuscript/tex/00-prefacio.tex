\chapter{Prefácio}

Este livro foi desenvolvido a partir de uma série de palestras que dei no
decurso de alguns anos, principalmente no Mosteiro de Cittaviveka. Nestas
palestras explorei a relevância dos ensinamentos do Buddha sobre o kamma para a
prática da meditação. À primeira vista pode parecer que estes dois tópicos não
estão muito relacionados: o kamma é um ensinamento sobre o comportamento,
enquanto que a meditação, aparentemente, é sobre a ausência da acção, não é
verdade? Ou podemos ter a ideia de que: `O kamma é sobre aquilo que eu fui numa
vida anterior, aquilo ao qual eu estou preso agora e aquilo que será o meu
renascimento. O kamma é acerca de ser alguém, enquanto a meditação é acerca de
não ser seja quem for.' Não é assim. Espero que os textos que se seguem,
compilados a partir de palestras e que se tornaram ensaios, ajudem a clarificar
que os princípios do kamma ligam o comportamento `externo' à prática `interna'
da meditação. E que a meditação é um tipo de kamma -- o kamma que leva ao final
do kamma. Com efeito, `kamma e o fim do kamma' é um bom resumo daquilo que o
Buddha tinha para oferecer, como um caminho para o bem"-estar e para o Despertar.

\subsection{Os três conhecimentos do Buddha}

A experiência do Dhamma do Buddha tem a sua fundamentação nos `três
conhecimentos', constatações que se diz terem ocorrido sequencialmente ao
Buddha, numa noite. Apesar de praticar meditação e ascetismo a um nível muito
elevado, o Buddha sentiu que esta prática não lhe tinha dado frutos em termos da
sua busca da `transcendência da mortalidade'. Mas eis que, ao praticar mais
tranquilamente, surgiram três compreensões e, com estas, o seu objectivo foi
atingido.

A primeira destas compreensões foi a consciência de vidas anteriores. Este
conhecimento transcendeu a definição mais fundamental das nossas vidas -- a
divisão que ocorre na morte do corpo. Surgiu a compreensão de que aquilo que é
experienciado como uma `pessoa', constitui uma manifestação num processo em
decurso, ao invés de um eu único e isolado.

Entrar em contacto com esta força vital subjacente não constituiu um grande
conforto: expandiu a questão da existência para além da duração de uma única
vida, para um purgatório existencial constituído por um `vaguear' sem fim
(\emph{saṁsāra}). Contudo, por outro lado, existia uma consciência que tinha
transitado para essa panorâmica transpessoal. A porta para a `transcendência da
mortalidade' tinha começado a abrir"-se.

A segunda compreensão foi que a `direcção' desse vaguear não era aleatória --
este toma a sua direcção de acordo com a qualidade ética das acções que a pessoa
empreende. Surgiu o conhecimento que existem energias que são disruptivas ou
abusivas e que não sustentam a lucidez ou a saúde. Por outro lado, existem
energias que são harmoniosas, que alimentam ou que estão claramente em sintonia.
`Más' e `boas' (ou inadequadas / deletérias / `escuras' e
adequadas / salutares / `luminosas' em termos Budistas) não são, consequentemente,
apenas juízos de valor impostos por uma sociedade, mas sim referências a
energias que são palpáveis em termos psicológicos, emocionais e físicos. A acção
empreendida de acordo com uma energia sã constitui um suporte para o bem"-estar e
para a harmonia, tal como o contrário tem o efeito oposto. Este constitui o
princípio de causa e efeito em termos éticos, ou `\emph{kamma"-vipāka}'.

Contudo, \emph{kamma"-vipāka} tem um significado mais profundo, que vai para além
de acções e resultados. Se a consciência cognitiva se apega a uma acção (física,
verbal ou psicológica) dá origem a uma consciência cognitiva à qual é dada uma
forma pessoal através desse mesmo apego. Esta acção cria a impressão de um eu,
que é o resultado dessa acção, e é `condimentada' pela qualidade ética dessa
acção. Dito de uma forma simples: não se trata tanto de eu criar kamma mas sim
do kamma `me' criar. Desta forma \emph{kamma"-vipāka} transcende a separação
entre acção e agente, mergulhando a consciência num campo de significado ético,
onde todas as acções formam e informam o `eu', o `meu', o `eu próprio'.

Da mesma forma, uma vez que o kamma surge numa maré de energia causal em
decurso, os resultados das acções podem ter lugar em vidas futuras. E isto
significa que o kamma nos prende ao processo de nascimento e morte -- algo que o
Buddha recordou em termos dramáticos:

\begin{quote}

  ``Há muito que (repetidamente) experienciaste a morte de uma mãe\ldots{} de um
  pai\ldots{} a morte de um irmão\ldots{} a morte de uma irmã\ldots{} a morte de
  um filho\ldots{} a morte de uma filha\ldots{} perda relativamente a
  familiares\ldots{} perda relativamente a riqueza\ldots{} perda relativamente a
  saúde\ldots{} As lágrimas que derramaste sobre a perda relativamente à doença
  enquanto transmigravas e vagueavas durante este longo, longo tempo -- a chorar
  e lamentar por estares associado ao que é desagradável, estares separado
  daquilo que é agradável -- têm um volume que supera o total da água dos quatro
  grandes oceanos.

  Porque é que isto sucede? A transmigração surge de um começo que não é
  passível de ser reconstituído. Não é evidente um ponto de começo, apesar dos
  seres dificultados pela ignorância e algemados pelo anseio continuarem a
  transmigrar e a vaguear. Assim, há muito, que experiencias tensão, dor, perda,
  há muito que preenches os cemitérios -- o suficiente para ficar desencantado
  com tudo o que é fabricado, o suficiente para ficar desapaixonado, o
  suficiente para ser libertado.''

  \href{https://suttacentral.net/sn15.3/en/thanissaro}{SN 15.3}

\end{quote}

Contudo, nesta admoestação existe igualmente a mensagem da possibilidade da
libertação deste \emph{saṁsāra}: através do acto de limpar e de largar, bem como
da `cessação' desses mesmos padrões de energia que transportam a causa e o
efeito -- e~o~`eu'.

Estes últimos foram vistos a partir do terceiro conhecimento, que trouxe a
libertação das `coisas fabricadas', das emoções, dos impulsos, das sensações e
das respostas que constituem a matéria"-prima do apego. Este conhecimento é a
compreensão dos preconceitos subjacentes (\emph{āsava}) que condicionam o apego
através do qual o \emph{saṁsāra} funciona.

\subsection{Renascimento e kamma}

O \emph{saṁsāra} não se expressa por intermédio de um corpo ou uma identidade.
Os corpos mantêm"-se dependentes das condições apenas durante uma vida. A
identidade -- por exemplo filha, mãe, director, inválido, etc. -- surge como
dependente de causas e condições. Aquilo que anteriormente foi referido como
`transmigração' não é `renascimento', mas o processo pelo qual uma corrente
duradoura de apego continua a gerar seres sencientes. Para além disso, esta
corrente não é algo que apenas ocorre na morte, mas que é alimentada de forma
contínua pelo kamma no momento presente. Através de uma tendência denominada
`devir', o \emph{kamma} forma algo semelhante a um código genético psicológico.
Este código, que constitui o padrão da herança cármica de cada indivíduo, é
formado através de processos dinâmicos chamados \emph{saṅkhāra}. Tal como o
código genético pessoal de cada um, o \emph{saṅkhāra} retém os esquissos
originais cármicos e, desta forma, permanecemos de dia para dia, em termos
relativos, a mesma pessoa.

Traduzido de forma variada como `formações', `formações volitivas',
`fabricações', etc., eu traduzo \emph{saṅkhāra} como `programas e padrões'.
Alguns destes programas são funções, tal como o metabolismo, que se encontram
ligados à força vital (\emph{ayusaṅkhāra}); alguns são veiculados pela
consciência que é gerada a partir de vidas anteriores; e alguns são formados
através de interacções da vida presente. Os programas mais significativos do
ponto de vista da transmigração são aqueles que enraízam o apego no nosso corpo
e na nossa mente. Estas são três `corrupções' -- a corrupção associada à
informação dos sentidos com sentimentos que vão desde o deleite à aversão; a
corrupção de `devir/tornar"-se algo' que gera a sensação de um eu autónomo; e a
corrupção da ignorância que obscurece a verdade acerca do \emph{saṁsāra}. Estas
corrupções são, na melhor das hipóteses, insatisfatórias e, na pior, causadoras
de tensão ou de dor, uma vez que programam de forma contínua a mente para
depender das características mutáveis da informação sensorial e para formar uma
identidade com base no terreno instável dos estados de espírito. Devido a este
desconforto existencial, a consciência não"-Desperta reage com hábitos de
ganância, aversão e confusão.

As boas notícias é que não estamos tão embutidos no \emph{saṁsāra} como parece.
Nem todos os aspectos da mente se encontram presos nas corrupções. Nós podemos
`ter consciência' das corrupções. E são apenas as corrupções que têm de ser
eliminadas. A acção que constitui uma preparação e que culmina com este corte é
o Nobre Caminho, o Caminho Óctuplo que o Buddha ensinou. Praticar este caminho é
o kamma que leva ao final do kamma.

Os ensinamentos sobre o kamma, sobre causa e efeito, proporcionam"-nos um Caminho
para a libertação. De forma abreviada, quando sabemos como funciona e temos as
capacidades e as ferramentas para desconstruir os programas da consciência
cognitiva, podemos parar de construir o \emph{saṁsāra}. Mais detalhadamente, os
ensinamentos do Buddha constituem um guia para a acção sustentada, tanto no
domínio externo como interno, em que a prática no domínio externo -- o de viver
uma vida moral e receptiva -- estabelece as directrizes e encoraja as
capacidades de forma a esclarecer o domínio interno. Factores causais
transpessoais, tais como o estar presente, a investigação, a concentração e a
equanimidade, podem proporcionar o efeito de abolição da modelação das
corrupções. Esta prática, mais do que extrair a pessoa do \emph{saṁsāra}, vai
desligar o processo do \emph{saṁsāra} para aquela consciência individual. O que
resta para aquele indivíduo pode ser resumido como um corpo e uma mente a
funcionarem durante esta vida, bem como uma consciência sem preconceitos, que
não participa na dinâmica da continuação dos renascimentos.

\subsection{O kamma por detrás deste livro}

Admito a existência de determinados enviesamentos. Existem muitas formas de
meditar e muitos professores experientes que podem constituir, neste campo, um
guia para os outros. Contudo, acontece com frequência que as pessoas necessitam
de ajuda na integração da atenção meditativa e da realização interior nas suas
vidas quotidianas. Isto, em parte, deve"-se ao facto da compreensão que as
pessoas ganham nos retiros de meditação não surgir do contexto da vida e das
actividades do dia a dia, mas sim de um ambiente especializado, tal como um
centro de meditação. Nestas condições, é dada uma atenção considerável a
temáticas como a consciência da respiração. Contudo, os temas que não são
relevantes, em termos imediatos, para a solidão e tranquilidade de estar sentado
num retiro, não recebem a atenção necessária para uma vida harmoniosa. As
relações interpessoais, quando saímos de um cenário de `ausência de contacto,
ausência de fala' de um retiro, parecem difíceis de integrar e, contudo, são
parte da vida de todos nós. De igual modo, se desenvolvemos a meditação no
sentido de aceitarmos pacificamente aquilo que está presente na nossa
consciência, como é que tomamos decisões no que diz respeito a escolhas
relativas à nossa subsistência e desenvolvemos planos sustentáveis para o
futuro? De igual forma, será que podemos obter orientação no tipo de
desenvolvimento pessoal que nos encoraja a sermos responsáveis, que nos permite
aceitar, partilhar ou questionar a autoridade, bem como tudo o resto de que a
sociedade necessita em termos de indivíduos com maturidade?

Infelizmente pode acontecer que as pessoas tenham experiências `espirituais'
válidas a um nível subtil mas permaneçam contudo bastante alheadas aos seus
próprios conceitos e ângulos mortos, em termos de acções e de interacções
pessoais. O que ajuda, na meditação e na vida do dia a dia, é aprender a manter
e a moderar o nosso propósito; como sermos sensíveis e autênticos para connosco
próprios e para com os outros; e como valorizarmos e ajustarmos a qualidade de
esforço que colocamos nas nossas vidas. Tudo isto, e mais, se insere sob o
tópico relativo ao kamma. Desta forma, temos necessidade de conhecer muito bem o
que constitui um kamma cuidadoso, em termos de contextos internos subtis e de
contextos externos mutáveis mais óbvios.

Assim, saliento a importância da compreensão da causalidade, como uma chave para
o Despertar, uma vez que transforma a prática num caminho a seguir na vida como
um todo, em vez de uma técnica de meditação. Isso enfatiza o facto de o
\emph{saṁsāra} ser um hábito e não um lugar. Desta forma, `sair do
\emph{saṁsāra}' não envolve indiferença para com o mundo, o corpo ou os outros,
nem se refere à conquista de uma área de território psico"-espiritual refinado.
Trata"-se do abandono dessa negatividade, dessa indiferença e dessa ânsia; e isso
significa cultivar e honrar o bom kamma. Por sua vez, esta constitui a base para
a total libertação do mal"-estar pessoal. A beleza e a profundidade da explanação
do Buddha é, para mim, esta unidade: Verdade, Virtude e Paz podem surgir a
partir do mesmo foco.

Estas palestras não foram dadas em série, mas sim espaçadas no tempo. Deste
modo, foi necessário trabalhá"-las em termos de revisão, de forma a homogeneizar
a linguagem, cortar as repetições e acrescentar material a que os ouvintes,
nessa mesma altura, tiveram acesso através de outras palestras. Mesmo depois de
tudo isto concretizado, pareceu"-me que alguns aspectos careciam de uma
explicação mais completa. Consequentemente, introduzi mais material escrito para
além do material apresentado nas palestras, de forma a ampliar este último.
Pensei igualmente em colocar algumas citações e notas do Cânone Pali e, como se
estava a tornar bastante teórico, intercalei os ensaios com algumas instruções
de meditação, breves mas relevantes. De qualquer modo, este livro não constitui
uma exposição passo-a-passo, mas sim um vislumbre de vários pontos de interesse
e uma revisão desses pontos a partir da perspetiva do kamma.

Gostaria de agradecer a ajuda de Ajahn Thaniya, que selecionou e comentou os
textos, e de Dorothea Bowen, que realizou o trabalho de revisão dos textos
centrais. Ocasionalmente, o manuscrito circulou também entre outros amigos
contemplativos, com vista a obter críticas e comentários: Ajahn Amaro, Ajahn
Kovida, Ajahn Viradhammo e os amigos de Ottawa, Irmã Cintāmani, David e Nimmala
Glendining, Gabriel Hodes, Douglas e Margaret Jones. Ajahn Kusalo ocupou"-se da
composição tipográfica e do \emph{design}. Este livro inclui ainda o bom
\emph{kamma} de muitas outras pessoas e faço votos que isto continue a guiá"-las
no sentido de um maior bem"-estar e liberdade.

\bigskip

{\raggedleft
Ajahn Sucitto\\
Cittaviveka, 2007
\par}
