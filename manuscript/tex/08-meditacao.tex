\chapterNote{Afabilidade}

\chapter{Meditação}

\tocChapterNote{Afabilidade}

Com frequência iniciamos um período de meditação com interiorização e
verificando o nosso estado geral. O que é necessário é desenvolver um sentimento
de amizade para connosco próprios, estabelecendo uma atitude de não julgamento e
de interesse em proporcionar algum bem-estar imediato nas nossas vidas. A forma
mais imediata de conseguir isto é agora, precisamente onde nos encontramos, a
partir de uma atitude aplicada.

Estabeleça a sua presença no local onde se encontra sentado, pondo de lado as
suas preocupações. Depois interrogue-se: `Como estou neste momento?' Repita isto
algumas vezes, lentamente, e, apesar das sensações físicas ou estados de
espírito puderem alterar-se, atenda ao sentimento geral mais contínuo de ser
quem é.

Se a mente começar às voltas com memórias ou coisas que o/a leitor/a deve fazer
ou ser, tranquilize a energia ao seguir, durante um pouco, as expirações e as
inspirações. Adicione a sugestão de que essas respirações são algo certo,
\mbox{tranquilizador} e que contribui para o seu bem-estar. Pense lentamente ao longo
de toda a expiração: `Que eu esteja bem'.

Depois pense: `Como seria se\ldots{} eu me encontrasse na presença de alguém ou de
algo que estivesse a olhar para mim com amizade?' (Pode mesmo lembrar-se do seu
cão!) Acrescente o pensamento: `Como é que isso seria? Como é que eu
experienciaria isso?' e esteja atento a qualquer ressonância no coração.

Lembre-se de qualquer altura na sua vida na qual alguém se mostrou contente por
o/a ver, lhe fez algum favor, lhe deu alguma gentil atenção ou apreciou a sua
presença. `Como é isso, neste momento?' Depois: `Será que o meu corpo tem
conhecimento disso?' Esteja atento/a a algum decréscimo da tensão ou aumento de
energia -- particularmente na cara e na região do coração.

Ponha de parte reflexões mais gerais ou memórias dessa pessoa ou desse tempo e
regresse ao momento específico e à forma como o sentiu. Pode repetir isto com
várias pessoas e com vários episódios.

Quando conseguir estabelecer esse processo, permaneça no efeito ao nível do
coração e do corpo e, da mesma forma, diminua o pensamento. Gradualmente,
simplifique e consolide o processo até chegar a uma imagem simples (de
cordialidade ou de iluminação, por exemplo) ou a uma sensação física (de
conforto ou de contentamento). Sente-se com isso, levando isso para o varrimento
pelo corpo, como uma massagem. Expanda a sua consciência dessa sensação em
termos da sua disposição geral, até que o processo cognitivo deixe de ser
necessário.

À medida que se aquieta nisto, respire-o para a sua presença. Depois expanda-o
através da pele para o espaço que se situa imediatamente à sua volta. Pode
querer expressar essa benevolência para determinadas pessoas ou para os outros
seres em geral.

Seguidamente, traga à mente alguém relativamente a quem não nutre sentimentos
fortes. Considere encontrar essa pessoa fora do contexto onde habitualmente a
encontra. Imagine essa pessoa a divertir-se, ou preocupada, ou aflita. Despenda
algum tempo a completar a sua impressão dessa pessoa de forma simpática. `Que
ele/ela esteja bem.' Expanda a sua consciência do sentimento desse desejo: note
como isso afecta a sua disposição geral e o tónus do seu corpo. Disfrute do
facto de se sentir sintonizado/a de forma mais empática.

Permita que o sentimento e o seu efeito se instalem e aquietem. De seguida tenha
em conta alguém com quem sente dificuldades. Concentre-se num aspecto do
comportamento dele/a que não considere difícil. Considere encontrar essa pessoa
fora do contexto onde habitualmente a encontra. Imagine essa pessoa a
divertir-se, ou preocupada, ou aflita. Despenda algum tempo a completar a sua
impressão dessa pessoa. Sinta como é não se sentir assustado/a ou irritado/a com
essa pessoa. À medida que sente a sua descontracção, traga ao seu pensamento:
`Que eu me encontre livre de conflitos.' Expanda a sua consciência desse desejo
e dessa energia.

Agora pode ser possível estar apenas consigo, ao invés de estar dentro de si.
Explore a sensação da pessoa que acha que é, as suas disposições e sentimentos,
energias, processos de pensamento. E seja como for que ele/ela seja: `Que
ele/ela esteja bem. Que eu não tenha qualquer conflito com ele/ela.'

Quando quiser concluir, regresse à simples presença do corpo -- a sensação do
interior e a sensação periférica da pele, aquietando e estabilizando antes de
abrir os olhos.
