\chapterNote{Recordar}

\chapter{Meditação}

\tocChapterNote{Recordar}

Sente"-se com uma postura alerta e verticalizada, que permita ao corpo estar
confortável e sem se remexer, mas que encoraje a vigília. Deixe os seus olhos se
fecharem ou semicerrarem. Traga o foco da mente conscientemente ao corpo,
sentindo o seu peso, as suas pressões, pulsações e ritmos. Traga à mente a
sugestão de se estabelecer onde se encontra neste momento e pôr de lado, por
agora, outras preocupações ou interesses.

Realize algumas expirações lentas e longas, a sentir a sua respiração a sair
para o espaço que o rodeia. Deixe que a inspiração comece por si própria. Sinta
como a inspiração vai buscar o ar ao espaço à sua volta. Entre em sintonia com o
ritmo desse processo e interrompa quaisquer pensamentos que o distraiam através
do restabelecimento da sua atenção em cada expiração.

Traga à mente qualquer exemplo de acções de pessoas que o tenham tocado de uma
forma positiva, em termos de bondade, paciência ou compreensão. De forma
repetida, toque o coração com alguns exemplos específicos, estabelecendo"-se no
sentimento que evocam.

Permaneça, durante um minuto ou dois, com a sua recordação mais profunda,
cultivando uma atitude de curiosidade: `Como é que isto me afecta?' Aperceba"-se
de qualquer efeito no coração: pode haver uma qualidade de elevação, de acalmia
ou de firmeza. Pode mesmo detectar uma alteração no tónus global do corpo. Dê"-se
a si próprio todo o tempo para estar aqui, sem qualquer objectivo específico
para além de sentir como se relaciona com o momento, numa atitude de compaixão
observadora.

Estabeleça"-se nesse sentimento e concentre"-se em particular na tonalidade da
disposição, que pode ser de luminosidade, de estabilidade ou de elevação. Ponha
de lado o pensamento analítico. Permita que, na mente, quaisquer imagens surjam
e desapareçam. Permaneça e expanda a consciência da sensação de vitalidade ou de
tranquilidade, conforto, espaço ou luz.

De acordo com o tempo e a energia, conclua o processo sentindo plenamente quem
`você é' nesse estado. Primeiro sinta como está em termos físicos. Seguidamente
note quais as tendências e as atitudes que parecem naturais e importantes,
quando permanece nesse seu local meritório. Seguidamente leve essas tendências e
atitudes para a sua situação do dia a dia interrogando"-se: `O que é importante
para mim neste momento?', 'O que tem maior importância?' Depois dê"-se tempo a si
próprio para que as prioridades da acção se estabeleçam de acordo com a
resposta.
