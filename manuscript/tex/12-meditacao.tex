\chapterNote{Encontrar-se com o espaço: uma meditação em pé}

\chapter{Meditação}

\tocChapterNote{Encontrar-se com o espaço: uma meditação em pé}

Fique em pé com os pés paralelos, afastados à largura do corpo. Entregue o peso
do seu corpo ao chão, através das plantas dos pés. Como o corpo está acostumado
a manter"-se levantado ou a inclinar"-se sobre algo, com frequência
`esquece"-se' de se apoiar nos pés -- então descontraia conscientemente os
joelhos, as nádegas e os ombros. Descontraia o maxilar e a zona em torno dos
olhos.

Cultive a sensação de que o sítio onde está de pé oferece"-lhe suporte e é
completamente seguro. Pode saber isto na sua cabeça, mas não no peito, na
garganta ou nos ombros. Por isso, gradualmente, verifique o corpo. Depois sinta
através da pele, ficando consciente de `tocar' o espaço. Permita que o corpo
sinta e reconheça plenamente que o espaço (primeiramente à frente, depois em
cima e seguidamente atrás) se encontra desobstruído e não é intrusivo.
Desenvolva esse tema. Por exemplo, interrogue"-se: `o que é que está atrás de
mim?' E depois reflicta: `atrás de mim está um apoio forte. Nada do qual me
tenha de proteger.'

Verifique a postura periodicamente, de forma a impedir que as nádegas, o peito,
os ombros e o abdómen fiquem tensos. Mantenha os joelhos suaves, deixando que o
chão suporte o peso do corpo. Deixe o corpo explorar esta sensação de ser
suportado pelo chão. Irá descontrair"-se, encontrar estabilidade e a respiração
vai"-se tornar mais ampla, com o seu ritmo a ajudar a receber e a libertar
qualquer tensão. Vai surgir uma sensação de preencher plenamente o espaço à sua
volta. Pode sentir"-se um pouco maior e mais `em casa'.

Permaneça com a sensação geral do corpo, sem perder a sensação de estar num
espaço e sem dar atenção a qualquer fenómeno externo em particular. Mantenha a
sua atenção onde a sensação do corpo entra em contacto com a sensação do espaço.
Provavelmente a mente vai querer ir para algum lado: para o corpo, para um
pensamento ou para uma atitude, ou para fora, para algum objecto visual. Vai
querer ter um propósito ou algo a que se agarrar. Pode existir uma luta para se
livrar de disposições e sentimentos. Contudo, mantenha"-se simplesmente centrado
na energia do corpo, ou nas disposições que surgem face à sensação de encontro
com o espaço que o rodeia. A energia do corpo pode ser experienciada como
correntes ascendentes ou como tremores; pode ser sentida no peito ou no abdómen.
Naturalmente, podem ocorrer estados emocionais correspondentes, tais como
entusiasmo ou nervosismo. Pode experienciar rubores de tensão que se libertam.
Sintonize"-se com o eixo vertical do corpo -- um fio imaginário que liga as
plantas dos pés, ao sacro, à coluna e para cima, através do pescoço e do alto da
cabeça. Estenda esse fio para baixo em direcção ao chão e para cima, através do
alto da cabeça, para o espaço acima de si. Deixe que o seu corpo seja uma conta
neste fio. Inspire e expire de forma a proporcionar uma sensação de estabilidade
e bem"-estar.

Não embarque em quaisquer estados físicos ou emocionais, mas mantenha"-se
consciente do conjunto do fio, do eixo de equilíbrio -- ou da sua maior parte,
tanto quanto lhe for possível. Dentro desta sensação alargada do corpo, permita
que as energias e as disposições se movimentem, à medida que, muito lentamente,
perscruta conscientemente no sentido descendente pela sua cabeça, pela garganta
e pela parte superior do tronco. Utilize a actividade de `trazer à mente' e
`avaliar': ou seja, pense em `testa' e seguidamente reflicta sobre a forma como
sente a testa em termos das características dos elementos. Sente-a como firme,
sólida ou apertada (terra)? Está quente ou fria (fogo)? Existem movimentos de
energia ou palpitações nessa área?

Pode detectar tensões subtis na zona dos olhos ou em redor da boca, ou na zona
da garganta e da parte superior do tronco. Se isso acontecer, centre"-se
novamente no eixo de equilíbrio e, lentamente, estenda a sua atenção ao longo da
área na qual se está a focar e ao espaço imediato que circunda o seu corpo.
Pratique o encontro com seja o que for que surja, sem se envolver nisso.
Contudo, se surgir uma sensação de aperto, emotividade ou agitação, conecte"-se
com o eixo de equilíbrio, suavizando e ampliando a sua atenção.

Desenvolva a sensação de ser visto nesse estado de abertura, de forma simples e
grata. Esteja simplesmente presente com isso e com a forma como está a sentir.
Dê"-se tempo para sentir e disfrutar da sensação de se encontrar num espaço
benevolente. Pode ser benéfico imaginar que se encontra na luz, numa temperatura
amena ou na água.

Continue esta prática, movendo a sua atenção lentamente pelo tronco, pelas
pernas, até ao chão. O/A leitor/a pode não ter vigor ou tempo para trabalhar
todo o corpo. Se for o caso, desloque"-se mais rapidamente e procure percorrer o
tronco ou, pelo menos, uma das seguintes áreas: garganta e parte superior do
peito; coração e centro do peito; plexo solar e zona média do tronco; parte
inferior do abdómen, por baixo do umbigo.

Pratique de acordo com a sua capacidade e quando sentir que deve concluir,
despenda algum tempo a limpar o espaço das imagens e impressões. Depois
concentre"-se novamente na pele, distinguindo as suas fronteiras à volta de todo
o corpo. Seguidamente, sem perder a sensação de espaço, sinta a sua coluna e o
centro do corpo dentro deste espaço delimitado. Conclua a meditação ao
reconhecer os sons à sua volta, o campo visual e depois os objectos específicos
que o rodeiam. Movimente"-se ligeiramente, orientando"-se através do tacto.
