\chapter{Nota do Editor}

Ajahn Sucitto é conhecido e aclamado pela sua fabulosa e elaborada forma de
explanar o Dhamma, uma forma clara e objectiva, com uma precisão incisiva, mas
por vezes também poética. A sua escolha de palavras é exímia e cuidadosa, sendo
cada palavra apresentada fruto de uma estruturada e prolongada reflexão.

Queremos referir o trabalho árduo e dedicação que Alda Santos colocou nesta
tradução: colocar frases muito elaboradas e já por vezes longas no Inglês (e que
se transformam ainda mais longas no Português) de uma forma fluída e acessível é
uma mestria para a qual é necessário muito talento e engenho.

Este é um livro esclarecedor dos princípios e práticas do Dhamma e aconselhamos
o leitor a cultivar paciência e concentração na sua leitura, pois as palavras
escolhidas são por vezes as menos esperadas, de forma a `não passarmos por cima'
e a nos determos no significado mais profundo daquilo que está a ser
transmitido. Frases por vezes longas requerem uma leitura cuidadosa, sendo até
por vezes necessário repetir a leitura para que se possa absorver o significado
em toda a sua profundidade e plenitude. Mas vale a pena, pois a riqueza e
aprimoramento deste texto pode deixar uma marca indelével no praticante, com
resultados inequívocos para a sua investigação, que poderão vir a constituir um
farol na sua prática e investigação, para o resto da sua vida.

As sugestões de meditação apresentadas constituem também um forte método de
realização interior a ser colocado em prática.

Deixamos assim aqui esta nota de agradecimento e encorajamento.

Que seja cumprido o propósito do Dhamma.
