\chapterNote{Libertar-se dos `eus'}

\chapter{O Kamma dos Relacionamentos}

\tocChapterNote{Libertar-se dos `eus'}

\begin{quote}
  ``\ldots{} ele não cogita o seu mal, nem o mal alheio, nem o mal de ambos; e
  ele não experiencia na sua própria mente sofrimento nem mágoa. Desta
  forma\ldots{} o Nibbāna é diretamente visível, imediato, convidando"-nos a vir
  ver, investigar, digno de aplicação, a ser experienciado individualmente por
  aquele que é sábio.''

  \quoteRef{\href{https://suttacentral.net/an3.55/en/bodhi}{AN 3.55}}
\end{quote}

Mesmo quando a meditação que faz sozinho corre muito bem, pode reparar que os
contactos sociais o agitam. As opiniões sobre os outros, preocupações, atracção,
irritação: como limpar tudo isto? Como é que trabalhamos com o kamma dos
relacionamentos?

\subsection{A comunidade de valor}

O relacionamento é uma parte significativa da vida e a interacção faz parte das
famílias, das amizades e das comunidades, bem como dos mosteiros. No mosteiro
onde vivo, até mesmo um retiro é um misto de solidão e de comunhão. Encorajamos
a contenção e a introspeção ao passarmos dias em conjunto uns com os outros em
silêncio, alternando períodos de meditação na Sala com outros de meditação a
andar. Existe igualmente o tema da cooperação: realizamos tarefas em conjunto
todas as manhãs durante cerca de uma hora: varremos, limpamos as casas de banho,
os chuveiros, etc. Ocorrem palestras e sessões de diálogo. E existe igualmente o
espaço para sermos indivíduos: depois de algumas semanas de prática conjunta, um
determinado número de pessoas fica em solidão durante um período de tempo,
enquanto as restantes permanecem em grupo. Entretanto, pessoas exteriores ao
mosteiro trazem comida: algumas preparam a comida e oferecem"-na aos monges, às
monjas e à restante comunidade. Algumas pessoas chegam para apoiar de outras
formas, durante estes meses de retiro, pois respeitam este empenho e querem
ajudar"-nos a realizar o trabalho da meditação. Assim, embora muitos de nós
estejamos por nossa conta com as nossas sensações físicas, energias e estados
mentais, de outras formas estamos todos juntos nesta actividade. Aquilo que nos
sustem enquanto indivíduos é a prática da amizade espiritual.

\enlargethispage{\baselineskip}

A vida constitui uma situação partilhada: estamos neste planeta com outras
pessoas. Enquanto praticantes de meditação fazemos igualmente parte de uma busca
de significado e de verdade que tem sido levada a cabo por milhões de pessoas
através dos tempos. No decurso desta demanda surgiram grandes professores e
ensinamentos -- encontramo"-nos inseridos nessa comunidade de propósito e de
Despertar. Trata"-se de uma comunidade de valor (puñña): mais do que um simples
grupo de pessoas, constitui um `campo' de acções adequadas e de resultados
adequados que as pessoas têm cultivado. É importante que nos vejamos a nós e aos
outros nestes termos -- se vemos as nossas vidas apenas em termos da nossa
história pessoal e dos nossos amigos próximos, ou da nossa nacionalidade ou do
nosso género, temos uma concepção de nós próprios definida sobre termos que
causam conflito com aqueles que se encontram fora do nosso grupo.

A forma adequada de nos considerarmos uns aos outros é pensar em como, ao invés
de complicarmos a vida uns dos outros, podemos trabalhar em conjunto no sentido
do Despertar. Assim, recordamos o nosso potencial para o bom kamma, ao termos em
consideração aquilo que é meritório e bom em cada um de nós. Depois tentamos
viver de acordo com esse potencial, em termos da forma como cada um de nós age e
interage. Isto é a consciência moral (\emph{hirī}): vermos e respeitarmos o bem
em nós próprios e sermos veementes em viver de acordo com isso. E vermos e
respeitarmos o bem nos outros e sermos veementes em viver de acordo com isso
constitui o cuidado em sermos correctos (\emph{ottappa}). Esta consciência e
este cuidado são chamados `os guardiões do mundo' -- desde que escutemos o seu
conselho, o nosso mundo pessoal está alinhado com a integridade e a empatia que
apoiam o Despertar.

Contudo, se negligenciarmos a valorização das nossas acções e das acções dos
outros, podemos perder o contacto com essa consciência e com esse cuidado.
Talvez simplesmente esperamos que as outras pessoas se tornem ideais e,
consequentemente, ficamos negativos e críticos em relação às suas imperfeições.

Ou podemos imaginar que todos os outros são iluminados, ou próximo disso, e que
nós somos os atrasados do grupo, com falhas e necessidades peculiares. Tudo isto
constitui kamma mental negativo: a mente adoptou uma visão do `eu', ao invés de
ter reconhecido certas características positivas directamente. Claro, que
existem imperfeições e áreas que ainda não desenvolvemos, mas apenas depois de
reconhecer o bem é que nos encontramos preparados para apontar as imperfeições.
É a empatia desta consciência e deste cuidado, e não o julgamento, que os torna
os guardiões do mundo.

\subsection{Proliferação e opiniões}

Estabelecer julgamentos e adoptar opiniões fixas acerca dos outros obscurece a
consciência relativamente à especificidade do kamma de cada um e do facto de
todos nos termos de desenvolver de formas que resultem para nós. As opiniões dão
lugar ao kamma mental, e este, para o bem ou para o mal, não é um assunto de
pouca monta. Se não lidamos conscientemente com o kamma mental, este ocorre por
defeito,e vai buscar as suas directivas à tendência mental que for dominante
nesse momento. A tendência para a má vontade pode tomar as rédeas, de forma que
nos tornamos juízes intolerantes -- que fulana não ajuda muito, que fulano está
sempre a cabecear quando está a meditar\ldots{} Permanecermos naquilo que nos
desagrada, e proliferar com base nesta tendência reduz a pessoa a um ou dois
traços de personalidade, amontoando o sofrimento nas nossas mentes, sendo que
isso resulta numa perda ao nível do coração, o que incapacita seriamente a nossa
competência para praticar.

De modo que existe a necessidade de vigiar a tendência para formar opiniões e
para proliferar (\emph{papañca}).\pagenote{O papel da proliferação, enquanto
  origem de sofrimento, é referido em
  \href{https://suttacentral.net/mn18/en/sujato}{MN 18} e também em
  \href{https://suttacentral.net/dn21/en/sujato}{DN~21.2.2}. Constitui o tema
  principal de \emph{Concept and Reality in Early Buddhist Thought} do Venerável
  Nyanananda (\emph{Buddhist Publication Society}).}
Esta tendência cresce a partir da sede psicológica que quer solidificar a mente
num `eu' -- quer no nosso, quer no dos outros. Ou seja, concebemo"-nos como
tendo determinada característica ou estado e fazemos o mesmo relativamente aos
outros. Depois partimos do princípio que somos melhores, piores ou semelhantes a
eles. O ponto de referência é enviesado: uma característica não é uma pessoa. As
características, boas ou más, dependem do kamma e podem mudar. Contudo, devido à
sede psicológica de formar um ser sólido contínuo, a mente formula `eu próprio',
`tu próprio' e `aqueles outros' a partir do fluxo mutável de energias,
comportamento e aparência. Qualquer auto"-imagem, seja ela qual for, tem
necessidade de se basear em terreno sólido, em algum ponto de vista ideológico
ou disposição ou contexto fixos -- gosto que as coisas sejam feitas `à minha
maneira' porque sei como funcionar dentro desses parâmetros. Mas seguir `a minha
maneira' não me vai fazer sair dos meus hábitos e do meu programa de kamma.
Assim, partilhar uma situação oferece possibilidades para olhar para o apego à
`minha maneira' e para a largar.

Contudo, isso constitui um desafio: a nossa sensação de normalidade é perturbada
quando as nossas próprias rotinas e os nossos pressupostos individuais são
postos em causa. Num retiro de silêncio, particularmente em comunidades
compostas por várias nacionalidades como os mosteiros, não conseguimos estar
sempre a par dos detalhes do que está a acontecer. Não obstante, a prática deve
seguir os valores fundamentais, tais como os preceitos, a amizade e o
desenvolvimento da confiança. Isto porque, a não ser que tenhamos confiança em
nós próprios e nos outros, não conseguimos ultrapassar o nosso apego à certeza e
a termos as coisas sob controlo. Então formamos opiniões sobre os outros e sobre
como as coisas deviam ser, ao invés de investigarmos os nossos pressupostos e as
nossas atitudes relativamente ao significado daquilo que achamos que `devia
ser'. Aí, algo em nós contrai e fecha e, à medida que começamos a agarrar"-nos à
`minha maneira', sentimo"-nos alienados\ldots{} e, nessa desorientação, o
reflexo da proliferação começa a funcionar, de forma rápida e densa: `Porque é
que isto é assim? Devia ser assado!' 

O resultado final consiste na perda da
nossa sensação de fluir e do nosso contacto com o Dhamma. Começamos a julgar os
outros\ldots{} e talvez a criticarmo"-nos igualmente por o fazer. O kamma da
proliferação começa a transformar o momento presente em imagens bizarras da
forma como as coisas sempre são, foram, serão ou deverão ser. Constitui um
programa de agitação emocional, expectativa, anseio, frustração e desespero.
Sendo que nos afasta do encontrar, penetrar e abandonar o apego e o sofrimento
da mente.

O culminar do processo de proliferação traduz"-se na opinião fixa sobre nós
próprios (uma noção fixa de nós próprios), o que constitui o maior obstáculo ao
Despertar. Esta opinião nem sempre sobressai: ninguém anda por aí a dizer ou a
pensar `meu\ldots{} eu.. isto sou eu próprio'. Na realidade, com frequência
temos o oposto: como budistas pensamos `isto não sou eu, nem é meu\ldots{} mas
as coisas deviam ser desta maneira, isto é correcto.' Esta `maneira' é,
evidentemente, a psicologia, o sistema, o panorama global -- mas a noção ou
ideia de si próprio sustenta isso. 

Isto ocorre porque o modo como as coisas
deviam ser, ou parecem ser, qualifica a forma como eu me percepciono a mim
próprio -- enquanto alguém que está em contacto com a verdade ou que pertence à
equipa vencedora. Se eu sustento esta `maneira', que é a maneira `correcta',
então passo a ser um elemento apreciado do grupo. Posso mesmo ganhar respeito ao
sacrificar o meu sentido de mim próprio aparente em prol dos ideais que
projectei no grupo. Esta projecção proporciona à sensação do `eu' subjacente
alguns valores para que se possa sentir sólido -- o que pode ter a sua
utilidade. Mas podemos apegar"-nos a essa opinião, e quando as outras pessoas
têm uma outra `maneira', podemos sentir"-nos ofendidos. Mais uma vez dá"-se uma
perda de equilíbrio e, com isso, surgem várias formas de presunção: `Porque é
que não praticam todos da forma como eu pratico?' `Como é que ela é tão
descontraída e acha tudo tão fácil? Afinal a prática é algo intenso e a vida é
sofrimento, caramba!' Pois, antes de julgarmos quem está certo e por aí adiante,
podemos aprender como as pessoas nos mostram a tendência cármica de projectar e
proliferar.

Desta forma, parte da prática consiste em ver através das reacções,
proliferações e opiniões que surgem no âmbito da maneira como sentimos,
adoramos, avaliamos ou irritamos os outros. Aprendemos aquilo que realmente
temos de ter em conta: as mensagens do meu kamma encontram"-se dentro da maneira
como sinto e avalio os outros e a mim próprio. Pode existir insegurança,
sobrevalorização ou impaciência; pode haver idolatria ou ideais nobres de uma
vida global em harmonia. Mas maus ou bons, sempre que estes se tornam opiniões
fixas sobre a `minha maneira' e as `outras pessoas', vão existir problemas. De
modo que é necessário investigar isto, para vermos como as opiniões são criadas
e como modelam um kamma adicional. Isto é importante, uma vez que à volta disto
são gerados preconceitos, conflitos, cisões e até mesmo guerras.

Os pontos de vista têm o seu valor pois proporcionam"-nos um resumo da nossa
experiência, de forma a podermos retê"-la para uma referência futura fácil. Com
um ponto de vista ou opinião temos um ponto de partida. Contudo, o problema é
que os pontos de partida oferecem uma falsa sensação de solidez. As pessoas
podem adoptar uma qualquer opinião religiosa, política ou mesmo nutricional,
apenas porque oferece uma base sólida a partir da qual podem ter uma opinião
sobre si próprios e sobre os outros. De seguida, a actividade emocional e o
kamma mental são gerados em consonância.

Por exemplo, ocasionalmente temos pessoas no mosteiro que são muito diligentes
na sala de meditação\ldots{} mas é difícil trabalhar com elas na cozinha porque
as coisas têm de ser feitas à sua maneira. Isso não é correcto, não é verdade?
Contudo, em geral, as suas acções baseiam"-se naquilo que acham ser a forma mais
eficaz e eficiente de funcionar, de modo a providenciar alimentação para a
comunidade. E isso parece ser correcto\ldots{} Ou alguém fala nos momentos de
silêncio\ldots{} O que é errado! Mas talvez essa pessoa tenha sentido que alguém
precisava de um pouco de contacto, ou que um pouco de descontracção seria um bom
remédio\ldots{} A acção baseada na compaixão e no desapego às rotinas parece um
ponto de vista sábio, não é? Alguém quer sentar"-se quando é altura para
meditação a andar, ou andar quando é meditação sentada\ldots{} Talvez isso seja
o mais acertado para a sua prática. Mas podemos sentir: `Bem, tínhamos um acordo
para agirmos de certa forma, no sentido de fortalecermos a determinação do grupo
e minimizar as perturbações, e é esperado que as pessoas larguem as suas
perspetivas pessoais.' Isso também está certo! `Certo' carrega uma energia muito
poderosa, não é verdade? Podemos ficar verdadeiramente convencidos e
verdadeiramente zangados com o `certo'?

Agora, não estou a dizer que as questões do comportamento não devem ser tidas em
conta: esse constitui um dos valores da amizade espiritual. Mas esses valores
operam através de uma compreensão do kamma, bem como de que o kamma deve ser
investigado a partir de um ponto de vista de compaixão e de equanimidade. Este é
o ponto de vista correcto: exclui a proliferação relativa a um excerto
específico de comportamento, não o tornando numa opinião que `ele é assim' e que
`ela é daquelas que\ldots{}'. Se formamos um `eu' com base em qualquer padrão de
comportamento, as nossas atitudes ficam emperradas e tornam"-se dolorosas. Mesmo
um bom `eu' acaba por nos intimidar ou desiludir quando a pessoa não está à
altura da imagem que criámos dela. De modo que o único ponto de vista para o
`eu' que é útil ter presente é o seguinte: `eu sou dono do meu kamma\ldots{}
seja qual for o kamma que eu criar, para o bem ou para o mal, dele serei
herdeiro'.\pagenote{Este refrão constitui uma das recordações diárias
  recomendadas aos budistas praticantes. O texto completo inclui: ``Sou o dono
  do meu kamma, herdeiro do meu kamma, nascido do meu kamma, ligado ao meu
  kamma, permaneço suportado pelo meu kamma; seja qual for o kamma que eu criar,
  para o bem ou para o mal, disso serei o herdeiro.''
  \href{https://suttacentral.net/an5.57/en/bodhi}{AN 5.57}}

Não existem pessoas boas ou más, existe apenas kamma luminoso e escuro. Se
virmos a vida desta forma, isso ajuda"-nos a encararmo"-nos a nós e aos outros
de uma forma mais compreensiva e compassiva. Não ficamos emperrados em
autoimagens, uma vez que o ponto de vista do kamma também nos permite
compreender que o kamma constitui um processo em mudança. Se formulamos
opiniões, atitudes e reações que fixam as pessoas no seu kamma, então estamos a
apoiar o sofrimento. Se estabelecemos as opiniões, as atitudes e as reações que
permitem a mudança das intenções e dos valores num bom sentido, estamos a
relacionar"-nos de forma adequada com o kamma. Pode existir uma compreensão dos
comportamentos, confusos ou luminosos, que nos movem a todos, ao invés de
opiniões que me separam do outro e daquelas `outras pessoas'. Deste modo
estabelecemos as possibilidades para a nossa própria libertação e para a
libertação dos outros.

Quando percebemos esta mensagem, começamos a mudar a intenção da nossa prática:
em vez de querermos ter ou ser algo, passamos a lidar e a penetrar no sofrimento
envolvido na sensação de `eu'. Então ocorre uma libertação, que suscita
igualmente o nosso potencial de sabedoria, pureza e compaixão. E este é o
objectivo da prática do Dhamma, quer estejamos sós ou acompanhados,
independentemente do que está a ocorrer.

\subsection{Kamma: o campo relacional}

A metáfora do kamma enquanto campo:

\begin{quote}
  ``Ananda, se não existisse qualquer acção a amadurecer no domínio da
  sensualidade, será que conheceríamos o devir no domínio da sensualidade?

  Claro que não, senhor.

  Dessa forma, Ananda, a acção constitui um campo, a consciência uma semente e o
  anseio a humidade para que a consciência dos seres obstruídos pela ignorância
  e aprisionados pelo anseio, seja estabelecida a um nível mais baixo (ou de
  forma subtil, ou sem forma). De modo que novos devires e regressos são
  efectuados no futuro. Assim, Ananda, existe o devir.''

  \href{https://suttacentral.net/an3.76/en/thanissaro}{AN 3.76}
\end{quote}

O kamma é potente e astuto. Envolve as energias criativas e interpretativas que
são condicionadas pela história individual de cada pessoa. Aí encontra"-se o
resultado, o \emph{vipāka}, de termos nascido. Experienciamo"-nos a nós próprios
como tendo uma existência nos campos de sensações, disposições, atitudes e
informações dos sentidos que se sobrepõem e nos dizem como estamos e em que
ponto. Discernimos e adquirimos significado e aprendemos a partir do contexto
onde vivemos: do planeta, da sociedade, da família, bem como dos nossos próprios
corpos. E estes campos físicos e psicológicos dependem das consciências que
operam através dos cinco sentidos externos e da mente. Através dos sentidos
externos estamos continuamente a ser definidos como `estou aqui a receber esta
impressão'. A consciência da mente adiciona mais detalhes: através disto
sentimo"-nos definidos pela forma como os outros nos vêem e se relacionam
connosco; e, de forma ainda mais contínua, pelo modo como nos encaramos a nós
próprios. Como estamos ou como achamos que os outros pensam que estamos é
definido pelo \emph{feedback} do prazer, da dor, da recompensa e da culpa. A
identidade é programada, com uma autodefinição formada pelos padrões mais
entranhados de elogio e de culpa, valor, autoestima ou negligência. Assim, a
forma como temos sido (e somos) afectados consolida"-se naquilo que somos.

O nosso sentido fundamental de ser algo baseia"-se assim em padrões que surgem a
partir de nos encontrarmos inseridos em algo: um ventre, uma família, uma nação,
uma ordem mundial, e por aí adiante. Esta é a acção da consciência. A partir
desta base, o sentido mais pessoal do `eu' desenvolve"-se através do
\emph{feedback}, à medida que a minha aparência e as minhas acções são avaliadas
e gravadas. A partir deste aglomerado do que sentimos e da forma como o
interpretamos, surge o sentido de `eu sou isto', `eu estou nisto' ou mesmo `eu
quero ser diferente disto'. Deste modo, os padrões e os programas das nossas
mentes são estabelecidos através da relação. Se aquilo no qual (re)nascemos nos
oferece mensagens de boas"-vindas e de confiança, então os nossos padrões e
programas formam"-se com uma base de confiança básica em aqui estar. Mas se é o
contrário, se existiram mal"-entendidos, se nos alimentaram enviesamentos,
exageros e falsidades\ldots{} ou se a família ou sociedade nos transmite a
mensagem de que somos uma maçada ou um fardo, ou que temos de ser produtivos,
inteligentes e atraentes -- então, mesmo que adquiramos estas qualidades,
fazemo"-lo com base na ansiedade. Com efeito, podemos mesmo conseguir ser muito
inteligentes e vigorosos, de modo a dominar a sensação subjacente que, para
sermos bem"-vindos aqui, temos de dar provas do nosso valor. Podemos ter
desenvolvido uma incrível força de vontade de forma a nos sentirmos confiantes
no nosso sucesso e na nossa independência, e não precisarmos de ajuda. Mas uma
noção de `eu' que proclama a sua completa autonomia é um indicador de uma
vontade doente e de uma disfuncionalidade relacional. A história encontra"-se
cheia de génios brilhantes mas neuróticos, de Messias doidos e de psicopatas com
poderes mentais formidáveis.

Se não possuímos o sentimento de pertença a uma família boa e benevolente, ou a
uma sociedade que defende valores adequados, tais como a honestidade, a bondade
e a generosidade, o sentimento básico face ao contexto onde nos inserimos será
de falta de confiança. Assim, não estaremos inseridos numa comunidade de valor e
a nossa própria benevolência pode não ser valorizada. Neste tipo de cenário
temos de encontrar o nosso valor através daquilo que alcançamos e do nosso
\emph{know-how}. De forma que a base relacional torna"-se fortemente marcada
pela individualização -- `faz e alcança por ti próprio' -- com sentimentos
fracos no sentido da partilha, da cooperação e integração com os outros.

Ainda é mais grave se não conseguimos obter valor através dos nossos esforços
individuais: experienciamo"-nos como inúteis. E se quem julga o nosso valor
próprio é a nossa própria psicologia motivada pelo desempenho, nunca conseguimos
vencer: conseguimos sempre imaginar um estado melhor ou mais elevado no qual nos
podemos tornar. Esta perda do sentido de valor, ou sensação de ser dirigido,
pode ter como resultado o desespero existencial e os esgotamentos, a depressão,
o consumo de substâncias tóxicas e mesmo o suicídio. Infelizmente, isto acontece
com frequência nas sociedades ocidentais, nas quais existe uma tónica
considerável na realização individual, a par de um fraco sentido de algo ao qual
pertencemos naturalmente, sem ser necessário esforço de vontade própria.

Dada a natureza pouco fiável das relações sociais, o sentimento de pertença que
pode oferecer maior confiança é relativamente a um cosmos abrangente, um campo
de valor, de verdade ou sagrado. Esta constitui a referência mais importante --
não apenas a pertença a um grupo independentemente daquilo que este faz; nem a
um estado"-nação, independentemente da agenda dos seus líderes; nem a um culto
que segue um líder carismático. Se ingressarmos numa equipa como um elemento que
nada questiona, estabelecemos uma relação de dependência infantil, que nos pode
oprimir, subjugar e fazer emburrecer. Assim, é vital sentir e ser capaz de
aspirar a algo sagrado, que se encontra liberto de contaminações e disponível
graciosamente, para qualquer pessoa sábia, através do seu próprio cultivo. Em
termos budistas, este constitui um significado essencial do termo `Dhamma'.

O relacionamento consciente, de acordo com o Dhamma, é consequentemente crucial.
Uma vez que a nossa herança cármica é estarmos em relacionamento -- com o
planeta e com as pessoas com as quais o partilhamos -- a prática do Dhamma tem
de incluir uma sensação de relacionamento, de `viver com' que tenha significado
e propósito. Isto quer dizer aprender a interagir em situações da vida real
frequentemente confusas e enleadas, através da bondade básica e do respeito,
independentemente de quais e a quem pertencem os programas que estão a decorrer
no momento. Alteramos as regras relativas a estarmos inseridos numa relação
estrita dos vencedores e dos vencidos, do mais elevado e do mais baixo, para as
regras associadas ao desapego. Relacionamo"-nos porque estamos aqui e vamos
fazê"-lo de forma correcta. E essa mudança no sentido da integridade relacional
realinha as nossas tendências cármicas de formas valiosas e valorizadoras. Isto
porque, quando a nossa intencionalidade se baseia em nos estabelecermos em algo
que é justo, mútuo e solidário, não tem como objectivo o controlo ou o
estabelecimento de uma posição no seu âmbito. Não se trata de ser o melhor do
grupo, mas sim de estar envolvido nele de forma responsável. Dessa forma podemos
fruir do nosso grupo social, em vez de termos uma intenção baseada
compulsivamente em provar o nosso valor ou em negarmos a relação. Fazemos parte
dele com o objectivo de gerar o bem, não para ganhar troféus.

Lembro"-me de ler um relato de um jogo praticado por uma tribo da bacia do
Amazonas. O trabalhador rural britânico que estava a observar o jogo, ao
princípio não estava a conseguir compreender as regras. Notou que os jogadores
se dividiam em duas equipas, que não tinham necessariamente uma equivalência em
termos de número de elementos ou de força aparente. Cada equipa agarrava um
tronco grande de árvore e, carregando-o em ombros, começava a correr no sentido
de um ponto que se situava aproximadamente a cem metros. Os troncos também não
eram exactamente do mesmo peso ou tamanho. À medida que esse trabalhador
observava, uma equipa ultrapassava a outra e, quando o fazia, um elemento da
equipa da frente deixava-a e juntava"-se à outra equipa. Fosse qual fosse a
equipa da frente, saiam elementos desta para se reunirem à equipa que estava a
perder. À medida que a linha do final se aproximava, o entusiasmo aumentava até
que as equipas atravessavam a linha, frequentemente com escassa distância entre
elas. Por fim, o trabalhador rural descobriu o objectivo da corrida: era que as
duas equipas atravessassem a linha em simultâneo! Ambas as equipas tentavam
atingir esse objetivo através de uma grande atenção e de um esforço extenuante,
mas com uma intenção prevalecente de chegar a um sítio sem vencedores e sem
vencidos. Não é uma má analogia para as qualidades que apoiam a amizade
espiritual.\pagenote{Ver: \emph{Millenium}; David Maybury-Lewis. (Viking Penguin
  1992) O Buddha considerou a amizade espiritual como um suporte essencial para
  o Despertar, como na famosa citação: ``a amizade espiritual é a totalidade da
  vida santa.'' \href{https://suttacentral.net/sn45.2/en/bodhi}{SN 45.2}

  E também:

  ``Se aqueles que vagueiam e são elementos de outras seitas te perguntarem,
  ``Amigo, quais são os pré"-requisitos para o desenvolvimento que apoia o
  Despertar?'' deves responder, ``É a situação na qual um monge tem pessoas
  admiráveis como amigos, companheiros e colegas. Este é o primeiro
  pré"-requisito para o desenvolvimento que apoia o Despertar.''
  \href{https://suttacentral.net/an9.1/en/thanissaro}{AN 9.1}

  Consultar também: \href{https://suttacentral.net/an9.3/en/sujato}{AN 9.3} e
  \href{https://suttacentral.net/an7.54/en/thanissaro}{AN 7.54} para detalhes.}

A amizade espiritual constitui um trampolim que nos relembra destes valores e,
em determinado grau, os modela. Mas em última análise, a amizade espiritual não
é a criação de laços com um indivíduo ou um grupo -- trata"-se de cultivar um
relacionamento que proporciona, de forma constante, valores de moral, compaixão
e investigação da mente. Seguidamente leva"-nos à fundação mais firme do
relacionamento, a de conectar as nossas acções a um campo de valor.

\subsection{Relacionar-me com os outros tal como comigo próprio}

O kamma antigo é algo emaranhado e viscoso: é compulsivo e puxa"-nos. Quando
estamos em conflito, o nosso hábito é ver a outra pessoa através de um filtro
negro. Acontece igualmente que os nossos desejos e as nossas necessidades
apresentam as outras pessoas de acordo com um filtro cor"-de-rosa. Por vezes a
nossa impressão relativa a outra pessoa pode alterar"-se numa questão de horas!
Em qualquer dos casos, negro ou rosa, centramo"-nos num detalhe específico, e
uma sequência rápida de impressões, disposições e pensamentos ocorrem e isso
transforma"-se numa impressão global e, no final, numa afirmação de `eu sou' ou
`ele/ela é'. Isso é `proliferação'. É convincente porque surge a partir de uma
fonte aparentemente profunda e involuntária, e porque é familiar. Constitui uma
forma do processo de `devir' ou de formar um `eu', sendo que cria alguma
solidez, alguns pontos de vista, algumas marionetas com as quais nos podemos
ocupar. Mas a miragem do `devir' na realidade, priva"-nos de uma presença plena.
Repare, na próxima vez que a proliferação e a projecção tomarem o controlo, o
quão a noção de continuidade da sua presença física é perdida. Onde é que o
leitor está? Essa sensação de perda de presença é a marca da ignorância sobre a
qual se baseia o `devir'. Com ignorância perdemos acesso à clareza, à qualidade
de espaço e à empatia.

Por outro lado, por vezes deparamo"-nos nas nossas vidas com situações limite --
momentos nos quais é necessário tomar decisões, cenários nos quais somos
desafiados, interacções nas quais as diferenças de pontos de vista podem levar a
conflitos -- e não sabemos o que fazer. Por vezes, pode existir uma sensação
irritante de sermos alguém que necessita de encontrar uma resposta na vida.
Talvez exista um conflito entre a forma como sinto que sou e como gostaria de
ser. Estes são os grandes cenários nos quais sentimos dúvida e vacilamos, e que
podem manter"-nos ocupados ou frustrados durante anos. É difícil lidar com a
incerteza -- se não conseguimos chegar rapidamente a uma conclusão, podemos
deixar o assunto e ficar desatentos ou endurecermo"-nos numa opinião
preconcebida. Seja como for, muitas vezes tentamos barricar a incerteza, em vez
de lidarmos com ela e a libertarmos. Isto constitui ignorância -- um `não nos
tornarmos' proveniente de um reflexo de retracção. Trata"-se do impulso para ser
um `eu' afastado de um acontecimento ou de uma situação, e faz com que nos
`percamos a nós próprios' na perca de poder e na abdicação. As proliferações
baseadas nesta tendência cristalizam"-se em crenças relativas à vida de
resignação, medo ou inadequação: `não consigo lidar com as coisas, isto é tudo
demais para mim, sou pequeno e ineficaz'. E a generalização: `a existência é
apenas dor, tudo é uma perda de tempo', fica entranhada. Existe um emperramento
e isso é interpretado como `estou encravado' num `mundo sem significado'.

Temos necessidade de desenvolver as qualidades de amizade espiritual
relativamente a nós próprios. Mas relativamente a quê e como é que eu desenvolvo
essa amizade? Que parcela disso constitui uma fantasia baseada na necessidade de
ter uma auto"-imagem positiva? Bem, não está em causa uma imagem mas sim um
relacionamento viável com os nossos hábitos, estados de espírito e programas. Se
isto existir, podemos revê"-los ao invés de lhes reagir. O conselho habitual que
é dado aos monges pelos seus professores é `fazer um esforço amigável'. Esse é o
ponto. Quando temos de avançar a partir de um estado de desequilíbrio, temos de
aprender a confiar e valorizar a nossa capacidade para atingir o equilíbrio.
Trata"-se de um acto de fé, de ter fé em nós próprios, para além de qualquer
autoimagem. Isto porque, quando existe conflito a nível interno, a referência a
nós próprios surge com o pressuposto que algo está errado `comigo' e que tenho
de fazer algo para ser diferente daquilo que sou.

Contudo, limpar este padrão não ocorre apenas por afirmarmos que não temos nada
de errado. Isso é apenas outra auto"-imagem. O que restabelece o equilíbrio é
suspender o pressuposto `eu sou', colocar em pausa por um tempo a nossa
autodefinição e depois cuidar das características e das energias que podem
acalmar os padrões mentais subjacentes ao nosso aparente `eu'.

Para isto, sintonizamo"-nos com a experiência física de estarmos aqui sentados
ou de pé. Isto fornece"-nos uma âncora. O corpo não pode proliferar;
simplesmente indica"-nos onde estamos, sem julgamentos, análises ou
alternativas. Seguidamente existe a sensação do coração, no sentido de
estabelecer boa vontade relativamente a nós próprios. Contudo, tanto o corpo
como a sensação do coração, também têm a necessidade de ser específicos. E para
isto temos a atenção racional que se pode concentrar naquilo que acalma o corpo
e eleva o coração. Mais uma vez, trata"-se de coisas simples como a sintonia com
a postura ou lembrarmo"-nos das qualidades de Buddha e das amizades espirituais
-- o ponto principal é fazê"-lo, e confiar nessa simplicidade face às nossas
poderosas e fascinantes proliferações. E fazer isto em vez de nos tentarmos
decifrar ou de criarmos um grande psicodrama baseado na nossa história pessoal.

De modo que treinamos a mente racional para ser uma testemunha das nossas
tendências cármicas, ao invés de uma analista ou contadora de histórias.
Simplesmente testemunhamos o aspecto da nossa experiência ao nível do corpo, que
não pode criar histórias como o coração. Seguidamente, podemos unir o corpo e o
coração de forma solidária, ao reconhecer que, neste momento, o corpo se
encontra livre de ameaças, intrusões ou obstruções; depois encorajamos o coração
a realmente experienciar isso. Talvez exista desconforto, talvez exista tensão
no rosto, nas têmporas, no peito e na barriga. Mas algures na sensação física --
no final de uma inspiração, ou na pressão das plantas dos pés, existe uma
referência ao sentido de bem"-estar.

Se nos sintonizarmos com essa sensação física de à-vontade, isso levar"-nos-á ao
equilíbrio das nossas mentes, sendo que é apenas a partir desta base que podemos
ter um vislumbre (através do emaranhado de ansiedades e alterações de
disposição) de um simples fio de conforto emocional e espaço psicológico. Isto
significa uma mudança de um estado de tensão para um estado de maior confiança.
E é através deste cultivo que podemos distanciar"-nos dos enviesamentos e das
narrativas antigas. A cura é um resultado natural de encontrarmos o verdadeiro
equilíbrio.

À medida que encontramos equilíbrio e que a energia aquieta, podemos estender a
qualidade dessa confiança e intenção benevolente a todos os tecidos e estruturas
do corpo. Seguidamente estender essa qualidade ao espaço que nos rodeia: `Que
tudo isto possa estar liberto de ameaça ou tensão'. Podemos depois, de forma
mais específica, abranger as impressões relativas às outras pessoas,
especialmente àquelas com grande significado para nós, tanto bom como mau: quer
amigos, quer pessoas com as quais temos dificuldades. Desta forma, partilhamos
valor, perdoamos, apreciamos e empatizamos.

É favorável sermos directos e específicos. A prática não pode basear"-se numa
imagem `daquilo que eu sou e do que os outros deviam ser'. Ao invés, reconhece
os acontecimentos específicos e as tonalidades das disposições da nossa mente.
`Neste momento a sensação é esta, a impressão é esta\ldots{}' Na realidade é
apenas esta agitação, ou esta dor, ou este desejo, e é semelhante ao de todas as
outras pessoas. Nesse momento, podemos aceitar isso e não criar um `eu' baseado
nisso. E o mesmo acontece com o comportamento dos outros quando este altera as
nossas mentes: `A sensação é esta, a impressão é esta, não preciso de agarrar as
suas acções, que eles possam ser libertos delas.' Em vez de mergulharmos a nós
ou aos outros no kamma, aprendemos a perdoar ou a reconhecer com gratidão, mas
sempre com o espaço necessário para haver uma libertação relativa às projecções.
Isto constitui a equanimidade, que é o necessário para conseguirmos partilhar a
nossa prática com os outros.

\subsection{Encontro com o bom amigo}

Através do trabalho sobre as nossas próprias mentes, aprendemos que apenas
ultrapassamos as dificuldades a partir de uma relação baseada nos factores do
Despertar, tais como a consciência, a investigação, a concentração e a
equanimidade. Então a mente começa a apreciar a clareza e o espaço que estes
factores proporcionam -- pode descansar. Assim, sente"-se mais centrada neste
espaço estável do que em qualquer um dos estados que podem surgir,
sintonizando"-se com as tendências que o favorecem. Este processo é semelhante
ao encontro com um amigo de longa data -- reconhecemo"-nos e abrimo"-nos
mutuamente, de forma natural. Consequentemente, uma consciência equilibrada
entra em sintonia com essas tendências e esses resultados luminosos associados a
todas as ocasiões nas quais demonstrámos paciência ao invés de impaciência;
quando oferecemos atenciosidade, ao invés de indiferença ou negatividade; quando
a nossa persistência venceu o desejo de cortar e fugir; quando estivemos à
altura do desafio de estarmos presentes. E o kamma poderoso da renúncia
sintoniza"-nos com a fonte de todo a espaciosidade, com a intenção de largar, de
deixar ir.

A acção adequada que constitui um apoio para o grande coração: é isto que todo o
processo frustrante de contenção e cumprimento da norma de um grupo ajuda a
gerar. Posso sentir todo o desejo de saber, ter e conseguir algo mas, em vez de
acreditar nisso ou de seguir esses temas, posso sintonizar"-me com o que está a
acontecer à energia. Esse cintilar, essa contracção ou afundamento, constitui o
trilho do kamma -- e, na verdade, não queremos seguir e nos afundar
repetidamente nesse percurso\ldots{} Mas existe a via do não apego e da
renúncia, na qual a pureza da consciência e a pureza da acção apoiam"-se
mutuamente. É aqui que encontramos o nosso amigo mais verdadeiro -- numa
consciência que não transporta a marca do ego. Podemos confiar nela e abrir mão
de tudo o resto.

Uma consciência equilibrada e tranquila, como um bom amigo, também traz
sabedoria aos acontecimentos das nossas vidas. Por vezes podemos sentir"-nos
levados a expressar um pedido de desculpa ou gratidão; ou geramos kamma luminoso
a partir de uma presença plenamente estabelecida. Noutras ocasiões a resposta de
sabedoria traduz"-se apenas em permitir que as coisas se aquietem e em sentir a
paz ou equanimidade associadas a tal. O comportamento pessoal pode
\mbox{desenvolver"-se} e crescer através deste processo de relacionamento sábio com o
nosso kamma específico, tal como se manifesta nas nossas vidas. E estas formas
de comportamento, apesar de, no geral, serem calorosas, firmes, límpidas e
espaçosas, são específicas de cada praticante: todos os seres sábios que tenho
conhecido são personagens reais, com personalidades muito diferentes.
Ironicamente, tornamo"-nos pessoas mais autênticas quando largamos o ego. E a
razão para isto acontecer é que funcionamos com um potencial aumentado quando a
consciência não está aprisionada nas estratégias da auto"-imagem.

\clearpage

\section[Meditação: encontrar-se com o espaço]{Meditação}

{\centering
\subSectionFont\selectfont
\textit{Encontrar-se com o espaço: uma meditação em pé}
\par}

\bigskip

Fique em pé com os pés paralelos, afastados à largura do corpo. Entregue o peso
do seu corpo ao chão, através das plantas dos pés. Como o corpo está acostumado
a manter"-se levantado ou a inclinar"-se sobre algo, com frequência
`esquece"-se' de se apoiar nos pés -- então descontraia conscientemente os
joelhos, as nádegas e os ombros. Descontraia o maxilar e a zona em torno dos
olhos.

Cultive a sensação de que o sítio onde está de pé oferece"-lhe suporte e é
completamente seguro. Pode saber isto na sua cabeça, mas não no peito, na
garganta ou nos ombros. Por isso, gradualmente, verifique o corpo. Depois sinta
através da pele, ficando consciente de `tocar' o espaço. Permita que o corpo
sinta e reconheça plenamente que o espaço (primeiramente à frente, depois em
cima e seguidamente atrás) se encontra desobstruído e não é intrusivo.
Desenvolva esse tema. Por exemplo, interrogue"-se: `o que é que está atrás de
mim?' E depois reflicta: `atrás de mim está um apoio forte. Nada do qual me
tenha de proteger.'

Verifique a postura periodicamente, de forma a impedir que as nádegas, o peito,
os ombros e o abdómen fiquem tensos. Mantenha os joelhos suaves, deixando que o
chão suporte o peso do corpo. Deixe o corpo explorar esta sensação de ser
suportado pelo chão. Irá descontrair"-se, encontrar estabilidade e a respiração
vai"-se tornar mais ampla, com o seu ritmo a ajudar a receber e a libertar
qualquer tensão. Vai surgir uma sensação de preencher plenamente o espaço à sua
volta. Pode sentir"-se um pouco maior e mais `em casa'.

Permaneça com a sensação geral do corpo, sem perder a sensação de estar num
espaço e sem dar atenção a qualquer fenómeno externo em particular. Mantenha a
sua atenção onde a sensação do corpo entra em contacto com a sensação do espaço.
Provavelmente a mente vai querer ir para algum lado: para o corpo, para um
pensamento ou para uma atitude, ou para fora, para algum objecto visual. Vai
querer ter um propósito ou algo a que se agarrar. Pode existir uma luta para se
livrar de disposições e sentimentos. Contudo, mantenha"-se simplesmente centrado
na energia do corpo, ou nas disposições que surgem face à sensação de encontro
com o espaço que o rodeia. A energia do corpo pode ser experienciada como
correntes ascendentes ou como tremores; pode ser sentida no peito ou no abdómen.
Naturalmente, podem ocorrer estados emocionais correspondentes, tais como
entusiasmo ou nervosismo. Pode experienciar rubores de tensão que se libertam.
Sintonize"-se com o eixo vertical do corpo -- um fio imaginário que liga as
plantas dos pés, ao sacro, à coluna e para cima, através do pescoço e do alto da
cabeça. Estenda esse fio para baixo em direcção ao chão e para cima, através do
alto da cabeça, para o espaço acima de si. Deixe que o seu corpo seja uma conta
neste fio. Inspire e expire de forma a proporcionar uma sensação de estabilidade
e bem"-estar.

Não embarque em quaisquer estados físicos ou emocionais, mas mantenha"-se
consciente do conjunto do fio, do eixo de equilíbrio -- ou da sua maior parte,
tanto quanto lhe for possível. Dentro desta sensação alargada do corpo, permita
que as energias e as disposições se movimentem, à medida que, muito lentamente,
perscruta conscientemente no sentido descendente pela sua cabeça, pela garganta
e pela parte superior do tronco. Utilize a actividade de `trazer à mente' e
`avaliar': ou seja, pense em `testa' e seguidamente reflicta sobre a forma como
sente a testa em termos das características dos elementos. Sente-a como firme,
sólida ou apertada (terra)? Está quente ou fria (fogo)? Existem movimentos de
energia ou palpitações nessa área?

Pode detectar tensões subtis na zona dos olhos ou em redor da boca, ou na zona
da garganta e da parte superior do tronco. Se isso acontecer, centre"-se
novamente no eixo de equilíbrio e, lentamente, estenda a sua atenção ao longo da
área na qual se está a focar e ao espaço imediato que circunda o seu corpo.
Pratique o encontro com seja o que for que surja, sem se envolver nisso.
Contudo, se surgir uma sensação de aperto, emotividade ou agitação, conecte"-se
com o eixo de equilíbrio, suavizando e ampliando a sua atenção.

Desenvolva a sensação de ser visto nesse estado de abertura, de forma simples e
grata. Esteja simplesmente presente com isso e com a forma como está a sentir.
Dê"-se tempo para sentir e disfrutar da sensação de se encontrar num espaço
benevolente. Pode ser benéfico imaginar que se encontra na luz, numa temperatura
amena ou na água.

Continue esta prática, movendo a sua atenção lentamente pelo tronco, pelas
pernas, até ao chão. O/A leitor/a pode não ter vigor ou tempo para trabalhar
todo o corpo. Se for o caso, desloque"-se mais rapidamente e procure percorrer o
tronco ou, pelo menos, uma das seguintes áreas: garganta e parte superior do
peito; coração e centro do peito; plexo solar e zona média do tronco; parte
inferior do abdómen, por baixo do umbigo.

Pratique de acordo com a sua capacidade e quando sentir que deve concluir,
despenda algum tempo a limpar o espaço das imagens e impressões. Depois
concentre"-se novamente na pele, distinguindo as suas fronteiras à volta de todo
o corpo. Seguidamente, sem perder a sensação de espaço, sinta a sua coluna e o
centro do corpo dentro deste espaço delimitado. Conclua a meditação ao
reconhecer os sons à sua volta, o campo visual e depois os objectos específicos
que o rodeiam. Movimente"-se ligeiramente, orientando"-se através do tacto.
