\chapterNote{Autorreflexão}

\chapter{Existe um Fim?}

\tocChapterNote{Autorreflexão}

\begin{quote}
  ``\ldots{}com a destruição da ânsia vem a destruição do kamma; com a
  destruição do kamma vem a destruição do sofrimento.''

  \quoteRef{\href{https://suttacentral.net/sn46.26/en/bodhi}{SN 46.26}}
\end{quote}

Existem momentos durante os quais se interroga se a sua prática está a evoluir?
Por vezes distancia"-se dos detalhes mais refinados e reflecte: `Estou menos
ansioso e opinativo? Estou mais em paz comigo próprio e mais compassivo para com
os outros? Tudo isto tem"-me feito uma pessoa melhor?'

\subsection{Reconhecimento dos resultados}

Estas questões são razoáveis: estamos interessados nos resultados que podemos
atingir. E a prática do Dhamma tem alguns resultados imediatos: ficamos mais
conscientes dos impulsos e das sensações que correm nos nossos sistemas, bem
como com uma melhor noção sobre no que é que nos podemos apoiar para agir e o
que devemos colocar de lado. Começamos igualmente a estabelecer princípios --
aqueles que podem resistir às pressões da vida atarefada ou aos enviesamentos
dos meios de informação.

Noutros campos, temos de saber aquilo que procuramos. Em termos básicos,
procuramos e necessitamos duas coisas na vida: sentirmo"-nos bem e sentirmo"-nos
estáveis e seguros relativamente a onde nos situamos e para onde nos dirigimos.
Necessidades simples, aparentemente. Contudo, sem alguma orientação e prática, a
forma de suprir estas necessidades pode tornar"-se muito complexa. Pode
tornar"-se num exercício de andar na corda bamba, contrabalançando possíveis
ganhos e perdas no futuro com as variáveis do presente, um acto que nos empurra
contra obstáculos, contra contratempos e ao encontro de vastos espaços
desconhecidos\ldots{} De modo que, se apenas avaliamos como estamos em virtude
de acontecimentos diários, as conclusões não serão assim tão fidedignas. A mente
fica tensa por ter de lidar com muitas opções e ter de ter cuidado com
potenciais contratempos, sendo igualmente inundada por tudo o que lhe chega, e
sobre o qual temos pouco controlo: como as acções e o discurso dos outros, ou
aquilo que nos chega através dos meios de comunicação e da vida na cidade.
Depois existem os incómodos físicos, que podem fazer surgir disposições difíceis
e pouco inspiradoras. Para além disso, o processo contínuo da meditação expõe
alguns programas mentais pouco gratificantes: fantasias escapistas, crítica a
nós ou aos outros, ânsia por atenção e elogios, necessidade de termos controlo
sobre as situações -- tudo isto é bastante habitual. E juntamente com isto surge
a sensação de que isto é o que somos e que devemos ser de outra forma e não
desta. Assim, existe uma contracorrente de disposições que as pessoas
bem"-intencionadas atravessam quando se sintonizam com a prática do Dhamma.

Isto acontece porque estamos a despertar, sendo que este acordar do acidente da
vida torna"-nos conscientes das contusões. Acumulamos sofrimento resultante de
tudo pelo qual passámos, em termos de doenças e danos físicos, e ganhamos
algumas contusões à conta dos contratempos de relacionamentos confusos, das
acções inadequadas e do que na altura nos pareceu um bom divertimento, mas que
acabou por nos dar uma dor de cabeça. Sim, isto constitui parte da confusão em
torno das sensações: algumas acções são sentidas como aprazíveis na altura,
contudo deixam"-nos com uma ressaca, de um tipo ou de outro. Temos a necessidade
de nos sentirmos bem e com tudo sob controlo. Contudo, uma mente não treinada
cria formas inadequadas e pouco fiáveis para atingir estes objectivos. De forma
que ao despertarmos, temos necessidade de enfrentar os nossos emaranhados e as
nossas disfunções, com a vontade de retirar alguma compreensão a partir destes.
E isso significa tomar refúgio no Dhamma: na compreensão de que é através do
processo de enfrentar a tensão, o conflito e o incómodo que irá surgir algo mais
amplo e sábio. Temos de encontrar o valor e a confiança neste processo de
despertar e na realização do trabalho que o torna possível.

\subsection{Contemplar a intenção}

Em termos essenciais, trata"-se de encontrar valor na intenção. A intenção traz
consigo uma disposição: luminosa ou escura, firme ou errática. Pode ser suave ou
vigorosa. E estes estados e sentimentos que os acompanham constituem alimento
para o coração. De modo que o seu corpo pode não estar apto a atingir os padrões
olímpicos ou a ser um ícone de beleza. E talvez nunca vá ser Einstein ou Mozart.
Estados de humor, sensações, saúde e sucesso vêm e vão, mas o que realmente
conta é a intenção, porque aí temos uma palavra a dizer. A intenção é o kamma
imediato da mente e pode gerar sentimentos internos de firmeza, bondade, etc.,
que nos podem ajudar nas marés da vida. Assim, aquilo que devemos procurar não é
a sensação que surge na sequência do contacto dos sentidos, mas aquela que
acompanha a intenção. A luminosidade e a firmeza da intenção adequada
encorajam"-nos a continuar a seguir o percurso com clareza e consciência,
abandonando a ganância, o ódio e a ilusão. E o nosso potencial para a clareza,
força e presença de coração aumenta proporcionalmente.

Esta é a razão pela qual começamos a clarificar, a fortalecer e a confiar de
forma mais plena nas intenções subjacentes ao que estamos a fazer no momento.
Tomamos a responsabilidade sobre a nossa vida e estabelecemos alguns valores e
algumas bases éticas. Depois, para trabalharmos realmente nas raízes, temos a
meditação. Mas a prática da meditação começa, não com a produção de estados
mentais maravilhosos, mas com o estabelecimento de uma fundação estável e
confortável. Sermos conscientes do corpo e sermos amigos de nós próprios,
constituí uma fundação que nos permite ir mais fundo. Podemos dar atenção às
bases do corpo, do pensamento e da mente, trazendo"-os para um alinhamento que
constitui um apoio. Estabilizamos e iluminamos o corpo através da postura e da
respiração e, seguidamente, deixamos que a nossa mente sinta isso. Ou
recordamos, trazemos à mente, uma ideia ou uma imagem que origina inspiração,
calma ou gratidão, até sentirmos isso no corpo. Quando temos a sensação física
de inspiração, do bem"-estar e da estabilidade que acompanha um estado mental
adequado, sabemos que este tem como fundação algo no qual podemos confiar e que
apoia a calma e o discernimento. De forma que, ao invés de confiarmos num único
pensamento ou disposição, é na harmonia entre corpo, pensamento e mente que
encontramos uma sensação mais profunda de confiança.

Centrarmo"-nos na intenção leva à harmonia e reduz a actividade. Parte da
actividade mental é desnecessária, inútil e consiste apenas em impulsos e
reacções, o que na verdade é bastante desagradável. Assim, com a prática do
Dhamma o primeiro objectivo consiste em dar firmeza à qualidade da intenção,
assentar bem os pés na terra e clarear a mente relativamente a qualquer
actividade supérflua. Esta rectidão e clareza traz consigo uma sensação
agradável associada à intenção adequada do kamma luminoso, nesse momento.
Paralelamente, compare a sensação ligada ao sentimento de julgar ou de
manipular, com a sensação associada à intenção de ser generoso ou compassivo.
Quando examina isto na sua mente, chega à conclusão que é sensato produzir uma
grande quantidade de kamma luminoso, de forma a conseguir ter a força e a
capacidade para apoiar um bom sentimento relativamente a si próprio. Não se
trata de uma forma de moralidade, mas de saber que fizemos bem porque quisemos e
que nos sentimos bem com isso. Deste modo aprendemos para onde olhar, de maneira
a avaliar as nossas acções e a nossa vida.

Apesar da sensação associada a uma acção não ser, em última análise, o aspecto
mais importante, constitui, não obstante, uma fonte de apoio para a mente.
Leva"-nos a permanecer com a qualidade daquilo que estamos a fazer no momento e
a desenvolvê"-la. É importante fazer isto, porque quando nos comprometemos com
algo, vão existir momentos nos quais sentimos que não é isso que queremos fazer
-- quer se trate de fazer exercício físico, sentarmo"-nos a meditar ou deixar de
fumar. Mas depois regressamos à nossa intenção: lembramo"-nos e entramos em
contacto com a origem do nosso compromisso. Sinta essa intenção luminosa e
expanda a sua consciência dessa sensação. Irá de seguida continuar com uma
motivação que provém da elevação da mente e que pode ajudar a superar o mau
humor ou a desorientação que resulta de se largar hábitos cármicos antigos.

Assim, o primeiro sucesso em termos do Dhamma prende"-se com o dirigir a atenção
da mente para as origens da forma como se comporta, não a deixando prender"-se
apenas com aquilo que está a receber. Esta mudança permite"-nos colocar uma
questão simples e directa nas profundezas da nossa mente, do nosso coração e do
nosso sistema nervoso: o estado, opinião ou programa que estou a seguir está a
levar"-me para o sofrimento e para a tensão, ou está a libertar"-me deles?

\subsection{Auto-imagem e comportamento auto-imagem}

A questão é que os aspectos do comportamento encontram"-se frequentemente
ligados à auto"-imagem, por vezes até decorrendo desta. Existe um sentido de
`eu' que se baseia num espelhar ou apegar à disposição ou ao programa da mente
presente em cada instante. Assim sendo, elogios e culpa, fama e infâmia, ganho e
perda, felicidade e tristeza, passam a definir a forma como sentimos `a nossa
pessoa'. E, deste modo, a motivação detem"-se em como somos estimados em termos
de barómetros relativos a popularidade, finanças ou desempenho. Todo o tipo de
programas instintivos e reactivos desencadeia"-se a partir destes aspectos. Por
vezes queremos melhorar"-nos e tomamos as rédeas de uma situação. Outras vezes a
mente contrai"-se em defesa e autopiedade. Ou existe uma dúvida relativamente ao
que devíamos ou não devíamos fazer relativamente a nós próprios. Por vezes
queremos que alguém acredite em nós, nos valide, confirme que temos valor, que
somos iluminados, meio"-iluminados ou a caminho de lá chegar. O sentido do `eu'
cria enviesamentos no nosso comportamento. Isto acontece porque a identificação
resulta da necessidade muito básica e forte de nos sentirmos estáveis e sólidos,
e assim, quando existe ignorância, procuramos essa estabilidade em coisas que se
alteram. Se nos identificamos com o sermos elogiados ou estimados, então
sentimos que algo não está bem se isso se altera, originando insegurança. E
quando nos sentimos inseguros, temos a sensação de que essa insegurança
constitui um aspecto da nossa identidade. Obviamente, ninguém quer ter como
identidade um estado desagradável, por isso tentamo"-nos ver livres dele e
encontrar um lugar, nas nossas vidas, onde nos sintamos luminosos e confiantes.
E assim vai sucedendo, uma vez e outra e outra\ldots{} Isto parece razoável,
contudo a mente não atinge essa segurança ao agarrar"-se ou a reagir às suas
próprias disposições, pois estas são passageiras. Tentarmos ser um estado ou um
não"-estado permanente, apenas nos mantém agarrados e fixados a imagens
transientes, ou a lutar com elas. Sendo que tudo isto torna qualquer tentativa
de introspecção um processo de inquietude e calculismo.

Mais uma vez, o comportamento é a chave, não a identidade. O kamma mental é
importante. Devemos de estabelecer uma ligação consciente com os eventos da
nossa vida de forma a não ficarmos presos. Termos consciência faz com que não
nos fixemos, mas sim que estejamos a testemunhar o presente de forma empática. E
isto significa penetrar directamente os ímpetos inconscientes subjacentes que
formulam aquilo que experienciamos no momento. Se nos interrogamos quanto tempo
vai levar até ficarmos iluminados, é porque estamos a ser levados pelo ímpeto da
preocupação ou da dúvida. Se uma doença está a dificultar a meditação, existe o
potencial de sermos levados ao desânimo. Por isso, nesse momento a prática tem
de incluir algumas intenções amigáveis e de suporte ao nosso corpo, com menos
exigências de fulgor e vigor. Podemos fazer uma boa prática ao largar a
irritação relativa à forma como achamos que as coisas deviam ser ou a ansiedade
acerca de como poderão ser. Este largar constitui uma acção vital, e podemos
apoiar essa acção expandindo conscientemente as boas intenções que conseguimos
desenvolver e suster. Isto é feito ao largamos a construção de narrativas e de
imagens do eu (como fracasso, vítima, monstro) que as nossas dificuldades fazem
surgir e, em vez disso, centramo"-nos naquilo em que resplandecemos e somos
fortes.

Métodos como este levam ao desenvolvimento da nossa capacidade. Até a mera
intenção de se ser paciente e manter a determinação sobre os nossos padrões
éticos, já constitui um ganho em termos de capacidade mental. E ao sair de
padrões e programas habituais, começamos a despertar para além desta imagem que
se agarra àquilo que passa. À medida que conseguimos alguma liberdade em relação
ao apego, conseguimos mais facilmente ver essa identificação e, por conseguinte,
conseguimos ver como a nossa identidade constitui um processo em mutação. A
instabilidade a este nível não é uma fraqueza, mas sim um facto a ser
reconhecido. Quando isto é compreendido e integrado, surge a perspectiva de
largar, ou seja, o caminho para a Imortalidade.

\subsection{Devir, não devir e visão correcta}

O processo de identificação encontra"-se latente na auto"-imagem e é activado
pelo apego. À medida que ocorre o apego, o processo adquire carácter e traços
pessoais através de uma visão enviesada chamada `devir' (\emph{bhava}).
`Perceber' torna"-se `eu percebi' ou `estou preso nisto' ou `vou ser sempre
desta forma'. O apego liga este momento ao próximo momento e o devir faz com que
isto se torne um padrão, que pode depois ser projectado no tempo em termos de
expectativa ou de receio; ou tecido num retrato detalhado relativo a `ela é
assim' ou `eu nunca serei assim'. O devir é a ignorância em acção, o tecelão
principal dos padrões e dos programas, o `paizinho' de todo o \emph{saṅkhāra}.

De forma mais detalhada, o devir transporta as marcas do kamma antigo, de forma
que, quando existe apego, este (o devir) acrescenta os detalhes pessoais. Faz
isto por lidar com os estímulos através da habitual forma de `eu' e `meu': kamma
antigo. Podem existir misturas de impulsividade e de preocupação, ou traços mais
positivos como querer agir de forma a agradar aos outros: a `minha' forma de
reagir e de ver as coisas. O `devir' forma e informa a sensação contínua de
`eu', à medida que se desloca a de uma disposição da mente para outra disposição
da mente, através dos acontecimentos e cenários, bem como de várias vidas.
Procura um padrão estável e satisfatório. O aspecto mais duro, mais difícil de
aceitar, é que esta sensação de `eu' nunca consegue ser muito estável ou
satisfatória, uma vez que surge a partir da dependência de nos agarrarmos a um
estado qualquer de existência, estado este que se altera. Agarrarmo"-nos ao que
é mutável e instável conduzir obrigatoriamente a resultados insatisfatórios.
Quanto tempo está um vencedor satisfeito com o seu sucesso, antes de ter de
correr mais rápido, escalar uma montanha mais desafiante ou conseguir um negócio
ainda mais rentável?

E, entretanto, as circunstâncias alteram"-se. Talvez se dê a perda do
companheiro ou do emprego, talvez ocorra uma doença ou incapacidade, talvez a
confiança numa pessoa ou numa doutrina seja abalada, ou a capacidade de fazer
algo ou de fazer com que as coisas aconteçam seja cerceada: aí sentimo"-nos
desorientados. O nosso programa de `devir' descarrila e com isso surgem a mágoa,
a ansiedade e a zanga. Então, de forma a alterar o curso ou a suprimir estas
emoções, entramos novamente em actividade -- culpamo"-nos ou atafulhamo"-nos de
trabalho, ou procuramos algum tipo de afirmação \ldots{} E toda esta actividade
e energia constituem mais `devir', sendo que este apenas encontra outro conjunto
de disposições da mente em torno das quais vai criar uma nova identidade. Deste
modo existe um vício relativo ao kamma e uma nova convicção que tudo o que
existe para nós é a enxurrada de sensações e o ímpeto de nos tornarmos isto e
aquilo durante algum tempo. O processo de `devir' mantém"-nos à procura do nosso
derradeiro e duradouro estado. Mas este não existe.

Contudo, não saímos deste processo de `devir' através de intenções negativas ou
de desligarmos a mente. Isto constitui o `não devir' -- o gémeo sombra do
`devir'. O `não devir' procura libertar"-se das disposições da mente e situa"-se
próximo do niilismo, da retirada e da sensação de ausência de significado.
Trata"-se de um padrão popular nos anos de rebeldia da adolescência -- um padrão
que, não obstante, dá igualmente origem a atitudes, estilos, ícones e
compulsões. Não! Temos de crescer e ultrapassar o devir ao direccioná"-lo para
sítios nos quais o nevoeiro da ignorância se pode dissipar. Assim, conduzimos as
nossas intenções para as práticas que apoiam as \emph{pāramī} e os Factores do
Despertar. E começamos a rever a forma como a nossa mente funciona em termos de
adequado ou de inadequado, escuro ou luminoso, ao invés daquilo que sou ou
poderia ser. É esta visão de testemunhar, de avaliar sem apego e sem nos
espelharmos a nós próprios, que temos de ter presente à medida que o conteúdo e
os padrões da mente são revelados.\pagenote{`Quaisquer que tenham sido os
  reclusos ou bramânes a afirmar que a liberdade relativa ao devir surgiria
  através de outro tipo de devir, nenhum, digo"-vos, se libertou do devir. E
  quaisquer que tenham sido os reclusos ou bramânes a afirmar que a liberdade
  relativa ao devir surgiria através do não"-devir, nenhum, digo"-vos, se
  libertou do devir. Este sofrimento depende do apego. Com o fim de todo o
  apego, não é produzido qualquer sofrimento.'
  \href{https://suttacentral.net/ud3.10}{Udāna 3.10}}

O verdadeiro alcance desta visão altruísta está incluído no ensinamento das
`Quatro Nobres Verdades': do sofrimento, da sua origem, da sua cessação e do
caminho para essa cessação. Com este ensinamento podemos avaliar onde estamos
encalhados, o que podemos fazer acerca disso, onde nos estamos a libertar e como
o desenvolver. Apesar de constituir uma avaliação íntima, não se baseia na
auto"-imagem. Pelo contrário: é como usar uma radiografia ou um angiograma para
olharmos para o nosso estado. Não estamos a ver a auto"-imagem normal, através
do seu olhar fixo inflacionário ou contraído, mas sim a ver os nossos padrões em
termos das Quatro Nobres Verdades: esta é a área que causa sofrimento; é aqui
que é gerado; este é o aspecto mais claro e com mais recursos; é aqui que o
desenvolvimento pode ocorrer. As Quatro Nobres Verdades proporcionam"-nos um
mapa do kamma antigo que transportamos connosco, sobre como o novo kamma é
gerado e sobre o kamma que leva ao Despertar. Ou seja, se a mente estiver
estabilizada, aberta e desobstruída, podemos experienciar uma quietude
intrínseca e clara. É algo que podemos sentir apenas quando as energias
afectivas e impulsivas da mente se aquietam, algo que não possui nem intenção,
nem sensação e que não sustenta o devir e a auto"-imagem. É uma forma de
ausência de peso que, simultaneamente, é a coisa mais assente na terra e estável
que podemos conhecer.

\subsection{Estabelecer as Quatro Nobres Verdades}

A Primeira Nobre Verdade diz respeito ao `des"-conforto', à qualidade de
insatisfação, ao sofrimento e à tensão: \emph{dukkha}. À primeira vista, não
conseguimos escapar a este `des"-conforto': `nascimento é \emph{dukkha},
envelhecimento é \emph{dukkha}, morte é \emph{dukkha}\ldots{} estar ligado ao
que é desagradável, ser separado do que é agradável, não alcançar o que
desejamos é \emph{dukkha}. Resumidamente, os cinco agregados afectados pelo
apego são \emph{dukkha}.'\pagenote{As Quatro Nobres Verdades estão apresentadas
  em \href{https://suttacentral.net/sn56.11/en/bodhi}{SN 56.11}.}
O único aspecto que podemos aqui questionar é o seguinte: o que é isto dos cinco
agregados (\emph{khandhā})? Estes são: a forma material (tal como o nosso
corpo), os gradientes das nossas sensações, os significados sentidos, os padrões
e os programas (\emph{saṅkhāra}) e a consciência cognitiva. A nossa experiência
habitual é constituída por uma mistura destes agregados. E a verdadeira
compreensão de \emph{dukkha} significa a compreensão destes agregados, de forma
a podermos penetrar e largar o apego. Penetrar e largar o apego: mas como é que
fazemos isto?

Algumas coisas estão estabelecidas, sem qualquer opção. O corpo está destinado a
experienciar dor e morte; um aspecto das nossas sensações é serem
inevitavelmente dolorosas; os significados sentidos são condicionados; não temos
escolha relativamente aos programas funcionais, tais como respirar e sermos
seres sencientes; e a experiência de termos consciência cognitiva significa que
recebemos uma quantidade imensa e aleatória de informação através dos sentidos,
que ocupa a nossa mente, nem sempre em nosso proveito. Aquilo a que necessitamos
de dar atenção é, assim, aquilo sobre o qual temos alguma escolha: os programas
da nossa mente e as intenções que estes transportam. Aqui pode ser feita uma
escolha: com a intenção deliberada, podemos distanciar"-nos das intenções
estagnadas e danosas. Podemos largar a nossa aversão e perdoar os nossos
inimigos. Podemos abandonar hábitos, compulsões e vícios. E isto dá origem a
algum bem"-estar e à noção de que a libertação é possível.

Este grau de abrir"-mão altera as impressões"-de-contacto e amplia a atenção --
o que também afecta os nossos programas. Ou seja, quando mudamos o foco e `ter
as coisas feitas a tempo' para `ter paciência', a atenção é ampliada e então a
mente pode ficar aberta a reflexões mais amplas sobre aquilo que é adequado
neste momento. De igual modo, se desviarmos a atenção de temáticas irritantes
para outras não irritantes, mais uma vez entramos em contacto com estados mais
equânimes e de abertura, de onde resulta um comportamento mais equilibrado.
Acima de tudo, trata"-se de alterar a nossa visão de `eu', de `a minha maneira'
e de `porque é que a vida é tão injusta?', para `onde é que está a tensão e onde
é que acaba?'. O sofrimento e a tensão são assim cerceados, alguns programas de
longo termo são desligados e o crescimento pessoal avança. Consequentemente, a
compreensão da Primeira Nobre Verdade encoraja"-nos a trabalhar os
\emph{saṅkhāra}, de forma a colocarmo"-nos na pista dos padrões e dos programas.
Trata"-se da forma mais imediata que temos de nos libertarmos da tensão.

Por exemplo, quando olhamos para o quão ficamos magoados por ver a nossa nora a
maltratar o nosso filho, temos o instinto de nos envolver na discussão e tomar
partido do nosso filho. Em alternativa, podemos considerar o assunto como
dizendo apenas respeito a eles. Contudo, como sabemos, apesar de isto soar
verdadeiro, na realidade sentimo"-nos afectados, como que fazendo parte daquele
cenário. Então o que podemos fazer sem negar esse desconforto, mas também sem
nos tornarmos moralistas e gerar mais tensão para todos? Bem, podemos trabalhar
a forma como os nossos próprios padrões estão a designar a nora como `bruxa' e
ver se podemos fazer alguma coisa relativamente a isso. Talvez possamos ampliar
a nossa atenção: começar a ver o lado bom dela, a confraternizar amigavelmente e
a tentar compreender porque é que ela age daquela forma. Quando contemplamos a
forma como os nossos padrões funcionam, podemos considerar que muito daquilo que
os outros fazem resulta de um reflexo inconsciente. E podemos reconhecer, neste
exemplo, que tendemos a favorecer o filho e, talvez, não vejamos as suas facetas
exasperantes. Mas como é que ele despoleta o programa dela? Quaisquer que sejam
os resultados de tentar compreender, em profundidade, ao invés de reagir ao
\emph{dukkha} da situação, no mínimo libertam"-nos do nosso próprio sentimento
de frustração e de impotência. Desenvolvemos a intenção de compreensão, tornamos
a atenção mais ampla e menos reactiva e isso cria uma sensação muito melhor do
que simplesmente manter as pessoas em papéis e posições fixas.

À medida que trabalhamos sobre os nossos padrões e programas, desenvolvemos uma
compreensão da forma como estes prendem a mente a reflexos de agarrar e
rejeitar, de julgamento e especulação, de preocupação e de ânsia. Isto leva"-nos
à Segunda Nobre Verdade, segundo a qual \emph{dukkha} tem uma origem: no impulso
ou reflexo da ânsia e da aversão. Ou seja, existe a ânsia que se encontra ligada
a vermos, ouvirmos, tocarmos, saborearmos e cheirarmos; e a parcialidade disso
produz irritação e aversão quando não conseguimos aquilo que queremos. E existe
igualmente a ânsia por nos tornarmos e não nos tornarmos. Então, a primeira
coisa é fazer com que a mente fique suficientemente clara para verificar esses
reflexos, devido ao seu poder instintivo para despistar o Despertar com as
opiniões próprias. Para isto é preciso motivação (\emph{chanda}), ao invés de
ânsia: a responsabilidade para substituir as intenções reflexas por intenções
claras. Quanto da intenção se baseia na tentativa de não sermos quem achamos que
somos? Ou de tentarmos usar um sistema que melhore o reflexo nesse espelho com
distorções? Então é melhor reconhecermos, enquanto padrões, os desejos e o
egotismo inadequados, em vez de os negarmos lançando um véu sobre o processo do
apego. O apego à auto"-imagem deve ser abandonado.

A Terceira Nobre Verdade tem a ver com o fim de \emph{dukkha}. A intenção
associada a isto é que deve ser plenamente compreendida. Isto significa expandir
uma consciência da área do nosso domínio pessoal que é isenta de sofrimento e de
dor. Começamos por reconhecer a ausência de pressão e a presença de equilíbrio
nas nossas vidas: as alturas nas quais não estamos a construir um qualquer
futuro, passado ou presente; a sensação na qual sentimos quietude. É subtil
porque são exactamente a pressão, os impulsos e os repelões no coração que
atraem a atenção. Damos muito importância e levamos a vida a partir dos `uau!' e
dos `porquê eu?', dos nossos padrões emotivos. Assim, o reconhecimento do
não"-sofrimento exige uma intenção deliberada.

Por exemplo, quando existe dor física, o/a leitor/a consegue cultivar a atenção
de reparar onde não há dor? Se tem dor nas pernas, consegue reparar no conforto
do pescoço? Porque o padrão da mente consiste em gerar significados sentidos a
partir de sentimentos locais, sendo que daí surge a experiência `sinto dor'. É
um bom começo alterar isso para `sinto dor na minha perna'. Isso controla o
programa de \emph{saṅkhāra} que gera o `significado sentido'. Então, com a
compreensão de dukkha, temos em consideração que os corpos experienciam
sensações e que um tipo destas sensações é a dor. O corpo está a fazer aquilo
para que foi feito, o que nem sempre vai ao encontro do que gostaríamos. Assim,
uma pequena parte do sofrimento de possuirmos os agregados pode ser deixada de
parte. E isto permite um certo desapego, uma mudança para uma perspectiva mais
manobrável em relação à dor. Isto tem de ser compreendido, mantido vivo e
expandido. É aí que a mente não se encontra sob tensão. E se nos centramos nessa
ausência de tensão, conseguimos uma porta para o domínio da mente que não é
baseada na sensação, na interpretação e na reacção. Temos uma noção de um
conhecimento sereno (apesar de não ser acerca de sensações, dá uma `boa
sensação', da mesma forma como o alívio da pressão sabe bem). Este é o domínio
do não"-sofrimento, e é um lugar de estabilidade pois não se encontra preso a
sensações, significados sentidos e estados mentais.

Consegue notar o momento em que um pensamento termina? Ou reconhecer que uma
obsessão particular já não está a decorrer como habitualmente? Não parece muito
significativo, mas com isto estamos a fazer frente ao \emph{saṅkhāra} que se
identifica com os fenómenos e com os problemas, como se fossemos apenas
constituídos por estes. Reparar no momento em que o fenómeno desaparece
constitui uma forma de treinar a mente a vislumbrar o desapego. Expandimos a
consciência dos momentos nos quais a mente não está a procurar estimulação
através de pensamentos ou de memórias -- os tempos ou lugares na nossa
consciência que são tranquilos. Esta base desemaranhada deve ser realizada.

A Quarta Nobre Verdade é a verdade do Caminho, com a intenção que este deve ser
desenvolvido. Esta intenção integra todas as práticas de ética, de meditação e
de compreensão. A apresentação da Quarta Nobre Verdade leva"-nos a considerar a
visão correcta, o objectivo correcto, o discurso correcto, a acção correcta, o
modo de vida correcto, o esforço correcto, a consciência correcta e a
concentração correcta. O que está `correcto' em todos é o facto de se basearem
na verdade do kamma e não na noção de `eu'. E o processo do kamma é bem mais
directo do que o processo do `eu'. Tentar compreender e satisfazer uma
auto"-imagem é uma tarefa que, quanto mais tentamos, mais produz complexidades
-- mas podemos meter mãos à obra com o princípio básico de fazer o bem e
purificar as nossas intenções de uma forma directa. O Caminho pode ser
desenvolvido. E é o Caminho que encoraja o desemaranhar, pois amplia a
experiência no sentido do bem"-estar e da estabilidade resultantes de
carregarmos menos necessidades e menos autodefinições.

\subsection{Da força de vontade à renúncia}

No que diz respeito à sua abrangência, a intenção deve ser desenvolvida no
contexto das Quatros Nobres Verdades, incluindo desenvolver a própria qualidade
da intenção nesse âmbito. Ou seja, por vezes o largar é realizado
deliberadamente, como um raspanete que damos a nós próprios quando estamos
prestes a desviar"-nos seriamente do caminho. A força de vontade tem a sua
utilidade. Quando estamos hipnotizados, não devemos continuar a olhar fixamente
e a meditar no pêndulo que balança, mas sim usar a força de vontade para nos
libertarmos. O mesmo se aplica aos hábitos viciantes: temos de usar a nossa
força de vontade e de apoiar essa acção de libertação com uma determinação firme
e empenhada.

De igual modo, apenas para libertar algum espaço, um praticante de meditação
pode aplicar uma determinação firme no sentido de aguentar emoções dolorosas ou
de afastar as forças e as imagens do desejo que assediam a mente. Mas se durante
muito tempo apenas funcionamos dessa forma, corremos o risco de nos tornarmos
brutos e estúpidos. E a força de vontade cria dependência: as pessoas que
funcionam amplamente através da força de vontade dão por elas a procurar algo
onde aplicar essa determinação, uma vez que ficam desorientadas sem esse efeito
galvanizante. A força de vontade tem igualmente o efeito secundário de reduzir a
receptividade e a flexibilidade, sendo que isso prejudica a sintonia da mente
com energias mais subtis e, consequentemente, limita a capacidade de
investigação. Por si própria, a força de vontade não consegue proporcionar calma
ou sabedoria de realização interior.

Assim, a força de vontade é útil a curto prazo mas, quando temos de ultrapassar
a necessidade de sentir que temos as coisas sob controlo, ou a incapacidade para
aceitar responsabilidade, precisamos de intenções mais sensíveis e receptivas --
tais como uma bondade clara e não sentimental face a aspectos da nossa dor. Dito
de outra forma, a intenção é aprimorada quando se ajusta a formas subtis e mais
inteligentes.

Este desenvolvimento está alinhado com o decréscimo na densidade dos fenómenos
mentais com os quais cada um de nós tem de lidar. O trabalho contínuo no sentido
de fortalecer e treinar a consciência, bem como de ter uma vida mais
equilibrada, tem este objectivo. Este constitui um aspecto fundamental da
meditação.

A meditação requer igualmente o desenvolvimento do não envolvimento
(\emph{viveka}) e do desencanto (\emph{virāgā}). Assim, praticamos a utilização
e o ajustamento da atenção de forma a estarmos na presença de qualquer obstáculo
ou obsessão, sem ficarmos presos. Centrada nesse testemunho, a consciência
cognitiva distancia"-se de forma a ganhar perspectiva, analisa e larga as
imagens, as histórias e as energias que sustentam um obstáculo (a lista
fantasista de desejos ou os tribunais que me irão vingar (ou condenar)). E
conseguimos vislumbrar o não"-afligido, precisamente nessa capacidade de nos
distanciarmos e de deixarmos as coisas cessarem (\emph{nirodhā}). Então, nesse
ponto onde a tensão e o conflito começam, pode existir, ao invés, um distanciar
e um tipo de compreensão esclarecedora. E, ao estarmos assim plenamente
presentes, a contracção, o afundamento ou a rotação do obstáculo pára, e o
padrão quebra"-se.

Essa experiência da Terceira Nobre Verdade deve ser, então, aprofundada. Quando
a mente se encontra fora da força gravitacional dos obstáculos, dá"-se uma
profunda descontração e, à medida que vamos voltando nessa direcção, a intenção
torna"-se mais subtil. Parece que não existe nada para fazer. O fluxo de energia
mental parece parar ou tornar"-se tranquilo, sem forma. Não surge nada de
especial. Podemos interrogar"-nos: `então, e agora?' ou `será que é isto?' ou
mesmo querer suster esse estado. Neste ponto, a própria intenção deve ser
largada.

Consequentemente, o desenvolvimento final é o da renúncia (\emph{vossagga}), a
renúncia da intenção. E com isto dá"-se um abandono do domínio cármico. Então
como é que largamos o `fazer'? `Prestando atenção ao que é imortal' -- é a
resposta breve. Mais detalhadamente envolve a afinação da atenção e da intenção.
Normalmente, uma mudança vigorosa (por exemplo da irritação para a paciência)
surge através de prestarmos atenção à qualidade desagradável da irritação e aos
aspectos `não irritantes' da pessoa ou do acontecimento que nos está a
incomodar. `Muito bem, esperar durante uma hora pelo autocarro não é muito
divertido, mas não chove na paragem e a espera não me vai matar'. Ou podemos ter
presente a paciência, evocá"-la deliberadamente e prestar atenção a essa
qualidade. Dito de outra forma, para nos deslocarmos do sofrimento para o não
sofrimento, substituímos uma imagem ou um estado da mente por outro. Contudo, à
medida que temos maior capacidade para regular a irritação, pomos de lado as
imagens que a despoletam e depois investigamos as suas energias -- então a
irritação já não entra em acção e tende a dissipar"-se por si própria. A mente
regressa a um estado estável apenas através do não envolvimento, do desapego e
de não seguirmos a irritação. Este é o `não fazer' da sabedoria da realização
interior.

A profundidade desta prática está não só na forma como lida com o estado da
mente, como também com o apego a ele e até mesmo à opinião e à tendência
relativas ao devir. O devir é o que faz o estado da mente parecer tão sólido e
cria um dono que o experiencia, que tem de agir em conformidade ou fazer algo
relativamente a ele. Trata"-se de uma opinião baseada no apego ao kamma antigo
como `eu e meu' e que, desta forma, prepara o caminho para o novo kamma nessa
mesma linha.\pagenote{``Então, sabendo de que forma, vendo de que forma,
  poderemos por um fim imediato às corrupções? Dá"-se o caso no qual uma pessoa
  não instruída, uma pessoa comum\ldots{} Supõe que o corpo e a forma são o
  `eu'. Esta suposição é um \emph{saṅkhāra}\ldots{}. Ou não supõe que o corpo e
  a forma são o `eu', mas supõem que o `eu' possui forma\ldots{} ou que a forma
  está no `eu'\ldots{} ou que o `eu' está na forma\ldots{} ou que a sensação é o
  `eu'\ldots{} ou que o `eu' possui sensação\ldots{} ou que a sensação está no
  `eu'\ldots{} ou que o `eu' está na sensação\ldots{} ou que a percepção é o
  `eu'\ldots{} ou que o `eu' possui percepção\ldots{} ou que a percepção está no
  `eu'\ldots{} ou que o `eu' está na percepção\ldots{} ou que o \emph{saṅkhāra}
  é o `eu'\ldots{} ou que o `eu' possui \emph{saṅkhāra}\ldots{} ou que o
  \emph{saṅkhāra} está no `eu'\ldots{} ou que o `eu' está no
  \emph{saṅkhāra}\ldots{} ou que a consciência cognitiva é o `eu'\ldots{} ou que
  o `eu' possui consciência cognitiva\ldots{} ou que consciência cognitiva está
  no `eu'\ldots{} ou que o `eu' está na consciência cognitiva. Este assumir,
  este pressuposto é um \emph{saṅkhāra} \ldots.

  [\ldots{}]

  Ou\ldots{} podem ter uma opinião como a seguinte: `Este `eu' é o mesmo que o
  universo. Assim, depois da morte irei ser permanente, duradouro, eterno, não
  sujeito à mudança.\textquotesingle{} Esta opinião eternalista é um
  \emph{saṅkhāra} \ldots{} Ou\ldots{} podem ter uma opinião como a seguinte:
  \textquotesingle Eu talvez não exista, nem aquilo que é meu\textquotesingle.
  Esta opinião de aniquilação é um \emph{saṅkhāra} \ldots{} Ou\ldots{} podem
  ficar perplexos, duvidosos ou indecisos relativamente ao verdadeiro Dhamma.
  Essa perplexidade, dúvida e indecisão é um \emph{saṅkhāra}.

  Agora, qual é a causa, qual a origem desse \emph{saṅkhāra}, do onde é que ele
  nasce e é produzido? Quando uma pessoa não instruída, comum, é tocada por uma
  sensação que tem origem no contacto acompanhado pela ignorância, surge o
  desejo. Esse \emph{saṅkhāra} (tudo o que foi referido anteriormente) nasce
  disso. Assim, esse \emph{saṅkhāra}, bhikkhus, é impermanente, condicionado,
  tem origem dependente. Esse desejo\ldots{} Essa sensação\ldots{} Esse
  contacto\ldots{} Essa ignorância é impermanente, condicionada, tem origem
  dependente. Quando sabemos e vemos desta forma, as corrupções cessam
  imediatamente.''

  \href{https://suttacentral.net/sn22.81/en/bodhi}{SN 22.81}}
Dito de outro modo, continua a ver o `eu'. Mas com realização interior, não
estamos a tomar parte nesse processo, não nos estamos a identificar com a
irritação, nem a construir um padrão alternativo para a substituir, nem a ser
alguém que a suplanta. No domínio da realização interior, as coisas libertam"-se
por elas próprias. E quando conseguimos contactar essa qualidade de libertação,
esse hiato no tecido da mente padronizada -- isso é a Imortalidade. É como os
buracos numa rede: não é bom nem é mau. Não se apega porque não o consegue
fazer. Não tem essa energia e não tem essa perspectiva: situa"-se para além dos
padrões e dos programas.

\subsection{Pessoas altruístas: esvaziar o espelho}

Essa renúncia tem também um efeito a longo prazo: o de deixarmos de ter de
carregar os nossos programas cármicos. Não tenho de ser algo, simplesmente
porque nunca fui capaz de ser algo para começar -- tudo o que aconteceu foi um
emaranhado de actividade confusa. O dono, aparentemente preso, da mente é
exposto como sendo um fantasma, uma confusão da consciência. E, à medida que
essa confusão amaina, o mesmo acontece ao impulso da intenção -- existe um
sentimento de libertação, de leveza e de liberdade. Um momento desta renúncia
não constitui o fim da história, mas é um progresso, porque o impulso cármico
pára por um momento, existindo uma renúncia ao sentimento de agência e à tensão.
Pode ocorrer uma quietude involuntária, uma quietude mantida não
intencionalmente. E, a longo prazo, isso afecta de forma radical o nosso apetite
por estados de ser: podem ser um apoio e vitais para o Caminho, mas não são a
essência do Fruto do Despertar.

Porque enquanto existir a crença que um eu real é o dono, o autor ou o herdeiro
do kamma, essa crença sustenta os padrões segundo os quais nos sentimos bem ou
mal, implicando uma necessidade de agir em relação a isso. Quando essa crença é
abandonada temos paz, porque não há necessidade de fazer seja o que for. Mas não
se trata da crença de sermos alguém que é independente do kamma ou de que este
não existe. No domínio do kamma (causa e efeito) a habilidade de lidar com este
tem de ser exercitada e é na realidade necessária para se sair do emaranhado dos
agregados. Depois disso, o kamma pode terminar na Imortalidade.

A experiência da renúncia constitui o início de um processo gradual de Despertar
que inclui níveis progressivamente mais profundos da programação cármica. Mas o
primeiro estádio, de `entrada na corrente', resume"-se na eliminação de três
formas sob as quais os programas do `eu' operam -- orientações do eu, se
preferir. Estas incluem: orientação em torno da identidade histórica
(personalidade), orientação em torna da dúvida e orientação em torno das regras
e dos costumes. Ou seja, a pessoa comum pressupõe que a sua personalidade é uma
identidade histórica -- `gosto disto, não tenho jeito para isto, pertenço a esta
família, etnia, género'. O que nos pode alertar é o facto da identidade ter de
ser mantida através de um fluxo bastante constante de pensamentos e de emoções
que nos dizem o que devemos ser, ou que nos fornecem assuntos de carácter
pessoal, relacional, étnico ou de género de forma a ser agitada. Alguns deles
são monólogos de julgamento e exigência do Tirano Interior: os padrões que nos
levam a fazer algo de modo a provarmos que somos suficientemente bons e a
tornarmo"-nos mais úteis, estimados ou contentes. Alguns são disposições de
frustração e derrota. Mas todos dependem desta actividade psicológica: são kamma
antigo a ser regurgitado. Alguém que entra na corrente já lidou com a sua
herança a este nível de identidade até ao ponto no qual já se sente à vontade na
sua pele convencional.

De forma geral, as pessoas deixam"-se arrastar pelos livros e especulam em busca
da certeza: `será que vou encontrar aqui aquilo que preciso? Ela diz que o
despertar é assim, mas ele diz que é assado\ldots{}' Ou apega"-se a rotinas
diárias, sistemas religiosos ou metafísicos, ou sistemas de meditação do género
`Esta é a Via, faça dez destes a tais horas e depois vai ver que as coisas vão
começar a avançar'. Mas com a entrada na corrente estes padrões também são
abandonados: alguém que entra na corrente não está a tentar provar"-se a si
próprio de acordo com uma ideia ou um sistema; ou a falhar a esse nível e a
tentar encobrir o insucesso. Devido a ter descoberto maior estabilidade, clareza
e plenitude do coração através de uma experiência sem entraves, não tenta
orientar"-se de acordo com esses entraves. Alguém que entra na corrente
compreende a utilidade e as limitações das opiniões, ideias e ações, e sabe como
largá"-las.

De modo que aquilo que se manifesta é uma pessoa progressivamente mais
altruísta, grata pelo Dhamma e estável nos seus objectivos. Os reflexos de
irritabilidade e de sensualidade, de fascínio com as absorções da meditação, de
autorreferência, agitação e ignorância permanecem, mas alguém que entra na
corrente sabe o que é a auto"-imagem e onde esta pode terminar. Isso significa
que continua a trabalhar nos restantes entraves com confiança.

Por agora, este é um bom sítio para pararmos.

\clearpage

\section[Meditação: encontrar-se com o limiar]{Meditação}

{\centering
\subSectionFont\selectfont
\textit{Encontrar-se com o limiar}
\par}

\bigskip

Tome consciência de todo o corpo, centrando-a no eixo vertical e na respiração.

À medida que esta qualidade de estar centrado se torna clara, alargue a
amplitude da consciência. Expanda a consciência através do corpo e no espaço
imediato que o rodeia, tão longe quanto se sentir confortável e sem perder o
contacto com o centro. Contemple e disfrute as energias em mutação dentro dessa
esfera de consciência.

Irão surgir perturbações. Estas podem estar relacionadas com um som que ouve ou
com uma sensação física desagradável. Sinta a sua consciência a engelhar"-se ou
contrair"-se no limiar dessa perturbação. Talvez as coisas comecem a acelerar,
ou surjam impulsos para superar ou escapar da origem dessa perturbação. Em vez
de seguir esses impulsos, reconheça o que se está a passar e descontraia os
automatismos que estão a tentar lidar com a perturbação, para continuar assim a
alargar gradualmente a esfera da consciência, como se estivesse a abarcar ou
mesmo a abraçar a perturbação. Descontraia os contornos do limiar da perturbação
e, lenta e silenciosamente, contemple o efeito que isso tem.

Algumas perturbações, a nível mental, irão ocasionalmente ocorrer. Estas podem
estar ligadas a outras perturbações sensoriais, tais como a sensação de incómodo
associada a um barulho repetitivo na sala do lado. Ou podem ser puramente
mentais -- pensamentos sobre coisas que temos de fazer, ou uma memória agradável
ou um quebra"-cabeças interessante, que parecem estar a pedir"-nos para nos
envolvermos com eles. Por vezes trata"-se de um arrependimento relativo ao
passado, ou de uma dúvida sobre meditação. Reconheça qualquer uma destas
perturbações em termos de uma ondulação ou de uma agitação, uma alteração de
velocidade e de energia. Vá mais devagar e espere na presença dessa perturbação.
Não reaja nem esteja com pressa para mudar seja o que for. Ao invés, suavize a
sua atitude para com a agitação e tente discerni"-la em termos da sua energia.
Encontre"-se com o limiar dessa agitação e amplie a sua consciência sobre ele.

Continue a ampliar e a descansar na esfera da consciência. Deixe que as ondas da
perturbação sigam o seu próprio caminho. À medida que tudo se aquieta, sinta e
contemple esse efeito, sem palavras.

Quando sentir que é tempo de deixar a meditação, espere -- sinta a energia dessa
intenção. Amplie a sua consciência sobre o limiar dessa intenção que surgiu.
Contemple e abra"-se seja ao que for que seja revelado.

Regresse ao centro, sentindo a parte interna do corpo e a respiração. Abra"-se
ao espaço à sua volta, aos sons e, por fim, ao campo visual.
