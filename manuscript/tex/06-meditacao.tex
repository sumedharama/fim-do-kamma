\chapterNote{Dar corpo à mente}

\chapter{Meditação}

\tocChapterNote{Dar corpo à mente}

Sente-se numa postura verticalizada e dirija a consciência para a experiência no
momento presente. Interrogue-se: `Como é que eu sei que tenho um corpo?' Por
outras palavras, procure a experiência directa da noção do corpo -- as pressões,
energias, pulsações e vitalidade que traduzem a consciência do corpo.
Seguidamente, a partir dessa situação de sensibilidade directa, procure mais
detalhes.

Empurre um pouco para baixo o seu cóccix e o fundo pélvico. Repare como isso
ajuda a colocar a coluna num equilíbrio, no qual o sacro está direito e a região
lombar forma um arco com alguma tensão. Evite bloquear ou forçar. Faça um
ligeiro movimento no sentido descendente para formar o arco, em vez de forçar
uma curva exagerada com um ímpeto no sentido ascendente com os músculos da zona
lombar. Isto fornece à postura a sua fundação crucial: permite ao corpo ser
sustido por uma mola que transfere o seu peso para aquilo sobre o qual se
encontra sentado.

Desloque a sua consciência de forma gradual e delicada pela sua coluna vertebral
acima, desde a ponta do cóccix, através do sacro e das vértebras lombares e
dorsais. Estique o corpo ligeiramente no sentido ascendente, a partir das ancas.
Verifique o centro das costas, entre as extremidades inferiores das omoplatas:
dê vida a esta zona ao trazê-la para dentro, no sentido do coração. Indo para
cima, certifique-se que os ombros estão para baixo e descontraídos e faça um
varrimento com uma consciência descontraída desde a base do crânio até às faces
laterais do pescoço e ao longo do topo dos ombros. Leve a consciência para as
vértebras do pescoço -- conscientize-se de que existe uma sensação de espaço
entre a face posterior do crânio e a zona superior do pescoço. Isto pode ajudar
se colocar o seu queixo para dentro e o inclinar ligeiramente para baixo.
Verifique o equilíbrio geral -- se a cabeça se encontra equilibrada na coluna,
alinhada directamente acima da pélvis. Certifique-se que a coluna se encontra
descontraída. Descontraia igualmente os ombros, os maxilares e deixe que o peito
se abra. Permaneça algum tempo a sentir a estrutura óssea, dando a sugestão às
articulações situadas entre os braços e os ombros, por exemplo, para que estas
fiquem soltas e com uma sensação de abertura. Deixe os braços alongarem-se.
Descontraia-se no equilíbrio.

Esteja atento às sensações em termos físicos: por exemplo, a forma como o peso
do corpo se encontra distribuído; ou o grau de vitalidade e de calor interno que
se encontra presente. Sinta os movimentos subtis do corpo, mesmo quando este
está quieto -- o pulsar, o palpitar e as sensações rítmicas associadas à
inspiração e à expiração. Sinta-se confortável: avalie as sensações físicas em
termos de tranquilidade. Uma certa pressão numa determinada zona pode ser
sentida como sólida e proporcionar uma boa base, noutra zona pode sentir-se
aperto ou rigidez. As energias e as sensações internas que se movem pelo corpo
podem ser sentidas como agitadas ou vibrantes. Liberte-se das interpretações
mentais relativas às suas causas, bem como de quaisquer reacções relativas ao
facto de serem correctas ou incorrectas. Ao invés, espalhe a consciência
homogeneamente por todo o corpo, com uma intenção de harmonia e estabilidade.
Deixe que essa atitude seja sentida como uma energia que se espalha por todo o
corpo. Isto vai permitir que qualquer contracção se descontraia e vai trazer
luminosidade às áreas mais folgadas ou entorpecidas.

\enlargethispage{\baselineskip}

À medida que tudo se harmoniza, as sensações da respiração vão ficar mais
nítidas, profundas e estáveis. Poderá notar não apenas que a respiração flui até
ao abdómen, como também que esta produz uma sensação subtil de rubor ou de
formigueiro na cara, nas palmas das mãos e no peito. Permaneça com estas
sensações, explorando-as. É provável que a mente vagueie, mas acima de tudo
assegure-se que mantem a intenção no sentido da harmonia e da estabilidade.
Assim, quando se apercebe que a mente vagueou, nesse momento de
consciencialização -- faça uma pausa. Não reaja. Enquanto a mente paira durante
esse momento, introduza a questão: `Como é que eu sei que estou a respirar
agora?' Ou simplesmente: `A respirar?' Entre em sintonia com a sensação que
surge (seja qual for) que lhe indica que se encontra a respirar e siga a próxima
expiração, deixando a mente repousar nessa expiração. Veja se consegue
permanecer com a expiração até à última sensação, entrando depois na pausa que
antecede a inspiração. Depois, siga a inspiração de forma semelhante, até à
última sensação. Desta forma, deixe que o ritmo da respiração dirija a mente --
em vez de impor a ideia do `estar consciente' ao processo natural de respiração.

Explore a forma como experiencia a respiração em diferentes partes do corpo,
começando na barriga. `Como é que a barriga conhece a respiração?' Pode
experienciar isto como uma sensação de dilatação fluida. Permaneça aí durante
alguns minutos, deixando que a mente absorva. Depois, `Como é que o plexo solar
conhece a respiração?' Aqui pode ter uma sensação mais concreta, como um abrir e
fechar. Depois o peito, onde predominam as sensações `aéreas' de dilatação.
Verifique a garganta e a zona entre as sobrancelhas. Repare como a respiração
não constitui um modo único de sensações ou de energias mas que, em termos de
energia, a distinção entre inspiração e expiração é sempre reconhecível.

\enlargethispage{\baselineskip}

Por fim a sua mente vai querer aquietar e centrar-se num ponto do corpo --
permita-lhe escolher o que for mais confortável. Pode ser no peito ou nas
passagens aéreas do nariz, por exemplo. Depois continue a seguir e a estudar a
respiração, como anteriormente. À medida que a mente entra em fusão com a
respiração-energia, espalhe a sua consciência sobre todas as sensações do corpo,
como uma inundação ou instilação. As diferentes sensações da respiração podem
difundir-se e dissolver-se nessa energia. Permita alguma confiança, deixando que
a atenção cognitiva se descontraia e apoiando-se na fruição da energia subtil de
forma a manter-se consciente. Esteja presente, mas não envolvido em tudo o que
surge.

Quando quiser parar, leve a sua atenção de volta para as texturas da carne e
para a firmeza da estrutura óssea. À medida que sente esta presença sustentada,
permita que os seus olhos se abram sem olhar para nada em particular. Ao invés,
deixe que a luz e as formas ganhem configuração por elas próprias.
