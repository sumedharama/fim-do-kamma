\chapterNote{Encontrar"-se com o limiar}

\chapter{Meditação}

\tocChapterNote{Encontrar"-se com o limiar}

Tome consciência de todo o corpo, centrando-a no eixo vertical e na respiração.

À medida que esta qualidade de estar centrado se torna clara, alargue a amplitude da consciência. Expanda a consciência através do corpo e no espaço imediato que o rodeia, tão longe quanto se sentir confortável e sem perder o contacto com o centro. Contemple e disfrute as energias em mutação dentro dessa esfera de consciência.

Irão surgir perturbações. Estas podem estar relacionadas com um som que ouve ou com uma sensação física desagradável. Sinta a sua consciência a engelhar"-se ou contrair"-se no limiar dessa perturbação. Talvez as coisas comecem a acelerar, ou surjam impulsos para superar ou escapar da origem dessa perturbação. Em vez de seguir esses impulsos, reconheça o que se está a passar e descontraia os automatismos que estão a tentar lidar com a perturbação, para continuar assim a alargar gradualmente a esfera da consciência, como se estivesse a abarcar ou mesmo a abraçar a perturbação. Descontraia os contornos do limiar da perturbação e, lenta e silenciosamente, contemple o efeito que isso tem.

Algumas perturbações, a nível mental, irão ocasionalmente ocorrer. Estas podem estar ligadas a outras perturbações sensoriais, tais como a sensação de incómodo associada a um barulho repetitivo na sala do lado. Ou podem ser puramente mentais -- pensamentos sobre coisas que temos de fazer, ou uma memória agradável ou um quebra"-cabeças interessante, que parecem estar a pedir"-nos para nos envolvermos com eles. Por vezes trata"-se de um arrependimento relativo ao passado, ou de uma dúvida sobre meditação. Reconheça qualquer uma destas perturbações em termos de uma ondulação ou de uma agitação, uma alteração de velocidade e de energia. Vá mais devagar e espere na presença dessa perturbação. Não reaja nem esteja com pressa para mudar seja o que for. Ao invés, suavize a sua atitude para com a agitação e tente discerni"-la em termos da sua energia. Encontre"-se com o limiar dessa agitação e amplie a sua consciência sobre ele.

Continue a ampliar e a descansar na esfera da consciência. Deixe que as ondas da perturbação sigam o seu próprio caminho. À medida que tudo se aquieta, sinta e contemple esse efeito, sem palavras.

Quando sentir que é tempo de deixar a meditação, espere -- sinta a energia dessa intenção. Amplie a sua consciência sobre o limiar dessa intenção que surgiu. Contemple e abra"-se seja ao que for que seja revelado.

Regresse ao centro, sentindo a parte interna do corpo e a respiração. Abra"-se ao espaço à sua volta, aos sons e, por fim, ao campo visual.
