\chapter{Glossário}

Na lista de termos Pali apresentada abaixo, as palavras em português usadas são
a tradução das palavras em inglês escolhidas por Ajahn Sucitto para traduzir a
terminologia budista clássica.

\begin{glossarydescription}

% === A ===

\item[anicca] (Pali) Impermanence: one of the \emph{three characteristics of
    existence} along with not-self (\emph{anattā}) and unsatisfactoriness
  (\emph{dukkha}).

% === B ===

\item[borapet] (Thai) Tinospora crispa. Heart-shaped moonseed or guduchi.
  An extremely bitter vine used as a prophylactic and treatment for malaria.

  \emph{adhivacana"-phassa}: contacto"-designação

  \emph{ānāpānasati}: consciência da respiração

  \emph{anusaya}: tendências latentes; obsessões

  \emph{asaṅkhatā:} não"-programado; incondicionado

  \emph{āsava}: extravasamento; corrupção; afluxo; mácula

  \emph{avijjā}: ignorância; não saber

  \emph{ayusaṅkhāra}: força vital

  \emph{bhava:} tornar"-se; devir; ser; existência

  \emph{bojjhaṅgā:} factores do Despertar

  \emph{cetanā}: intenção/volição

  \emph{citta}: mente"-base; mente; coração

  \emph{citta"-saṅkhāra}: programa mental/emocional; formação mental

  \emph{chanda}: motivação; desejo

  \emph{dukkha:} sofrimento e stress; mal"-estar; insatisfação

  \emph{hirī}: estar ciente; vergonha

  \emph{jhāna}: absorção

  \emph{kamma}: acção; causa; carma

  \emph{karunā}: compaixão

  \emph{kaya"-saṅkhāra}: programa físico; fabricação física; formação física

  \emph{khandhā}: amontoados; cinco agregados

  \emph{mano}: mente"-órgão; mente; intelecto

  \emph{mettā}: pura bondade; gentileza, boa"-vontade; amor incondicional

  \emph{muditā}: alegria empática; apreciação

  \emph{nāma}: nome; mentalidade

  \emph{nirodhā}: cessação; parar; terminar

  \emph{ottappa}: ter cuidado; preocupação; medo da culpa

  \emph{papañca:} proliferar; proliferação; difusão; complicação; mundano

  \emph{pārami/pāramita:} perfeições

  \emph{paṭigha"-phassa}: contacto perturbador; sensação de repulsa

  \emph{pīti}: êxtase; deleite

  \emph{puñña:} valor; mérito

  \emph{samatha}: acalmar

  \emph{samādhi}: concentração; recolhimento

  \emph{sampajañña}: consciência plena

  \emph{saṁsāra:} deambular em; vagueio incessante

  \emph{saṅkhāra:} programas; padrões; fabricações; formações; formações mentais;

  \emph{saññā}: percepção; significados sentidos

  \emph{sati}: consciência

  \emph{sukhā}: bem"-estar; felicidade; prazer volitivo

  \emph{taṇhā}: sede (psicológica); anseio

  \emph{upekkhā}: equanimidade

  \emph{vacī"-saṅkhāra}: programa racional; fabricação de pensamento; fabricação verbal

  \emph{vicāra}: avaliação; pensamento continuado; ponderar; considerar

  \emph{viññaṇa}: consciência cognitiva

  \emph{vipāka}: efeito; resultado; kamma antigo

  \emph{vipassanā}: realização interior

  \emph{virāgā}: ausência de paixão; desencanto; desapego; desvanecimento

  \emph{vitakkā:} trazer ao pensamento; pensamento direccionado; pensamento inicial; pensar

  \emph{viveka}: não envolvimento; não"-apego; desapego

  \emph{vossagga:} renúncia; abrir mão; entrega; libertação

  \emph{yoniso manasikārā}: atenção compreensiva; atenção sábia; atenção adequada; atenção sistemática

% === C ===

% === D ===

% === E ===

% === F ===

% === G ===

% === H ===

% === I ===

% === J ===

% === K ===

% === L ===

% === M ===

% === N ===

% === O ===

% === P ===

% === Q ===

% === R ===

% === S ===

% === T ===

% === U ===

% === V ===

% === W ===

\end{glossarydescription}

